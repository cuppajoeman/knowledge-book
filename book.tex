%%%%%%%%%%%%%%%%%%%%%%%%%%%%%%%%%%%%%%%%%%%%%%%%%%%
% Document Setup
%%%%%%%%%%%%%%%%%%%%%%%%%%%%%%%%%%%%%%%%%%%%%%%%%%%

\documentclass[a4paper,11pt,oneside]{book}
\usepackage[T1]{fontenc}
\usepackage[utf8]{inputenc}
\usepackage{lmodern}
\usepackage{geometry}
% \usepackage{showframe}

%\newgeometry{vmargin={15mm}, hmargin={12mm,17mm}}   % set the margins
\geometry{left=2cm}
\geometry{right=2cm}
\setlength{\marginparwidth}{2cm}

\usepackage{knowledge}

\begin{document}

\frontmatter
\maketitle

%%%%%%%%%%%%%%%%%%%%%%%%%%%%%%%%%%%%%%%%%%%%%%%%%%%%%%%%%%%%%%%
% Add a dedication paragraph to dedicate your book to someone %
%%%%%%%%%%%%%%%%%%%%%%%%%%%%%%%%%%%%%%%%%%%%%%%%%%%%%%%%%%%%%%%
\begin{dedication}
Dedicated to the proofs that were left as exercises to the readers
\end{dedication}

%%%%%%%%%%%%%%%%%%%%%%%%%%%%%%%%%%%%%%%%%%%%%%%%%%%%%%%%%%%%%%%%%%%%%%%%
% Auto-generated table of contents, list of figures and list of tables %
%%%%%%%%%%%%%%%%%%%%%%%%%%%%%%%%%%%%%%%%%%%%%%%%%%%%%%%%%%%%%%%%%%%%%%%%
\tableofcontents
\listoffigures
\listoftables
%\listoftodos

\mainmatter

%%%%%%%%%%%
% Preface %
%%%%%%%%%%%

\chapter*{Preface}
This book contains knowledge that that me or my peers have obtained, the purpose is to explain things fundamentally and in full detail so that someone who has never touched the subject may be able to understand it. It will focus on conveying the ideas that are involved in synthesizing the new knowledge with less of a focus on the results themselves.


\section*{Structure of book}
The book is partitioned into different sections based on the domain it is involved with. There may be shared definitions and theorems throughout the chapters, but in general it will start more elementary and get more advanced.

\section*{Knowledge}
In this book you will find many results, will will characterize them as being one of the following
\begin{itemize}
    \item Theorems - Results that are of importantance and who's proof is not easily found (maybe using a novel idea)
  \item Propositions - Results of less importance who's proof could be constructed without a novel idea
  \item Lemmas - Results that are technical intermediate steps which has no standing as an independent result on first observation \footnote[2]{But sometimes they escape, as their usage becomes more than just an intermediate step,  as Zorn's or Fatou's Lemmas did}
    \item Corollaries - Results which follow readily  from an existing result of greater importance
\end{itemize}

\section*{Recommendations}
By now you might know that in order to actually get better at mathematics you have to engage with it. This book may be used as a reference at times, but I highly recommend trying to re-prove statements or coming up with your own ideas before instantly looking at the solutions.

\section*{About the companion website}
The website\footnote{\url{https://github.com/cuppajoeman/knowledge-book}} for this file contains:
\begin{itemize}
  \item A link to (freely downlodable) latest version of this document.
  \item Link to download LaTeX source for this document.
\end{itemize}

\section*{Acknowledgements}
\begin{itemize}
    \item A special word of thanks to professors who wanted to make sure I understood and learned as much as possible Alfonso Gracia-Saz\footnote{\url{https://www.math.toronto.edu/cms/alfonso-memorial/}}, Jean-Baptiste Campesato\footnote{\url{https://math.univ-angers.fr/~campesato/}}, Z-Module, riv, PlanckWalk, franciman, qergle from \#math on \url{https://libera.chat/}.
\end{itemize}

\chapter*{Contributing}
 
Contributions to the project are very welcome, let's delve into how to get started with this.

If you want to contribute to the project it's most likely that a contribution will fall into one of the following categories
\begin{itemize}
  \item Content Based 
  \begin{itemize}
      \item Adding Definitions, Theorems, \ldots
      \item Finishing TODO's
      \item Formatting of the book
  \end{itemize}
  \item Structural Layout of Project 
  \begin{itemize}
      \item Organization
      \item Simplyfing the existing structure of the directories 
      \item Making scripts which set up new structures
  \end{itemize}
  \item External
  \begin{itemize}
      \item Adding explanatory content to help onboard new users
      \item Getting others involved
      \item Creating infrastructure to support users (Github discussions)
  \end{itemize}
\end{itemize}

\section*{Content Based}

If you're looking to add content to the project \todo{finish this}

\chapter{Linear Algebra}

\section{Vectors}

\begin{definition}{Algebraic Dot Product}{algebraic_dot_product}
 Let $ u_{1} , u_{2} , \dotsc  , u_{n - 1} , u_{n}$ and $ v_{1} , v_{2} , \dotsc  , v_{n - 1} , v_{n}$ denote the components of $ \vec{u}$ and $ \vec{v}$ respectively, then we have:
\[
	\vec{v}  \cdot  \vec{u} \stackrel{\mathtt{D}}{=} \sum_{i=1}^{n} v_{i} u_{i}
\]
\end{definition}


\input{linear_algebra/definitions/geometric_dot_product}


\section{Matrices}

\begin{definition}{Matrix Multiplication}{matrix_multiplication}
    If $\mathbf{A}$ is an $m \times n$ matrix and $\mathbf{B}$ is an $n \times p$ matrix,
    \[
    \mathbf{A}=\left(\begin{array}{cccc}
        a_{11} & a_{12} & \cdots & a_{1 n} \\
        a_{21} & a_{22} & \cdots & a_{2 n} \\
        \vdots & \vdots & \ddots & \vdots \\
        a_{m 1} & a_{m 2} & \cdots & a_{m n}
        \end{array}\right), \quad \mathbf{B}=\left(\begin{array}{cccc}
        b_{11} & b_{12} & \cdots & b_{1 p} \\
        b_{21} & b_{22} & \cdots & b_{2 p} \\
        \vdots & \vdots & \ddots & \vdots \\
        b_{n 1} & b_{n 2} & \cdots & b_{n p}
    \end{array}\right)
    \]
    the matrix product $\mathbf{C}=\mathbf{A B}$ (denoted without multiplication signs or dots) is defined to be the $m \times p$ matrix 
    \[
    \mathbf{C}=\left(\begin{array}{cccc}
    c_{11} & c_{12} & \cdots & c_{1 p} \\
    c_{21} & c_{22} & \cdots & c_{2 p} \\
    \vdots & \vdots & \ddots & \vdots \\
    c_{m 1} & c_{m 2} & \cdots & c_{m p}
    \end{array}\right)
    \]
    such that
    \[
    c_{i j}=a_{i 1} b_{1 j}+a_{i 2} b_{2 j}+\cdots+a_{i n} b_{n j}=\sum_{k=1}^{n} a_{i k} b_{k j}
    \]
    for $i=1, \ldots, m$ and $j=1, \ldots, p$.
\end{definition}


\lstinputlisting[language=Python]{linear_algebra/programs/matrix_multiplication.py}

\chapter{First Order Logic}

\section{Languages}

\begin{definition}{First-Order Language}{first_order_language}
    A first order language $\mathcal{L}$ is an infinite collection of distinct
    symbols, no one of which is properly contained in another, sparated into the
    following cateogories:
	\begin{enumerate}
		\item Parentheses: (,).
		\item Connectives: $\lor, \neg$.
		\item Quantifier: $\forall$.
        \item Variables, one for each positive integer n: $v_{1}, v_{2}, \ldots,
        v_{n}, \ldots$ The set of variable symbols will be denoted $ \mathcal{ V
        }   $ .
        \item Equality symbol: $=$.
        \item Constant symbols: Some set of zero or more symbols. This set will
        be denoted as $ \mathcal{ C }   $ 
        \item Function symbols: For each positive integer $n$, some set of zero
        or more $n$-ary function symbols as $ \mathcal{ F }  $ .
        \item Relation symbols: For each positive integer $n$, some set of zero
        or more $n$-ary relation symbols as $ \mathcal{ R }  $ .
	\end{enumerate}
	Notice that the requirement that no symbol be properly contained within
	another requires that $ \mathcal{ C } , \mathcal{ F } , \mathcal{ R }     $
	all be disjoint.
\end{definition}
We may denote a language $ \mathcal{ L }   $ as $ \left( \mathcal{ C } ,
\mathcal{ F } , \mathcal{ R }      \right)  $ as these are the only parts of a
the language which may vary (note that we may have different variables, but
their usage and the fact that there are a countably infinite amount of them
makes it so that we don't have to specify them). Also note that we may construct
string, or ordered tuples of elements from a language, so for example we may
write $ ``( v _{ 1 } = \lor"  $, and say that $ ``(" \in ``( v _{ 1 } = \lor"
$, when we want to reference a string we use an equality symbol and a variable
which aren't in the language so we may write $ \gamma :\equiv ( v _{ 1 } = \lor
$ and say that $ ( \in  \gamma $ (notice that we dropped the use of quotations
as it should be clear now that these are strings).


\begin{definition}{A Term's Variable Replaced by a Term}{a_term's_variable_replaced_by_a_term}
    Suppose that $u$ is a term, $x$ is a variable, and $t$ is a term. We define the term $u_{t}^{x}$ (read ``$u$ with $x$ replaced by $\left.t "\right)$ as follows:
    \begin{enumerate}
        \item If $u$ is a variable not equal to $x$, then $u_{t}^{x}$ is $u$.
        \item If $u$ is $x$, then $u_{t}^{x}$ is $t$.
        \item If $u$ is a constant symbol, then $u_{t}^{x}$ is $u$.
        \item If $u: \equiv f u_{1} u_{2} \ldots u_{n}$, where $f$ is an $n$-ary function symbol and the $u_{i}$ are terms, then $u_{t}^{x}$ is $f\left(u_{1}\right)_{t}^{x}\left(u_{2}\right)_{t}^{x} \ldots\left(u_{n}\right)_{t}^{x}$
    \end{enumerate}
\end{definition}


\begin{itemize}
    \item  In the fourth clause of the definition above and in the first two clauses of the next definition, the parentheses are not really there. Because $u_{1}{ }_{t}^{x}$ is hard to read so the parentheses have been added in the interest of readability.
    \item If we let $t$ be $g(c)$ and we let $u$ be $f(x, y)+h(z, x, g(x))$, then $u_{t}^{x}$ is
        \[
        f(g(c), y)+h(z, g(c), g(g(c)))
        \]
\end{itemize}


\input{logic/definitions/a_term_is_substitutable_for_a_variable.tex}

To understand the motivation behind the fourth clause, consider the formula:
\[
\phi :\equiv \left( \forall x \right) \left( \forall y \right) x = y
\]

Then one might want to say that $ \left( \left( \forall y \right) x = y \right) _{ t }^{ x }   $  where $ t $ is any term

\begin{definition}{Free Variable in a Formula}{free_variable_in_a_formula}
    Suppose that $v$ is a variable and $\phi$ is a formula. We will say that $v$ is free in $\phi$ if
    \begin{enumerate}
        \item $\phi$ is atomic and $v$ occurs in (is a symbol in) $\phi$, or                               
        \item $\phi: \equiv(\neg \alpha)$ and $v$ is free in $\alpha$, or
        \item $\phi: \equiv(\alpha \vee \beta)$ and $v$ is free in at least one of $\alpha$ or $\beta$, or
        \item $\phi: \equiv(\forall u)(\alpha)$ and $v$ is not $u$ and $v$ is free in $\alpha$.
    \end{enumerate}
\end{definition}


\subsubsection*{Examples}
\begin{itemize}
    \item Thus, if we look at the formula
    $$
    \forall v_{2} \neg\left(\forall v_{3}\right)\left(v_{1}=S\left(v_{2}\right) \vee v_{3}=v_{2}\right)
    $$
    the variable $v_{1}$ is free whereas the variables $v_{2}$ and $v_{3}$ are not free. 
    \item A slightly more complicated example is
        \[
        \left(\forall v_{1} \forall v_{2}\left(v_{1}+v_{2}=0\right)\right) \vee v_{1}=S(0)
        \]
        \begin{itemize}
            \item In this formula, $v_{1}$ is free whereas $v_{2}$ is not free. Especially when a formula is presented informally, you must be careful about the scope of the quantifiers and the placement of parentheses.
        \end{itemize}
\end{itemize}

\subsubsection*{Notes}
\begin{itemize}
    \item We will have occasion to use the informal notation $\forall x \phi(x)$. This will mean that $\phi$ is a formula and $x$ is among the free variables of $\phi$. 
    \item If we then write $\phi(t)$, where $t$ is an $\mathcal{L}$-term, that will denote the formula obtained by taking $\phi$ and replacing each occurrence of the variable $x$ with the term $t$. 
\end{itemize}

\begin{definition}{L-Structure}{l-structure}
Fix a language $\mathcal{L}$. An $\mathcal{L}$-structure $\mathfrak{A}$ is a nonempty set $A$, called the universe of $\mathfrak{A}$, together with:
\begin{enumerate}
    \item For each constant symbol $c$ of $\mathcal{L}$, an element $c^{\mathfrak{A}}$ of $A$,
    \item For each $n$-ary function symbol $f$ of $\mathcal{L}$, a function $f^{\mathfrak{A}}: A^{n} \rightarrow A$, and
    \item For each $n$-ary relation symbol $R$ of $\mathcal{L}$, an $n$-ary relation $R^{\mathfrak{A}}$ on $A$ (i.e., a subset of $A^{n}$ ).
\end{enumerate}
We may denote an $\mathcal{L}$-Structure as follows
\[
\left( A, \mathcal{ C } ^{  \mathfrak{ A }   }  \stackrel{\mathtt{D}}{=} \left\{ c ^{ \mathfrak{ A }   } : c \in \mathcal{ C }   \right\}, \mathcal{ F } ^{ \mathfrak{ A }   } = \left\{ f ^{ \mathfrak{ A }   } : f \in  \mathcal{ F }   \right\}, \mathcal{ R } ^{ \mathfrak{ A }   } = \left\{ R ^{ \mathfrak{ A }   } : R \in  \mathcal{ R }   \right\}        \right) 
\]
\end{definition}


\begin{definition}{Restricted L-Structure}{restricted_l-structure}
    Suppose $ \mathcal{ L } =  \left( \mathcal{ V } , \mathcal{ C } , \mathcal{ F } , \mathcal{ R }       \right)    $ and $ \mathcal{ L } ^{ \prime  } =  \left( \mathcal{ V } ^{ \prime  }  , \mathcal{ C } ^{ \prime  }  , \mathcal{ F } ^{  \prime  }  , \mathcal{ R } ^{ \prime  }        \right)   $  are languages and $ \mathfrak{A} ^{ \prime  } =\left(A ^{ \prime  } , \mathcal{ C ^{ \prime  }  } ^{ \mathfrak{ A } ^{\prime} }  , \mathcal{ F ^{ \prime  }  } ^{ \mathfrak{ A } ^{\prime}   },\mathcal{ R ^{ \prime  }  } ^{ \mathfrak{ A } ^{\prime}   }         \right)   $ be an $\mathcal{L} ^{  \prime  } $-Structure, then we define 
    \[
        \mathfrak{A} ^{ \prime  } \upharpoonright_{\mathcal{L}} \stackrel{\mathtt{D}}{=} \left( A ^{ \prime  } ,  \left\{ c ^{ \mathfrak{ A } ^{ \prime  }    } : c \in  \mathcal{ C }   \right\}, \left\{ f ^{ \mathfrak{ A } ^{ \prime  }    } : f \in  \mathcal{ F }   \right\}, \left\{ R ^{ \mathfrak{ A } ^{ \prime  }    } : R \in  \mathcal{ R }   \right\}     \right)  
    \]
    In other words we have just created an $\mathcal{L}$-Structure $ \mathfrak{ A } ^{ \prime  } \upharpoonright_{\mathcal{L}} $ which intereprets everything the same was as $ \mathfrak{ A } ^{ \prime  }   $ but it is now defined for $ \mathcal{ L }   $ .
\end{definition}



\section{Deductions}
\input{logic/definitions/deduction_of_a_formula}

\begin{definition}{Logical Axioms}{logical_axioms}
 Given a first order language $ \mathcal{ L }   $ the set $\Lambda$ of logical
 axioms is the collection of all $ \mathcal{ L }-$formulas that fall into one of
 the following categories:
\begin{gather*}
    x=x  ~\text{for each variable}~ x \\
    \left[ \left( x _{ 1 } = y _{ 2 }  \right) \land \left( x _{ 2 } = y _{
    2}\right) \land \ldots \land \left( x _{ n } =  y _{ n }    \right)
    \right] \rightarrow \\
    \qquad \qquad \qquad \left( f\left( x_{1} , x_{2} , \dotsc , x_{n} \right) = f\left( y_{1} ,
    y_{2} , \dotsc , y_{n} \right)  \right) \\
    \left[\left(x_{1}=y_{1}\right) \wedge\left(x_{2}=y_{2}\right) \wedge \cdots
    \left(x_{n}=y_{n}\right)\right] \rightarrow \\
    \qquad \qquad \qquad \left(R\left(x_{1}, x_{2}, \ldots, x_{n}\right) \rightarrow
    R\left(y_{1}, y_{2}, \ldots, y_{n}\right)\right)\\
    \left( \forall x \phi  \right) \rightarrow \phi _{ t }^{ x } ~\text{if $ t $
    is substitutable for $ x $ in $ \phi  $ }~  \\
    \phi _{ t }^{ x } \to \left( \exists x \phi \right) ~\text{if $ t $ is
    substitutable for $ x $ in $ \phi  $ }~  
\end{gather*}
To refer to them easily we label them by  moving down the above list E1, E2, E3,
Q1, Q2 
\end{definition}


\begin{lemma}{Universal connection to Variable Assignment Function}{forall vaf}
    \[
    \Sigma \vdash \theta \text { if and only if } \Sigma \vdash \forall x \theta
    \]
\end{lemma}

%Note this lemma might seem quite strange, but note it actually makes sense, %todo{finish why}

\begin{proof}
    \begin{itemize}
        \item $ \Rightarrow $ 
        \begin{itemize}
            \item Suppose that $ \Sigma \vdash \theta$, therefore we have a deduction $ \mathcal{D}$ of $ \theta$, then the proof
                \begin{gather*}
                    \mathcal{D}\\
                    \left[ \left( \forall y \left( y =  y \right) \right) \land  \neg \left( \forall y \left( y =  y \right) \right) \right] \rightarrow \theta \tag{taut. PC}\\
                    \left[ \left( \forall y \left( y =  y \right) \right) \land  \neg \left( \forall y \left( y =  y \right) \right) \right] \rightarrow \left(  \forall  x \right)\theta \tag{QR}\\
                    \left(  \forall x \right) \theta \tag{PC}
                \end{gather*}
        \end{itemize}
        \item $ \Leftarrow $
        \begin{itemize}
            \item Suppose that $ \Sigma \vdash \forall x \theta$, so we have a deduction of it, call it  $\mathcal{D}$, then the following deduction suffices
            \begin{gather*}
               \mathcal{D} \\
               \forall x \theta\\
               \forall x \theta \rightarrow \theta _{x}^{x}\\
                \theta _{x}^{x}
            \end{gather*}
        \end{itemize}
    \end{itemize}
\end{proof}


\section{Completeness}

\begin{definition}{Consistent Set of L Formulas}{consistent_set_of_l_formulas}
Let $\Sigma$ be a set of $\mathcal{L}$-formulas. We will say that $\Sigma$ is inconsistent if there is a deduction from $\Sigma$ of $[(\forall x) x=x] \wedge \neg[(\forall x) x=x]$. We say that $\Sigma$ is consistent if it is not inconsistent.
\end{definition}


\begin{proposition}{Contradiction has no Model}{contradiction_has_no_model}
    \begin{center}
        There is no $\mathcal{L}$-Structure $ \mathfrak{ A }   $ such that $ \mathfrak{ A } \models \bot $ 
    \end{center}
\end{proposition}
\begin{proof}
    Suppose there was a model of $ \bot $, that would mean that we have an $\mathcal{L}$-Structure $ \mathfrak{ A }     $ such that $ \mathfrak{ A } \models \bot  $. Recall that $ \bot :\equiv \left[ \left( \forall x \right) x =  x \right] \land \neg \left[ \left( \forall x \right)x =  x \right] $, so then we would have to have $ \mathfrak{ A } \models  \left( \forall x \right)x =  x   $ and also not have $ \mathfrak{ A } \models \left( \forall x \right) x =  x  $ which is a contradiction.
\end{proof}


\begin{proposition}
{Implying a Contradiction Means no Model}{implying_a_contradiction_means_no_model}
Let \(\Sigma \) be a set of sentences, then
\[
\Sigma \models \bot
\]
if and only if \(\Sigma \) has no model
\end{proposition}
\begin{proof}
    \begin{itemize}
        \item Recall that \(\Sigma \models \bot \) means that for any
            \(\mathcal{L}\)-Structure \(\mathfrak{A} \) we have that
            \[
            \mathfrak{A} \models \Sigma \enspace \text{implies} \enspace
            \mathfrak{A} \models \bot
            \]
            but recall that \(\mathfrak{A} \models \bot \) is always false,
            therefore for the implication to hold we require that \(\mathfrak{A}
            \models \Sigma \) to be false, which means \(\Sigma \) has no model
            as \(\mathfrak{A} \) was arbitrary.
        \item Suppose that \(\Sigma \) has no model, then that means for any
        \(\mathcal{L}\)-Structure \(\mathfrak{A} \) we have that \(\mathfrak{A}
        \not \models \Sigma \), so \( \Sigma \models \bot  \) vacuously holds.
    \end{itemize}
\end{proof}


\begin{proposition}{Logical Contradiction}{logical_contradiction}
    Let $ \Sigma $ be a set of sentences, and let $ \phi  $ be a sentence, then
    \[
    \Sigma \models \phi ~\text{implies}~ \Sigma \cup \left\{ \neg \phi  \right\} \models \bot  
    \]
\end{proposition}
\begin{proof}
    \begin{itemize}
        \item Suppose that $ \Sigma \models \phi  $ that means for any $\mathcal{L}$-Structure $ \mathfrak{ A }   $ if $ \mathfrak{ A } \models \Sigma   $ then $ \mathfrak{ A } \models \phi   $ we want to show that $ \Sigma \cup \left\{ \neg \phi  \right\} \models \bot  $, recall that by the previous theorem that means that $ \Sigma \cup \left\{ \neg \phi  \right\}  $ has no model.
        \item Suppose for the sake of contradiction that $ \Sigma \cup \left\{ \neg \phi  \right\}  $ has a model $ \mathfrak{ B }   $ then certainly $ \mathfrak{ B } \models \Sigma   $ and by assuption we have that $ \mathfrak{ B } \models \phi   $ but also we know that $ \mathfrak{ B } \models \neg \phi   $ which is a contradiction, so therefore $ \Sigma  \cup \left\{ \neg \phi  \right\}  $ has no model and therefore
            \[
            \Sigma \cup \left\{ \neg \phi  \right\} \models \bot 
            \]
            as needed.
    \end{itemize}
\end{proof}


\begin{lemma}{Constant Extension still Consistent}{constant_extension_still_consistent}
If $\Sigma$ is a consistent set of $\mathcal{L}$-sentences and $\mathcal{L}^{\prime}$ is an extension by constants of $\mathcal{L}$, then $\Sigma$ is consistent when viewed as a set of $\mathcal{L}^{\prime}$-sentences.
\end{lemma}
\begin{proof}
    \begin{itemize}
        \item Suppose for the sake of contradiction that $ \Sigma  $ is not consistent when viewed as a set of $\mathcal{L}^{\prime}$-sentences, so $ \Sigma \vdash \bot $, considering the set of all deductions of $ \bot  $ from $ \Sigma  $ we may find a deduction $ \mathcal{ D }   $  which uses the least number of the newly added constants by the well ordering principle, let this number be $ n \in  \mathbb{N} $ and that $ n > 0 $ or else we would have a deduction from $ \mathcal{ L }   $ of $ \bot  $ which would be a contradiction.
        \item Let $ v $ be a variable that isn't used in $ \mathcal{ D }   $, note that we can do this because there are infinitely many variables, but only finitely many of them can be used in a deduction and let $ c $ be one of the newly added constants which is used in $ \mathcal{ D }   $ and let $ \mathcal{ D } _{ v } ^{ c }    $ be created where for each line $ \phi \in  \mathcal{ D }  $ we replace it with $ \phi _{ v }^{ c } $ , and note that the last line of $ \mathcal{ D } _{ v }   $ is still $ \bot  $, at this point we don't know if $ \mathcal{ D } _{ v }   $ is a deduction, so we have to check that it is.
        \begin{itemize}
            \item Note that if it is a deduction, then we've arrived at a contradiction since we have a deduction of $ \bot  $ with a number of newly added constants which is less than $ n $ (and $ n $ is minimal).
        \end{itemize}
        \item If $ \phi  $ is an equality axiom or an element of $ \Sigma $ then $ \phi _{ v } :\equiv \phi  $ because equality axioms only contain variables and not constants, also $ \Sigma  $ is a set of $ \mathcal{ L }   $ sentences and so it can't contain any of the new constants. Therefore equality axioms and any sentence from $ \Sigma  $ is not modified by this procedure, leaving them as valid steps in the deduction.
        \item If $ \phi  $ is $ \left( \forall x \right) \theta \rightarrow \theta _{ t }^{ x }  $ then $ \phi _{ v }  $ is $ \left( \forall x \right)\theta _{ v } \rightarrow \left( \theta _{ v }  \right) _{ t _{ v }  }^{ x  }  $, to see why we use $ t _{ v }  $ as well as $ \theta _{ v }  $ try $ \theta :\equiv c =  x $ and $ t :\equiv   c + 3 $ 
    \end{itemize}
\end{proof}



Notice that if we write $ \Sigma \models \bot $ it means that for any $\mathcal{L}$-Structure $ \mathfrak{ A }   $ if $ \mathfrak{ A } \models \Sigma   $ then $ \mathfrak{ A } \models \bot  $ by the above discussion that forces $ \mathfrak{ A } \models \Sigma   $ to be false, and therefore $ \Sigma  $ has no model.


\begin{theorem}{Completeness Theorem}{completeness}
Suppose that $\Sigma$ is a set of $\mathcal{L}$-formulas, where $ \mathcal{L}$ is a countable langauge  and $\phi$ is an $\mathcal{L}$-formula. If $\Sigma \models \phi$, then $\Sigma \vdash \phi$.

\section*{Setup}

\begin{itemize}
    \item We start by assuming that $ \Sigma \models \phi$, we must show that $ \Sigma \vdash \phi$.
    \item If $ \phi$ is not a sentence then we can always prove $ \phi'$ which is the same as $ \phi$ with all of it's variables bound
    \begin{itemize}
        \item We can do that by appending $ \left( \forall  x _{f}  \right)$ where each $ x_{f}$  is a free varaible of $ \phi$ to the front of $ \phi$ 
    \end{itemize}
\item Therefore we will prove it for all sentences $ \phi$ %\todo[inline]{justify why this is equivalent}
\end{itemize}

\end{theorem}


\section{Incompleteness}

\begin{definition}
{Representable Set}{representable_set}
A set \(A \subseteq \mathbb{N}^{k}\) is said to be representable (in \(N\)) if
there is an \(\mathcal{L}_{N T}\)-formula \(\phi(\undervec{x})\) such that
\[
\begin{array}
{ll}
\forall \undervec{a} \in A & N \vdash \phi(\overline{\undervec{a}}) \\
\forall \undervec{b} \notin A & N \vdash \neg \phi(\overline{\undervec{b}})
\end{array}
\]
In this case we will say that the formula \(\phi\) represents the set \(A\).
\end{definition}

\begin{definition}{Weakly Representable Set}{weakly_representable_set}
A set $A \subseteq \mathbb{N}^{k}$ is said to be weakly representable (in $N$ ) if there is an $\mathcal{L}_{N T}$-formula $\phi(x)$ such that
$$
\begin{array}{ll}
\forall a \in A & N \vdash \phi(\bar{a}) \\
\forall b \notin A & N \nvdash \phi(\bar{b})
\end{array}
$$
In this case we will say that the formula $\phi$ weakly represents the set $A$.
\end{definition}

\begin{definition}{Total Function}{total_function}
Suppose that $A \subseteq \mathbb{N}^{k}$ and suppose that $f: A \rightarrow \mathbb{N}$. If $A=\mathbb{N}^{k}$ we will say that $f$ is a total function
\end{definition}

\begin{definition}{Partial Function}{partial_function}
Suppose that $A \subsetneq \mathbb{N}^{k}$ and suppose that $f: A \rightarrow \mathbb{N}$, then we will call $f$ a partial function.
\end{definition}

\begin{definition}{Representable Function}{representable_function}
    Suppose that $f: \mathbb{N}^{k} \rightarrow \mathbb{N}$ is a total function. We will say that $f$ is a representable function (in $N$ ) if there is an $\mathcal{L}_{N T}$ formula $\phi\left(x_{1}, \ldots, x_{k+1}\right)$ such that, for all $a_{1}, a_{2}, \ldots a_{k+1} \in \mathbb{N}$
    \begin{itemize}
        \item If $f\left(a_{1}, \ldots, a_{k}\right)=a_{k+1}$, then $N \vdash \phi\left(\overline{a_{1}}, \ldots, \overline{a_{k+1}}\right)$
        \item If $f\left(a_{1}, \ldots, a_{k}\right) \neq a_{k+1}$, then $N \vdash \neg \phi\left(\overline{a_{1}}, \ldots, \overline{a_{k+1}}\right)$
    \end{itemize}
\end{definition}

\begin{definition}{Definable Set}{definable_set}
We will say that a set $A \subseteq \mathbb{N}^{k}$ is definable if there is a formula $\phi(x)$ such that
$$
\begin{array}{ll}
\forall a \in A & \mathfrak{N} \models \phi(\bar{a}) \\
\forall \underline{\downarrow b} \notin A & \mathfrak{N} \models \neg \phi(\bar{b})
\end{array}
$$
In this case, we will say that $\phi$ defines the set $A$.
\end{definition}

\begin{corollary}
    If $ A \subseteq \mathbb{N} ^{ k }  $ is definable by a $ \Delta  $ formula, then it is representable
\end{corollary}



\chapter{Topology}

\section{Topological Spaces and Continuous Functions}

\begin{definition}
{Topology on a Set}{topology_on_a_set}
Let \(X\) be a set. \(A\) collection \(\mathcal{T} \subseteq \mathcal{P}(X)\) of
subsets of \(X\) is called a topology on \(X\) provided that the
following three properties are satisfied:
\begin{enumerate}
    \item \(\emptyset \in \mathcal{T}\) and \(X \in \mathcal{T}\).
    \item \(\mathcal{T}\) is closed under finite intersections. That is, given
    any finite collection \(U_{1}, \ldots, U_{n}\) of sets in \(\mathcal{T}\),
    their common intersection \(U_{1} \cap \cdots \cap U_{n}\) is also an
    element of \(\mathcal{T}\).
    \item \(\mathcal{T}\) is closed under arbitrary unions. That is, if
    \(\left\{U_{\alpha}: \alpha \in I\right\}\) is a family of sets in
    \(\mathcal{T}\) (here \(I\) is some indexing set, which may be infinite),
    then their union \(\bigcup_{\alpha \in I} U_{\alpha}\) is also an element of
    \(\mathcal{T}\).
\end{enumerate}
\end{definition}


\begin{definition}{Open Subset}{open_subset}
The elements \( U \in  \mathcal{ T }   \) of a topology on \( X \) are called
open subsets of \( X \) or just ``open sets''.
\end{definition}


\begin{definition}
{Topological Space}{topological_space}
Given a set \(X\) and a topology \(\mathcal{T}\) on \(X\), the pair \((X,
\mathcal{T})\) is called a topological space. 
\end{definition}


\begin{proposition}{Open iff every point is in another Open
Set}{open_iff_every_point_is_in_another_open_set}
Given a set \( X \) and a topology \( \mathcal{ T } _{ X }   \), \( U \in
\mathcal{ T } _{ X }   \) if and only if 
\[
\forall x \in  U, \exists V \in  \mathcal{ T } _{ X } \enspace \text{such that}
\enspace x \in  V \subseteq U
\]
\end{proposition}
\begin{proof}
\( \Rightarrow  \) Assuming that \( U \) is open, we let \( x \in  U \) and take
\( V = U \) and surely \( x \in U \subseteq U \), as needed.\\
\( \Leftarrow  \) Suppose the converse, we'd like to show that \( U \) is open,
but note that for each \( x \in  U \) we have \( V _{ x } \in  \mathcal{ T } _{
X}   \) such that \( x \in V _{ x } \subseteq U \), thus \( \bigcup _{ x \in U }
V _{ x } \subseteq U \) but additionally every \( x \in U \) is also in the
union, therefore \( U = \bigcup _{ x \in  U } V _{ x }  \), that is to say that
\( U \) is an arbitrary union of open sets, thus it is open as well by the
definition of a topology.
\end{proof}


\subsection{Basis}

\begin{definition}
{Basis For a Set}{basis_for_a_set}
Let \(X\) be a set. A collection of sets \(\mathcal{B} \subseteq
\mathcal{P}(X)\) is called a basis on \(X\) if the following two properties
hold:
\begin{enumerate}
    \item \(\mathcal{B}\) covers \(X\). That is: \(\forall x \in X, \exists B
    \in \mathcal{B}\) such that \(x \in B \). Or, more concisely, \(X = \bigcup
    \mathcal{B}\).
        \begin{itemize}
            \item The reason why \(X = \bigcup \mathcal{B} \) is that \(\bigcup
            _{x \in X} B_{x} \) contains every \(x \in X \) and is a
            subset of \(X \) since each \(B_{x} \) is a subset of \(X \)
            therefore \(\bigcup _{x \in X} B_{x} = X \)
        \end{itemize}
    \item \(\forall B_{1}, B_{2} \in \mathcal{B}, \forall x \in B_{1} \cap
    B_{2}, \exists B \in \mathcal{B}\) such that \(x \in B \subseteq B_{1} \cap
    B_{2}\).
    \begin{itemize}
        \item Given a point \(x\) in the intersection of two elements of the
        basis, there is some element of the basis containing \(x\) and contained
        in this intersection.
    \end{itemize}
\end{enumerate}
We will call the elements of \(\mathcal{B} \) basis elements.
\end{definition}


\begin{definition}
{Topology Generated by a Basis (Basis
    Existance)}{topology_generated_by_a_basis_(basis_existance)}
    Let \(X\) be a set and \(\mathcal{B}\) a basis on \(X\)
    \[
    \mathcal{T}_{\mathcal{B}} = \left\{U \subseteq X: \forall x \in U, \exists B \in
    \mathcal{B} \text{such that} x \in B \subseteq U \right\}
    \]
\end{definition}


\begin{corollary}
{Basis is a subset of the Topology it Generates}{basis_is_a_subset_of_the_topology_it_generates}
Let \(\mathcal{B} \) be a basis for a set \(X \), then
\[
\mathcal{B} \subseteq \mathcal{T} _{\mathcal{B}}
\]
\end{corollary}
\begin{proof}
   Let \(B \in \mathcal{B} \) then we note that \(B \subseteq X \) since
   \(\mathcal{B} \subseteq \mathcal{P} \left(X\right) \). Additionally
   for every \(x \in B \) \(B \) itself is an element from \(\mathcal{B}
   \) such that \(x \in B \subseteq B \), so \(B \in \mathcal{T} _{
   \mathcal{B}} \)
\end{proof}


\input{topology/lemmas/intersection_of_two_elements_from_the_topology_generated_by_a_basis_(basis_existance)_is_closed}

\input{topology/propositions/topology_generated_by_a_basis_(basis_existance)_is_a_topology}

\begin{definition}
{Topology Generated by a Basis
    (Union)}{topology_generated_by_a_basis_(union)}
    Let \(X\) be a set and \(\mathcal{B}\) a basis on \(X\), we define:
    \[
    \mathcal{T}_{\mathcal{B}} = \left\{\bigcup \mathcal{C}: \mathcal{C}
    \subseteq \mathcal{B} \right\}
    \]
    and say that \(\mathcal{T} _{B}\) is called the topology generated by \(
    \mathcal{B}\), note that \(\mathcal{C} \)'s elements are subsets of \(
    X \) (basis elements)
\end{definition}


\input{topology/propositions/topology_generated_by_a_basis_(union)_is_a_topology}

\begin{proposition}
{Generating by Basis Existance or Union Yields Same
Topology}{generating_by_basis_existance_or_union_yields_same_topology}
\[
\left\{U \subseteq X: \forall x \in U, \exists B \in \mathcal{B}
~\text{such that} x \in B \subseteq U \right\} = \left\{\bigcup \mathcal{C} :
\mathcal{C} \subseteq \mathcal{B} \right\}
\]
\end{proposition}
\begin{proof}
    \begin{itemize}
        \item \(\subseteq \) Let \(U \) be an element of the left hand side,
        we'd like to show it's an element of the right hand side. For each point
        \( x \in  U \) we have \( B _{ x } \in \mathcal{ B }   \) such that \( x
        \in B _{ x } \subseteq U\), therefore \( U = \bigcup _{ x \in  U } B _{
        x}  \) and so \( U \) is an element of the right hand side.
        \item \(\supseteq \) Suppose that \(\mathcal{C} \subseteq \mathcal{B} \)
        then for any \(C \in \mathcal{C} \) we know that \(C \in \mathcal{
        B} \) (it's a basis element) and therefore we know that \(C \) is an
        element of the left hand side since we can take \(B = C \). Then since
        the left hand side is a topology \(\bigcup \mathcal{C} \) is also a
        part of the left hand side as it's an arbitrary union.
    \end{itemize}
\end{proof}


\begin{lemma}
{Basis Criterion}{basis_criterion}
Let \(X\) be a topological space. Suppose that \(\mathcal{C}\) is a collection
of open sets of \(X\) such that for each open set \(U\) of \(X\) and each \(x\)
in \(U\), there is an element \(C\) of \(\mathcal{C}\) such that \(x \in C
\subset U\). Then \(\mathcal{C}\) is a basis for the topology of \(X\).
\end{lemma}
\begin{proof}
    \begin{itemize}
        \item We'll show it's a basis, so let \(x \in X \) now since \(X \)
        is open we know that there is \(C \in \mathcal{C} \) such that
        \(x \in C \subseteq X \), as needed.
        \item We continue to the second condition, so we take \(C _{1}, C _{
        2} \in \mathcal{C} \) and then let \(x \in C _{1} \cap C _{2}
        \) (we are allowed to do that, because if \(C _{1} \cap C _{2} =
        \varnothing \) then the statement holds vacuously). Since \(\mathcal{C
} \) is a collection of open sets we know that \(C _{1} \cap C _{
        2} \) is also open with respect to \(X \), thus by assumption we have \(C
        _{3} \) such that \(x \in C _{3} \subseteq C _{1} \cap C _{2} \)
        \item The statement also claims that \(\mathcal{T} _{\mathcal{C}
} = \mathcal{T} \)where \(\mathcal{T} \) is the topology of \(X \)
        \begin{itemize}
            \item \(\subseteq \) Suppose \(W \) is an element of the topology
            generated by \(\mathcal{C} \), then \(W = \bigcup C \) where \(C
            \subseteq \mathcal{C} \), but note that every element of \(C \)
            is an element of \(\mathcal{T} \) since \(\mathcal{C} \)
            is a collection of open subsets of \(X \), therefore \(W \) being
            an arbitrary union is also open with respect to \(X \), that is \(
            W \in \mathcal{T} \).
            \item \(\supseteq \) Suppose \(U \in \mathcal{T} \) then if
            \(x \in U \) we have \(C \in \mathcal{C} \) such that \(x
            \in C \subseteq U\), then by the basis existance definition of a
            generated topology we can see that \(U \in \mathcal{T} _{\mathcal{
            C}} \)
        \end{itemize}
    \end{itemize}
\end{proof}



\begin{proposition}
{Finer is Equivalent to Basis Containment}{finer_is_equivalent_to_basis_containment}
Let \(\mathcal{B}_{1}\) and \(\mathcal{B}_{2}\) be two bases on a set \(X\),
then \(\mathcal{T}_{\mathcal{B}_{1}} \subseteq \mathcal{T}_{\mathcal{B}_{2}}\)
if and only if for every \(x \in X\) and \(B_{1} \in \mathcal{B}_{1}\)
containing \(x\), there is a \(B_{2} \in \mathcal{B}_{2}\) such that \(x \in
B_{2} \subseteq B_{1}\)
\end{proposition}
\begin{proof}
\begin{itemize}
    \item \( \Rightarrow  \) 
    \begin{itemize}
        \item Suppose \( \mathcal{ T } _{ \mathcal{ B } _{ 1 }    } \subseteq
        \mathcal{ T } _{ \mathcal{ B } _{ 2 }   }    \). Now let \( x \in  X \)
        and \( B _{ 1 } \in  \mathcal{ B } _{ 1 }   \) where \( x \in B _{ 1 }  \)
        (Note that we can do this because any basis covers \( X \)),
        \hyperref[corollary:basis_is_a_subset_of_the_topology_it_generates]{additionally
        we have that every basis is a subset of the topology it generates}
        therefore \( B _{ 1 } \in \mathcal{ T } _{ \mathcal{ B } _{ 1 } }     \) and
        so by assumption we have that \( B _{ 1 } \in \mathcal{ T } _{ \mathcal{
        B} _{ 2 }   }   \) which by definition means that \( \forall x \in B _{
        1} , \exists B _{ 2 } \in  \mathcal{ B } _{ 2 }, \) such that \( x \in B
        _{ 2 } \subseteq B _{ 1 } \) which is exactly what we wanted to show.
    \end{itemize}
    \item \( \Leftarrow  \) 
    \begin{itemize}
        \item Suppose the reverse, so let \( U \in \mathcal{ T } _{ \mathcal{ B
        } _{ 1 }   }   \) we must show that \( U \in  \mathcal{ T } _{ \mathcal{
        B } _{ 2 } }  \). Since \( U \in \mathcal{ T } _{ \mathcal{ B } _{ 1 }
        }    \)  this means that \( \forall x \in  U, \exists B _{ 1 } \in
        \mathcal{ B } _{ 1 }    \) such that \( x \in  B _{ 1 } \subseteq U \)
        is a true statement. Recall that we'd like to prove that \( U \in
        \mathcal{ T } _{ \mathcal{ B } _{ 2 }   }   \), namely that \( \forall x
        \in U\) we have \( B _{ 2 } \in \mathcal{ B } _{ 2 }   \) such that \( x
        \in  B _{ 2 }  \subseteq U\) 
        \item Therefore let \( x \in  U \) by the fact that \( U \in  \mathcal{
        T} _{ \mathcal{ B } _{ 1 }   }   \) we have \( B _{ 1  } \in \mathcal{ B
        } _{ 1 }   \) such that \( x \in \mathcal{ B } _{ 1 } \subseteq U  \). 
        \item By our original assumption (which is \( \forall x \in  X \) and \(
        B _{ 1 }  \in \mathcal{ B } _{ 1 }  \) containing \( x \) we have  \( B
        _{ 2 } \in  \mathcal{ B } _{ 2 }  \) with \( x \in B _{ 2 } \subseteq B
        _{ 1 } \)), we get \( B _{ 2 } \in  \mathcal{ B } _{ 2 }  \) such that
        \( x \in  B _{ 2 } \subseteq B _{ 1 } \) and recall that \( B _{ 1 }
        \subseteq U \) so we have \( x \in B _{ 2 } \subseteq U \) as needed,
        thus \( U \in  \mathcal{ T } _{ \mathcal{ B } _{ 2 }   }   \) 
    \end{itemize}
\end{itemize} 
\end{proof}


\begin{definition}
{Standard Topology on the Real Line}{standard_topology_on_the_real_line}
    If \(\mathcal{B} \) is the collection of all open intervals on the real
    line, that is:
    \[
    \left(a, b\right) = \left\{x : a < x < b \right\}
    \]
    then the topology generated by \(\mathcal{B} \) is called the standard
    topology on the real line
\end{definition}


\begin{definition}
{Lower Limit Topology}{lower_limit_topology}
     The topology generated by \(\mathcal{B} = \left\{\left[ a, b
\right) \subseteq \mathbb{R}: a < b \right\} \) is defined as the lower
     limit topology on \(\mathbb{R} \)
\end{definition}
The corresponding topological space \(\left(\mathbb{R}, \mathcal{S}\right)
\) is called the Sorgenfrey line.


\input{topology/definitions/k-topology_on_r}

\begin{proposition}
{Lower Limit and K-Topology are Strictly Finer than the
Standard Topology on
R}{lower_limit_and_k-topology_are_strictly_finer_than_the_standard_topology_on_r}
Let \(\mathcal{T} _{\ell} \) be the lower limit topology, \(\mathcal{T
} _{\mathcal{K}} \) be the \(\mathcal{K} \)-topology and \(
\mathcal{T} \) the standard topology on \(\mathbb{R} \), then
\[
\mathcal{T} \subset \mathcal{T} _{\ell} \enspace \text{and} \enspace
 \mathcal{T} \subset \mathcal{T} _{\mathcal{K}}
\]
\end{proposition}
\begin{proof}
    We show this by using
    \hyperref[proposition:finer_is_equivalent_to_basis_containment]{this}. So
    let \(x \in X \) and \(\left(a, b\right) \) be a basis element for
    \(\mathcal{T} \) then \(\left[ x, b\right) \) is a basis element of
    \(\mathcal{T} _{\ell} \) with \(\left[ x, b\right) \subseteq \left(
    a, b\right) \). Now see that \(\left[ x, d\right) \) is an element of \(
    \mathcal{T} _{\ell} \) (it is a basis element) for which there is no
    basis element of \(\mathcal{T} \) that contains \(x \) and is a subset
    of \(\left[ x, d\right) \) because they are open intervals, therefore \(
    \mathcal{T} \subset \mathcal{T} _{\ell} \).\\
    As for the \(\mathcal{K} \) topology we can see that for any basis
    element \(\left(a, b\right) \) of \(\mathcal{T} \) we can use that
    same basis element in \(\mathcal{T} _{\mathcal{K}} \) and thus we
    trivially get that \(\mathcal{T} \subseteq \mathcal{T} _{\mathcal{K}} \), we
    also note that for the basis element \( \left( -1, 1 \right) \setminus
    \mathcal{ K }   \) of \( \mathcal{ T } _{ \mathcal{ K }   }   \) and \( x =
    0\) there is no basis element of \( \mathcal{ T }   \) that contains \( 0 \)
    and is a subset of \( \left( -1, 1  \right) \setminus \mathcal{ K }   \) 
\end{proof}


\begin{definition}
{Subbasis}{subbasis}
A subbasis \(\mathcal{S}\) for a topology on \(X\) is a collection of subsets of
\(X\) whose union equals \(X \). 
\end{definition}


\begin{definition}
{Topology Generated by a
Subbasis}{topology_generated_by_a_subbasis}
 The topology generated by the subbasis \(\mathcal{S}\) is defined to be the
 collection \(\mathcal{T}\) of all unions of finite intersections of elements of
 \(\mathcal{S}\). In another light we may say that this topology is generated by
 the basis which is contstructed of finite intersections of elements of \(
 \mathcal{S} \)
\end{definition}


\begin{proposition}
{Topology Generated by a Subbasis is a
Topology}{topology_generated_by_a_subbasis_is_a_topology}
Let \(\mathcal{S} \) be a subbasis, the topology generated by \(\mathcal{
S} \) is a topology
\end{proposition}
\begin{proof}
To show that this is true we will use first show that that the set of finite
intersections of elements of \(\mathcal{S} \) (let's call this \(\mathcal{
B} _{\mathcal{S}} \enspace \Winkey \)) is a basis, then we know that \(\left\{
\bigcup B : B \subseteq \mathcal{B} _{\mathcal{S}} \right\} \) is a
topology by~\ref{proposition:topology_generated_by_a_basis_(union)_is_a_topology}\\
Let \(x \in X \) then since \(\bigcup \mathcal{S} = X \) we know that \(x
\in S\) for some \(S \in \mathcal{S} \), note that we say that a single
set is a finite intersection, and therefore \(S \in \mathcal{B} _{\mathcal{
S}} \) so we've found a basis element of \(\mathcal{B} _{\mathcal{S}
} \) which contains \(x \). Now for the second condition let \(\mathcal{I},
\mathcal{J} \) be finite index sets, then set \(B _{1} = \bigcap _{ \alpha \in
\mathcal{I}} S _{\alpha} \) and \(B _{2} = \bigcap _{ \alpha \in \mathcal{J}} S
_{\alpha} \), but then their intersection is \( \bigcap _{ \alpha \in \mathcal{
I} \cup \mathcal{ J }    } S _{ \alpha  }  \) and is therefore still a finite
union of elements of \( \mathcal{ S }   \), and so this set satisfies the second
condition for being a basis.
\end{proof}


\begin{definition}
{Order Topology}{order_topology}
Let \(X\) be a set with a simple order relation; assume \(X\) has more than one
element. Let \(\mathcal{B}\) be the collection of all sets of the following
types:
\begin{enumerate}
    \item All open intervals \((a, b)\) in \(X\).
    \item All intervals of the form \(\left[a_{0}, b\right)\), where \(a_{0}\)
    is the smallest element (if any) of \(X\).
    \item All intervals of the form \(\left(a, b_{0}\right]\), where \(b_{0}\)
    is the largest element (if any) of \(X\). The collection \(\mathcal{B}\) is
    a basis for a topology on \(X\), which is called the order topology.
\end{enumerate}
If \(X\) has no smallest element, there are no sets of type (2), and if \(X\)
has no largest element, there are no sets of type (3).
\end{definition}


\begin{proposition}
{Order Topology Basis}{order_topology_basis}
The set \(\mathcal{B} \) specified by~\ref{definition:order_topology} is a
basis.
\end{proposition}
\begin{proof}
Let \(x \in X \) if \(x \) is the smallest or largest element of \(X \) then
sets \(\left[ x, x + \varepsilon\right) \) or \(\left(x - \varepsilon,
x \right] \) respectively will work. Otherwise \(\left(x - \varepsilon, x
+ \varepsilon\right) \) will work.\\
Now we take two elements from \(\mathcal{B} \) and find a third contained
within both.
\begin{itemize}
    \item \(\left(a, b\right) \cap \left(c, d\right) = \left(x, y\right)
    \) for some \(x, y \)
    \item \(\left(a, M \right] \cap \left(b, M \right] = \left(\max\left(a, b\right),M \right] \)
    \item \(\left[m, a\right) \cap \left[m, b\right) = \left[m, \min\left(
    a, b\right)\right) \)
    \item \(\left(a, b\right) \cap \left(d, M \right] = \left(\max\left(a, d\right), b\right) \) and \(\left(a, b\right) \cap \left[ m, d
\right) = \left(a, \min\left(b, d\right)\right) \)
\end{itemize}
Therefore the second condition of a basis is satisfied so \(\mathcal{B} \)
is a basis.
\end{proof}


\begin{definition}
{Product Topology of two Topological Spaces}{product_topology_of_two_topological_spaces}
Let \(X\) and \(Y\) be topological spaces. The product topology on \(X \times
Y\) is the topology having as basis the collection \(\mathcal{B}\) of all sets
of the form \(U \times V\), where \(U\) is an open subset of \(X\) and \(V\) is
an open subset of \(Y\).
\end{definition}


\begin{proposition}{Open Cartesian Products form a
Basis}{open_cartesian_products_form_a_basis}
The set \( \mathcal{ B } \) specified by
\ref{definition:product_topology_of_two_topological_spaces} is a basis.
\end{proposition}
\begin{proof}
    Let \( x \in  X \) then the set \( X \times Y \) is a basis element since \(
    X\) and \( Y \) are open with respect to themselves. \\
    Let \( U _{ 1 } \times V _{ 1 }  \) and \( U _{ 2 } \times V _{ 2 }  \) be
    two elements of \( \mathcal{ B }   \) then  by
    \ref{proposition:commutivity_of_cartesian_product_and_(union,intersection)}
    we can see
    \[
        \left( U _{ 1 } \times V _{ 1 }  \right) \cap \left( U _{ 2 } \times V
        _{ 2 }  \right) = \left( U _{ 1 } \cap U _{ 2 }  \right) \times \left( V
        _{ 1 } \cap  V _{ 2 } \right) 
    \]
    and thus for the second condition of a basis we can trivially choose \( B =
    B\).
\end{proof}


\begin{proposition}
{Basis For the Product Topology of Two Topological
Spaces}{basis_for_the_product_topology_of_two_topological_spaces}
Let \(X, Y \) be topological spaces generated by the basis \(\mathcal{B},
\mathcal{C} \) then
\[
\mathcal{D} = \left\{B \times C: B \in \mathcal{B} \enspace \text{and}
\enspace C \in \mathcal{C} \right\}
\]
is a basis for the topology \(X \times Y \).
\end{proposition}
\begin{proof}
   We will prove it using the \hyperref[lemma:basis_criterion]{basis
   criterion}. Let \(W \) be an open set of \(X \times Y \), and let \(
   \left(x, y\right) \in W \) since \(\left\{U \times V: U \enspace
   \text{open in} \enspace X \enspace \text{and} \enspace \text{open
   in} \enspace Y \right\} \) is a basis for the topology \(X \times Y \)
   (and a basis covers the set \(X \times Y \)) we have some \(U \times V \)
   such that \(\left(x, y\right) \in U \times V \subseteq W \).
   Additionally, since \(\mathcal{B}, \mathcal{C} \) are bases for \(X, Y \) we
   get \(B \) and \(C \) such that \(x \in B \subseteq U \) (since \(U \) is
   open in \(X \) and
   using~\ref{definition:topology_generated_by_a_basis_(basis_existance)}) and
   \(y \in C \subseteq V \) so therefore \(\left(x, y\right) \in B \times C
   \subseteq W \) thus we've found an element of the basis in question which is
   a subset of of the open set, meaning we've satisfied the basis criterion and
   thus \(\mathcal{D} \) is a basis.
\end{proof}


\begin{proposition}
{Subbasis for the Product Topology of two Topological
Spaces}{subbasis_for_the_product_topology_of_two_topological_spaces}
The set
\[
\mathcal{S} = \left\{\pi _{1} ^{-1} \left(U\right) : U \enspace
\text{open in} \enspace X \right\} \cup \left\{\pi _{2} ^{-1} \left(V
\right) : V \enspace \text{open in} \enspace Y \right\}
\]
is a subbasis for the product topology on \(X \times Y \).
\end{proposition}
\begin{proof}
    We will show \(\mathcal{T} _ \mathcal{S} \subseteq \mathcal{T} _{X
    \times Y} \) directly and to show that \(\mathcal{T} _{X \times Y}
    \subseteq \mathcal{T} _ \mathcal{S} \) we use~\ref{proposition:finer_is_equivalent_to_basis_containment}. Recall that
    from~\ref{proposition:inverse_image_of_the_projection_mapping},
    \begin{gather*}
    \left\{\pi _{1} ^{-1} \left(U\right) : U \enspace
    \text{open in} \enspace X \right\} \cup \left\{\pi _{2} ^{-1} \left(V
\right) : V \enspace \text{open in} \enspace Y \right\}\\
    \verteq\\
    \left\{U \times Y: U \text{open in} \enspace X \right\} \cup \left\{X
    \times V: V \enspace \text{open in} \enspace Y \right\}
    \end{gather*}
    So any element of \(\mathcal{S} \) is an element of the topology on \(
    X \times Y\) because every element of \(\mathcal{S} \) is of the form \(A
    \times B \) where \(A \) is open in \(X \) and \(B \) is open in \(Y \)
    so that they (any element of \(\mathcal{S} \)) are basis elements of the
    basis which generates \(X \times Y \), therefore arbitrary unions of finite
    intersections of elements of \(\mathcal{S} \) are still part of the
    topology (by the definition of topology), thus we have directly shown that
    \(\mathcal{T} _{\mathcal{S}} \subseteq \mathcal{T} _{X \times Y
} \). \\
    Now let \(U \times V \) be a basis element of \(\mathcal{T} _{X \times
    Y} \) but note that
    \begin{align*}
        U \times V &= \left(U \times Y\right) \cap \left(X \times V\right)
        \\
                   &= \pi _{1} ^{-1} \left(U\right) \cap \pi _{2} ^{-1}
                   \left(V\right)
    \end{align*}
    that is to say that \(U \times V \) is equal to a basis element of \(
    \mathcal{T} _{\mathcal{S}} \) and therefore
    from~\ref{proposition:finer_is_equivalent_to_basis_containment} it follows
    trivally that \(\mathcal{T} _{X \times Y} \subseteq \mathcal{T} _{
    \mathcal{S}} \), hence \( \mathcal{ T } _{ \mathcal{ S }   } = \mathcal{ T }
    _{ X \times Y } \) 
\end{proof}


\subsection{The Subspace Topology}

\begin{definition}{Subspace Topology}{subspace_topology}
    Let $X$ be a topological space with topology $\mathcal{T}$. If $Y$ is a subset of $X$, the collection
    \[
    \mathcal{T}_{Y}=\{Y \cap U \mid U \in \mathcal{T}\}
    \]
    is a topology on $Y$, called the subspace topology. With this topology, $Y$ is called a subspace of $X$; its open sets consist of all intersections of open sets of $X$ with $Y$.
\end{definition}


\begin{proposition}{The Subspace Topology is a
Topology}{the_subspace_topology_is_a_topology}
\( \mathcal{ T } _{ Y }   \) is a topology
\end{proposition}
\begin{proof}
    Note that \( \varnothing = Y \cap \varnothing \in  \mathcal{ T } _{ Y }
    \) and that \( X = Y \cap  X \in  \mathcal{ T } _{ Y } \). Now consider an
    arbitrary union: \( \bigcup _{ \alpha \in \mathcal{ J }   } \left( U _{
    \alpha } \cap Y \right)  \) then by the fact that
    \hyperref[proposition:intersection_factors]{intersection factors} we can see
    that it equals \( \left( \bigcup _{ \alpha \in \mathcal{ J }   } U _{ \alpha
    } \right) \cap Y \) which is an element of the subspace topology. Again
    using the fact that the intersection factors, we can see that for the finite
    intersection we have \( \bigcap _{ i \in  \left[ n \right]  } \left( U _{ i
    } \cap  Y \right) = \left( \bigcap _{ i \in  \left[ n \right]  } U _{ i }
    \right) \cap  Y\) again an element of the subspace topology, thus it is
    indeed a topology.
\end{proof}


\begin{proposition}{Basis for the Subspace
Topology}{basis_for_the_subspace_topology}
If \( \mathcal{ B }   \) is a basis for the topology of \( X \), then 
\[
\mathcal{ B } _{ Y } = \left\{ B \cap Y: B \in  \mathcal{ B }   \right\}  
\]
is a basis for the subspace topology on \( Y \).
\end{proposition}
\begin{proof}
    We will proceed by attempting to satisfy the
    \hyperref[lemma:basis_criterion]{basis criterion}, so let \( U \) be an open
    set of \( X \) and consider \( y \in U \cap  Y \) since \( y \in U \) and \(
    \mathcal{ B }  \) generates the topology on \( X \), then by the basis
    existance definition, we get \( B \in \mathcal{ B }   \) such that \( y \in
    B \subseteq U\) since \( y \in Y \) we can say the following \( y \in B \cap
    Y \subseteq U \cap Y\) therefore we've found an element of \( \mathcal{ B }
    _{ Y } \)  contained within our open set \( U \cap Y \) as needed, thus \(
    \mathcal{ B } _{ Y }   \) is a basis for \( \mathcal{ T } _{ Y }   \) .
\end{proof}


\begin{proposition}{Subspace Topology
Transitivity}{subspace_topology_transitivity}
Let \( Y \) be a subspace of \( X \). If \( U \) is open in \( Y \) and \( Y \)
is open in \( X \) , then \( U \) is open in \( X \) 
\end{proposition}
\begin{proof}
   If \( U  \) is open in \( Y \) then \( U =  Y \cap  V \)  where \( V \) is
   open in \( X \) since \( Y \)  is also open in \( X \) then their
   intersection is open in \( X \), in other words \( U  \) is open in \( X \) 
\end{proof}


\begin{proposition}{Subspace Product is same as Normal
Product}{subspace_product_is_same_as_normal_product}
If \( A \) is a subspace of \( X \) and \( B \) is a subspace of \( Y \), then
the product topology on \( A \times B \) is the same as the subspace topology \(
A \times B\) inherits as a subspace of \( X \times Y \).
\end{proposition}
\begin{proof}
\begin{itemize}
    \item As a subspace
    \begin{itemize}
        \item The set of elements \( U \times V \) where \( U \) is open in \( X
        \) and \( V \) is open in \( Y \)  is a basis for \( X \times Y
        \)
        \item Therefore \( \left\{ \left( U \times  V \right) \cap \left( A \times B
        \right): U \in \mathcal{ T } _{ X } ~\&~ V \in \mathcal{ T } _{ Y }
        \right\}  \) is a basis for the subspace \( A \times B \) with
        respect to \( X \times Y \) by
        \ref{proposition:basis_for_the_subspace_topology}
    \end{itemize}
    \item As a product
    \begin{itemize}
        \item \( A \times B \) is generated by the basis \( \left\{ J \times
        K: J \enspace \text{open in} \enspace A, K \enspace \text{open in}
        \enspace B \right\}  \) 
        \item Since \( A \) is a subspace of \( X \) and \( B \) is a subspace
        of \( Y \) we know that
        \[
        \mathcal{ T } _{ A } = \left\{ U \cap A: U \enspace \text{open in}
        \enspace X \right\}  \enspace \text{and} \enspace  
        \mathcal{ T } _{ B } = \left\{ V \cap B: V \enspace \text{open in}
        \enspace B \right\} 
        \]
    \end{itemize}

    \item With all that context in mind, observe the following :
    \begin{gather*}
        \left\{ J \times K: J \enspace \text{open in} \enspace A, K \enspace
        \text{open in} \enspace B \right\} \\
        \verteq\\
        \left\{ \left( U \cap A \right) \times \left( V \cap  B \right):
        U \in \mathcal{ T } _{ X } \enspace \& \enspace V \in \mathcal{ T }
        _{ Y } \right\} \\
        \verteq \enspace \text{by
        \ref{proposition:commutivity_of_cartesian_product_and_(union,intersection)}}
        \enspace  \\
        \left\{ \left(  U \times V \right) \cap \left( A \times B \right) :
        U \in \mathcal{ T } _{ X } \enspace \& \enspace V \in  \mathcal{ T }
        _{ Y } \right\} 
    \end{gather*}
    \item Therefore the two basis are identical, and thus the topology they
    generate are identical, as needed.
\end{itemize}
\end{proof}


\subsection{Closed Sets and Limit Points}

\begin{definition}{Closed Set}{closed_set}
A subset \( A \) of a topological space \( X \) is closed if and only if \( X
\setminus A \) is open.
\end{definition}


\begin{proposition}{Properties of Closure}{properties_of_closure}
Suppose \( X \) is a topological space, then the following holds:
\begin{enumerate}
    \item \( \varnothing , X \) are closed
    \item Arbitrary intersections of closed sets are closed
    \item Finite unions of closed sets are closed
\end{enumerate}
\end{proposition}
\begin{proof}
    \begin{enumerate}
        \item \( X \setminus \varnothing = X \) and \( X \setminus X =
        \varnothing  \) and since \( X \) is a topology we know that \(
        \varnothing , X \in  \mathcal{ T }   _{ X }  \) 
        \item Suppose \( \left\{ A _{ \alpha  }  \right\} _{ \alpha \in
        \mathcal{ J }   }  \) are closed sets, then from
        \hyperref[theorem:demorgan's_laws]{DeMorgan's Laws} we know that 
        \[
        X \setminus \bigcap _{ \alpha \in \mathcal{ J }   } = \bigcup _{ \alpha
        \in \mathcal{ J }  } \left( X -  A _{ \alpha  }  \right) 
        \]
        which is an aribtrary union of open sets and is therefore open.
        \item Suppose that \( A _{ i }  \) is closed for \( i \in  \left[ n
        \right] \) then again from DeMorgan we have
        \[
        X \setminus  \bigcup _{ i = 1 } ^{ n } A _{ i } = \bigcap _{ i = 1 }^{ n
        } \left( X \setminus A _{ i }  \right) 
        \]
        thus it's open as it's equal to a finite intersection of open sets.
    \end{enumerate}
\end{proof}


\begin{proposition}{Closed in a Subspace}{closed_in_a_subspace}
Suppose \( Y \) is a subspace of \( X \), then \( A \) is closed in \( Y \) if
and only if it equals the intersection of a closed set of \( X \) with \( Y \) 
\end{proposition}
\begin{proof}
    \( \Leftarrow  \), assume that \( A =  C \cap  Y \) with \( C \) closed in
    \( X \), then \( X \setminus C \) is open in \( X \). Thus \( \left( X - C
    \right) \cap Y \in \mathcal{ T } _{ Y }   \) by
    \ref{definition:subspace_topology} by
    \ref{proposition:associativity_of_intersection_and_set_difference} we get
    that \( \left( X \setminus C \right) \cap  Y = Y \setminus A \) so \( Y
    \setminus  \) is open in \( Y \) which means \( A \) is closed in \( Y \)
    .\\
    \( \Rightarrow  \)  Assuming that \( A \) is closed in \( Y \) so then \( Y
    \setminus A\) is open in \( Y \) by the definition of the subspace topology
    that means that \( Y \setminus A = U \cap Y \) where \( U  \) is open in \(
    X\), setting \( A =  Y \cap  \left( X \setminus U \right)  \) satisfies the
    given equation by \ref{proposition:intersection_as_set_difference} and thus
    \( A \) is equal to the intersection a closed set of \( X \) with \( Y \)
    since \( U  \) was open in \( X \) making \( X \setminus U \) closed.
\end{proof}


\begin{definition}{Interior of a Set}{interior_of_a_set}
Given a subset \( A \) of a topological space \( X \), the interior of \( A \)
is the union of all open sets contained in \( A \), and is written as \(
\operatorname{ int } \left( A \right)  \).Note as a union of open sets, it's
open itself.
\end{definition}


\begin{proposition}{Interior is Itself}{interior_is_itself}
Suppose that \( A \) is open if and only if  \( \operatorname{ int } \left( A \right) = A  \) 
\end{proposition}
\begin{proof}
If \( A \) is open then \( A \) is an open set set containing \( A \) therefore
as \( \operatorname{int} \left( A \right)  \) is the union of all open sets
containing \( A \), we can see that \( \operatorname{int} \left( A \right) = A
\) since it is a union of sets \( X \subseteq A \) with \( A \) itself. \\
Suppose \( \operatorname{int} \left( A \right) = A \) since \(
\operatorname{int} \left( A \right)  \) is an arbitrary union of open sets then
it is still open, thus \( A \) is open.
\end{proof}


\begin{definition}{Closure of a Set}{closure_of_a_set}
Given a subset \( A \) of a topological space, the closure of \( A \) is the
intersection of all closed sets containing \( A \), and is denoted by \(
\overline{A}  \). Note that from \ref{proposition:properties_of_closure} that \(
\overline{A} \) is closed.
\end{definition}


\begin{proposition}{Closure is Itself}{closure_is_itself}
Suppose that \( A \) is closed if and only if \( A = \overline{A}  \) 
\end{proposition}
\begin{proof}
    Recall that \( \overline{A}  \) is defined as the intersection of all closed
    sets which contain \( A \), thus the smallest set which could contain \( A
    \) is \( A \) itself, since \( A \) was assumed to be closed, the
    intersection is equal to \( A \).\\ 
    Supposing that \( \overline{A} = A \) then since \( \overline{A}  \) is an
    arbitrary intersection of closed sets, then by
    \ref{proposition:properties_of_closure} \( A \) is closed as well.
\end{proof}


\begin{proposition}{Subset Relationship between Interior and
Closure}{subset_relationship_between_interior_and_closure}
Suppose that \( A \) is a subset of a topological space \( X \), then 
\[
\operatorname{ int } \left( A \right)  \subseteq A \subseteq \overline{A} 
\]
\end{proposition}
\begin{proof}
    By definition \( \operatorname{ int } \left( A \right)  \) is the union of
    sets which are subsets of \( A \) thus it is a subset of \( A \), on the
    otherhand \( \overline{A}  \) is an intersection of sets which all contain
    \( A \) thus \( A \subseteq \overline{A}  \), so we've deduced that \(
    \operatorname{ int } \left( A \right) \subseteq A \subseteq \overline{A}
    \) as needed.
\end{proof}


\begin{proposition}{Closure in a Subspace}{closure_in_a_subspace}
Let \( Y \) be a subspace of \( X \) and let \( A \) be a subset of \( Y \)
then with \( \overline{A}  \) being the closure of \( A \) in \( X \), then the
closure of \( A \)  in \( Y \) is \( \overline{A}  \cap Y \) 
\end{proposition}
\begin{proof}
    Let \( B \) be the closure of \( A \) in \( Y \). Since \( \overline{A}  \)
    is closed in \( X \) then \( \overline{A} \cap Y \) is closed in \( Y \) as
    it an intersection of a closed set of \( X \) with \( Y \) (see
    \ref{proposition:closed_in_a_subspace}). By the definition of closure \( B
    \) is the intersection of all closed subsets of \( Y \) which contain \( A
    \) and thus \( B \subseteq \left( \overline{A} \cap Y \right)  \). Also
    since \( B \) is closed in \( Y \) then it equals \( C \cap  Y \) for some
    set \( C \) which is closed in \( X \) (by
    \ref{proposition:closed_in_a_subspace} again), recall that \( \overline{A}
    \) is closed in \( X \) and thus by it's definition we get that \(
    \overline{A} \subseteq C \) which yields \( \overline{A} \cap Y \subseteq C
    \cap Y = B\) so we've shown that \( \overline{A} \cap Y \subseteq B
    \subseteq \overline{A} \cap Y\) and so \( B = \overline{A} \cap Y \) .
\end{proof}


\begin{proposition}{Element of Closure
Characterization}{element_of_closure_characterization}
Let \( A \) be a subset of a topological space \( X \), then
\( x \in  \overline{A}  \) if and only if every open set \( U \) containing \( x
\) intersects \( A \) 
\end{proposition}
\begin{proof}
    To prove \( A \Leftrightarrow B \) one may equivalently prove \( \neg A
    \Leftrightarrow \neg B \), we perform the latter.\\
    Suppose \( x \not\in \overline{A}  \) let us show that there is some open
    set \( U \) containing \( x \) which doesn't intersect \( A \). By
    considering the set \( U = X \setminus \overline{A}  \) then \( x \in  U \)
    and \( A \cap  U = \varnothing  \) (since \( A \subseteq \overline{A}  \))
    so they do not interesect.\\
    Conversely, if we have an open set \( U \) containing \( x \) which doesn't
    intersect with \( A\), then \( X \setminus U \) is a closed set of \( X \)
    which contains \( A \), since \( \overline{A}  \) is the intersection of all
    closed sets which contain \( A \) then we can see that \( \overline{A}
    \subseteq X \setminus U \) now since \( x \in U \) then \( x \not\in X
    \setminus U \) and since \( \overline{A} \subseteq X \setminus U \) then
    also \( x \not\in \overline{A}  \) which is what we needed to prove.
\end{proof}



\begin{proposition}{Element of Closure Basis
Characterization}{element_of_closure_basis_characterization}
Let \( A \) be a subset of a topological space \( X \), then
supposing that the topology of \( X \) is given by a basis, then \( x \in 
\overline{A}  \) if and only if every basis element \( B \) containing \( x \)
intersects \( A \) 
\end{proposition}
\begin{proof}
Since every basis element is part of the topology it generates
(\ref{corollary:basis_is_a_subset_of_the_topology_it_generates}) then this means
that every \( B \in  \mathcal{ B }   \) are open sets. Now suppose that \( x \in
\overline{A} \) then by \ref{proposition:element_of_closure_characterization}
every open set \( U \) containing \( x \) intersects \( A \), then consider
every \( B \in  \mathcal{ B }   \) where \( x \in  B \), since \( B \) is open
then we know that \( B \) and \( A \) intersect, so every basis element \( B \)
containing \( x \) intersects \( A \). \\
For the other direction we assume that every basis element that contains \( x \)
intersects \( A \). Now considering \( U \in \mathcal{ T } _{ \mathcal{ B }   }
\) by \hyperref[definition:topology_generated_by_a_basis_(basis_existance)]{the
basis existance definition of a topology generated by a basis}, we can see that
an open set \( U \) that contains \( x \) contains a basis element which
intersects \( A \) and thus \( U \) also intersects \( A \) since \( B \subseteq
U\), thus by \ref{proposition:element_of_closure_characterization} \( x \in
\overline{A}  \), as needed.
\end{proof}


\begin{definition}{Neighborhood of a Point}{neighborhood_of_a_point}
A neighborhood \( U \) of a point \( x \) is an open set \( U \) which contains
\( x \).
\end{definition}


\begin{definition}{Limit Point}{limit_point}
If \( A \) is a subset of a topological space \( X \) we say that \( x \) is a
limit point of \( A \) if and only if every neighborhood of \( x \) intersects
\( A \) in some point other than itself. Note that \( x \) need not be an
element of \( A \).
\end{definition}


\begin{proposition}{Closure as Union of Original Set and Limit
Points}{closure_as_union_of_original_set_and_limit_points}
Let \( A \) be a subset of a topological space \( X \) and let \( A ^{ \prime  }
\) denote the limit points of \( A \) then: 
\[
\overline{A} = A \cup  A ^{ \prime  } 
\]
\end{proposition}
\begin{proof}
    If \( x \in  A ^{ \prime  } \) then by definition every neighborhood \( U \)
    of \( x \) has that \( U  \cap \left( A \setminus \left\{ x \right\}
    \right) \neq \varnothing \) thus by
    \ref{proposition:element_of_closure_characterization} we can see that \( x
    \in  \overline{A}  \), that is \( A ^{ \prime  } \subseteq \overline{A}  \),
    now one should not forget that \( A \subseteq \overline{A}  \)
    (\ref{proposition:subset_relationship_between_interior_and_closure}) thus
    having both in tandem yields \( A ^{ \prime  }  \cup  A \subseteq
    \overline{A} \).\\
    For the reverse inclusion, let \( x \in  \overline{A}  \) and we'll prove
    that \( x \in  A ^{ \prime  } \cup  A \), but if \( x \in A \) then surely
    it's true, thus we may assume that \( x \not\in A \). Due to the fact that
    \( x \in \overline{A}  \) we know that every neighborhood \( U \) of \( x \)
    intersects \( A \), but \( x \not\in A \) so the intersection point is not
    \( x \), in other words we have that \( U \cap  \left( A \setminus \left\{ x
    \right\}  \right) \neq \varnothing  \) and so \( x \in A ^{ \prime  }  \)
    which tells us \( x \in A \cup A ^{ \prime  }  \).
\end{proof}


\begin{proposition}{Closed if it Contains it's Limit Points}{closed_if_it_contains_it's_limit_points}
A subset \( A \) of of a topological space \( X \) is closed if and only if it
contains all of it's limit points, that is 
\[
A ^{ \prime  } \subseteq A
\]
\end{proposition}
\begin{proof}
    \( A \) is closed if and only if \( A = \overline{A}  \), by
    \ref{proposition:closure_as_union_of_original_set_and_limit_points} we see
    that \( A =  A \cup  A ^{ \prime  } \), this holds if and only if \( A ^{
    \prime  } \subseteq A \) because if \( A ^{ \prime  }  \) contains an
    element which is not an element of \( A \) we get a contradiction since \( A
    =  A ^{ \prime  } \cup  A\)  therefore every element of \( A ^{ \prime  }
    \) is an element of \( A \), namely \( A ^{ \prime  } \subseteq A \) 
\end{proof}


\begin{definition}{Convergence}{convergence}
Given a topological space \( X \) we say that a sequence \( x _{ 1 } , x _{  2 }
, \ldots  \) converges to a point \( x \in  X \) if for each neighborhood \( U
\) of \( x \) there is an \( n \in \mathbb{N} ^{ + }  \) such that for all \( n
\ge N\) we have \( x _{ n } \in  U \) 
\end{definition}


\begin{definition}{T1 Space}{t1_space}
A topological space \( X \) is said to be \( T _{ 1 }  \) if for any \( x, y
\din X \), there are neighborhoods \( U, V \) of \( x, y \) respectively such
that \( y \not\in U \) and \( x \not\in V \) 
\end{definition}


\begin{definition}{Hausdorff Space}{hausdorff_space}
A topological space \( X \) is called a Hausdorff space if for each pair \( x, y
\din X\) there exists disjoint neighborhoods \( U, V \) of \( x, y \). A space
with this property is said to be \( T _{ 2 }  \) .
\end{definition}


\begin{proposition}{Every finite point set in a T1 space is
Closed}{every_finite_point_set_in_a_t1_space_is_closed}
Let \( S \) be a finite subset of a topological space \( X \) given then \( T _{
1}  \) property, then \( S \) is closed.
\end{proposition}
\begin{proof}
Since finite unions of closed sets are closed
(\ref{proposition:properties_of_closure}), then we can see that we may
equivalently prove that every one point set is closed.\\
Let \( x \in  X \), we will show \( \left\{ x \right\}  \) is closed in \( X \),
so we will show that \( X \setminus \left\{ x \right\}  \) is open via
\ref{proposition:open_iff_every_point_is_in_another_open_set}. Let \( a \in  X
\setminus \left\{ x \right\}  \) then since \( a \neq x \) by the \( T _{ 1 }
\) axiom, we get a neighborhood \( U _{ a }  \) of \( a \) such that \( x
\not\in U _{ a }  \), thus by
\ref{proposition:open_iff_every_point_is_in_another_open_set}, \(  X \setminus
\left\{ x \right\}  \) is open and therefore \( \left\{ x \right\}  \) is
closed.
\end{proof}



\begin{proposition}{T1 Space, Limit Point iff Every Neighborhood contains
Infinitely Many
Points}{t1_space,_limit_point_iff_every_neighborhood_contains_infinitely_many_points}
Let \( X \) be a space satisfying the \( T _{ 1 }  \) axiom, and \( A \) a
subset of \( X \), then the point \( x \) is a limit point of \( A \) if and
only if every neighborhood of \( x \) contains infinitely many points of \( A \) 
\end{proposition}
\begin{proof}
    \( \Leftarrow  \) If every neighborhood of \( x \) contains infinitely many
    points of \( A \), then it intersects \( A \) in some point other than \( x
    \) so therefore \( x \in A ^{ \prime  } \) by definition.\\
    \( \Rightarrow  \) Supposing that \( x \in A ^{ \prime  }  \) and for the
    sake of contradiction that some neighborhood \( U \) of \( x \) intersects
    \( A \) in only finitely many points, then also \( U \setminus \left\{ x
    \right\}  \) intersects \( A \) in only finitely many points, so let \(
    \left( U \setminus \left\{ x \right\} \cap A \right) = \left\{ x_{1} , x_{2}
    , \dotsc , x_{n} \right\}  \), but then since  \( \left\{ x_{1} , x_{2}
    , \dotsc , x_{n} \right\} \) is a finite point set in a \( T _{ 1 }  \)
    space, we know that it is closed so that \( X \setminus \left\{ x_{1} ,
    x_{2} , \dotsc , x_{n} \right\}  \) is open (and note that \( x \in  X
    \setminus \left\{ x_{1} , x_{2} , \dotsc , x_{n} \right\}  \) since \( x
    \not\in U \cap \left( A \setminus \left\{ x \right\}  \right) = \left\{
    x_{1} , x_{2} , \dotsc , x_{n} \right\} \)). Thus \( U \cap X \setminus
    \left\{ x_{1} , x_{2} , \dotsc , x_{n} \right\}  \) is an open set
    containing \( x \) and therefore a neighborhood of \( x \), but note that
    this set doesn't intersect \( A \setminus \left\{ x \right\}  \) because if
    \( k \in U \cap \left( X \setminus \left\{ x_{1} , x_{2} , \dotsc , x_{n}
    \right\} \right)    \) then \( k \not\in \left\{ x_{1} , x_{2} , \dotsc ,
    x_{n} \right\}  \) so \( k \not\in A \setminus \left\{ x \right\}  \), this
    is a contradiction with the fact that \( x \) was a limit point, because
    when that's true every neighborhood of \( x \) intersects \( A \setminus
    \left\{ x \right\}  \). Thus we conclude that every neighborhood of \( x \)
    contains infinitely many points of \( A \).
\end{proof}



\begin{proposition}{Hausdorff Yields Unique
Convergence}{hausdorff_yields_unique_convergence}
If \( X \) is a Hausdorff space then a sequence of points \( X \) converges to
at most one point of \( X \) 
\end{proposition}
\begin{proof}
    Suppose \( x _{ n }  \) is a sequence of points of \( X \) which converges
    to \( x \). Let \( y \in  X \) such that \( y \neq x \), then we have two
    disjoint neighborhoods \( U _{ x } , V _{ y }  \) since \( U _{ x }  \)
    contains all the points of \( x _{ n }  \) except for finitely many, then \(
    V _{ y } \) can only contain finitely many points of \( x _{ n }  \) so that
    \( x _{ n }  \) cannot converge to \( y \) 
\end{proof}


\subsection{Continuous Functions}

\begin{definition}{Continuous Function}{continuous_function}
Let \( X, Y\) be topological spaces, then a function \( f : X \to Y  \) is said
to be continuous if for each open subset \( V \) of \( Y \) the set \( f ^{-1}
\left( V \right)  \) is an open subset of \( X \) 
\end{definition}


\begin{proposition}{Continuous Function in terms of Basis
Elements}{continuous_function_in_terms_of_basis_elements}
Suppose \( X, Y \) are topological spaces where \( Y \) is generated by the
basis \( \mathcal{ B }   \) then \( f \) is continuous if the inverse image of
every basis element is open in \( X \).
\end{proposition}
\begin{proof}
    From \ref{definition:topology_generated_by_a_basis_(union)} we know that
    given \( V \in  \mathcal{ T } _{ Y }    \) that \( V =  \bigcup _{ \alpha
    \in \mathcal{ J }   } B _{ \alpha  }  \) thus by
    \ref{proposition:inverse_image_respects_set_operations} we have
    \[
    f ^{-1} \left( V \right) = \bigcup _{ \alpha \in \mathcal{ J }   } f ^{-1}
    \left( B _{ \alpha  }  \right) 
    \]
    So then \( f ^{-1} \left( V \right)  \) is open with respect to \( X \) so
    long as \( f ^{-1} \left( B _{ \alpha  }  \right)  \) is for each \( \alpha
    \in  \mathcal{ J }  \) 
\end{proof}


\begin{proposition}{Continuity Equivalences}{continuity_equivalences}
Let \( X, Y \) be topological spaces; let \( f : X \to Y \) then the following
are equivalent:
\begin{enumerate}
    \item \( f \) is continuous
    \item For every \( A \subseteq X \) we have that \( f\left( \overline{A}
    \right) \subseteq \overline{f\left( A \right) }  \) 
    \item For every closed set \( B \subseteq Y  \), \( f ^{-1} \left( B \right)
    \) is a closed in \( X \) 
    \item For each \( x \in  X \) and neighborhood \( V \) of \( f\left( x
    \right)  \), there exists a neighborhood \( U \) of \( x \) such that \(
    f\left( U \right) \subseteq V \) 
\end{enumerate}
\end{proposition}
\begin{proof}
    To show that they are all equivalent we will prove that \( 1 \Rightarrow 2
    \) , \( 2 \Rightarrow 3 \) , \( 3 \Rightarrow 1 \) and \( 1 \Rightarrow 4
    \), \( 4 \Rightarrow 1 \), this will show all possible bi-implications as
    all bi-implications using \( 1, 2, 3 \) are proving using the cycle \( 1
    \Rightarrow 2 \Rightarrow 3 \Rightarrow 1 \) and for any bi-implication
    which uses \( 4 \), say \( a \Rightarrow 4 \) then it may be obtained by
    getting starting from \( a \) getting to \( 1 \) (always possible as \( a \)
    is in the cycle which contains \( 1 \)) which leads to \( 4 \), to show \( 4
    \Rightarrow a\) we see it's possible by entering the cycle from \( 4 \)
    through \( 1 \) .\\
    \( 1 \Rightarrow 2 \) Suppose that \( f \) is continuous and let \( A
    \subseteq X \) we must show that if \( k \in  f\left( \overline{A}  \right)
    \) then we have that \( k \in  \overline{f\left( A \right) }  \), to do that
    we will use
    \ref{proposition:element_of_closure_characterization}, supposing that \( k
    \in  f\left( \overline{A}  \right)  \) then we know that \( k = f\left( x
    \right)  \) where \( x \in  \overline{A}  \) we want to show that \( k \in
    \overline{f\left( A \right) } \) so as mentioned let \( U \) be an open set
    which contains \( k \), we must show that \( U \) intersects \( f\left( A
    \right)  \) since \( U \) is open in \( Y \) we know that \( f ^{-1} \left(
    U\right)  \) is open in \( X \) since we assumed that \( f \) is continuous.
    Recall that \( k \in  U \) and that \( k =  f\left( x \right)  \) so then \(
     x \in f ^{-1} \left( U \right) \) so \( f ^{-1} \left( U \right)  \) is an
     open set of \( X \) which contains \( x \in \overline{A}  \) thus since
     \ref{proposition:element_of_closure_characterization} is an if and only if
     we know that \( f ^{-1} \left( U \right)  \) must intersect \( \overline{A}
     \) at some point \( y \) so that \( U  \) intersects \( f\left( A \right)
     \) at the point \( f\left( y \right)  \) since \( y \in  A \cap f ^{-1}
     \left( U \right)  \) , therefore by
     \ref{proposition:element_of_closure_characterization} we know that \( k \in
     \overline{f\left( A \right) } \) as needed.\\
     \( 2 \Rightarrow 3 \) Suppose \( 2  \) holds and let \( B \subseteq Y \) be
     a closed set, we will show that \( f ^{-1} \left( B \right)  \) is closed
     in \( X \), for convience set \( A =  f ^{-1}  \left( B \right)  \) then
     we will show that \( A = \overline{A}  \) to satisfy
     \ref{proposition:closure_is_itself}. We are already aware that \( A
     \subseteq \overline{A}  \) by
     \ref{proposition:subset_relationship_between_interior_and_closure} so
     therefore we just have to show that \( \overline{A} \subseteq  A \) to do
     this we will show that we will take \( x \in  \overline{A}  \) and therefore we know  
     \( f\left( x \right) \in  f\left( \overline{A}  \right)   \) from \( 2 \)
     yielding \( f\left( x \right) \in  f\left( \overline{A}  \right) \subseteq
     \overline{f\left( A \right) } \subseteq B\)  where the final equality
     follows from \ref{proposition:image_of_the_inverse_image} and that fact
     that \( B \) is closed, specifically \( \overline{f\left( A \right) } =
     \overline{f\left( f ^{-1} \left( B \right)  \right) } \subseteq
     \overline{B} = B  \), so it holds \( f\left( x \right) \in  B  \)
     equivalently \( x \in  f ^{-1} \left( B \right) = A \) so \( \overline{A}
     \subseteq A \) yielding \( \overline{A} = A \). \\
     \( 3 \Rightarrow 1 \) Let \( V \) be an open set of \( Y \), and let \( B
     = Y \setminus  V\) therefore by
     \ref{proposition:inverse_image_respects_set_operations} we can see that 
     \[
     f ^{-1} \left( B \right) =  f ^{-1} \left( Y \right) \setminus f ^{-1}
     \left( V \right) = X \setminus f ^{-1} \left( V \right) 
     \]
     where the second equality is being justified by
     \ref{proposition:inverse_image_of_codomain}. Since \( f ^{-1} \left( B
     \right)  \) is a closed set of \( X \) due to this one can see that \( X
     \setminus f ^{-1} \left( V \right)  \) is also closed meaning that \(
     X \setminus \left( X \setminus f ^{-1} \left( V \right)  \right)   \) is an
     open set of \( X \) and we note that \( X \setminus \left( X \setminus f
     ^{-1} \left( V \right)  \right) =  f ^{-1} \left( V \right) \), namely that
     \( f ^{-1} \left( V \right)  \) is an open set of \( X \) as needed.\\
     \( 1 \Rightarrow 4 \) Let \( x \in  X \) and suppose \( v \) is a
     neighborhood of \( f\left( x \right)  \) then \( U = f ^{-1} \left( V
     \right)  \) is an open set containing \( x \) and thus a neighborhood of \(
     x\) and by \ref{proposition:image_of_the_inverse_image} one can see that \(
     f\left( U \right) =  f\left( f ^{-1} \left( V \right)  \right) \subseteq V
     \) as needed.
     \( 4 \Rightarrow 1 \) Now suppose the converse so to prove that \( f \) is
     continuous let \( V \) be an open set of \( Y \) and let \( x \) be a point
     of \( f ^{-1} \left( V \right)  \), so that \( V \) becomes a neighborhood
     of \( f\left( x \right)  \) 
\end{proof}


\subsection{The Product Topology}


\begin{definition}{Product Topology}{product_topology}
    Let $\mathcal{S}_{\beta}$ denote the collection
    \[
    \mathcal{ S } _{\beta}=\left\{\pi_{\beta}^{-1}\left(U_{\beta}\right) \mid U_{\beta} \text { open in } X_{\beta}\right\}
    \]
    and let $\mathcal{ S } $ denote the union of these collections,
    \[
    \mathcal{S}=\bigcup_{\beta \in J} \mathcal{S}_{\beta}
    \]
    The topology generated by the subbasis $\mathcal{ S } $ is called the product topology. In this topology $\prod_{\alpha \in J} X_{\alpha}$ is called a product space.
\end{definition}


\begin{definition}{Box Topology}{box_topology}
Let $\left\{X_{\alpha}\right\}_{\alpha \in J}$ be an indexed family of topological spaces. Let us take as a basis for a topology on the product space
$$
\prod_{\alpha \in J} X_{\alpha}
$$
the collection of all sets of the form
$$
\prod_{\alpha \in J} U_{\alpha}
$$
where $U_{\alpha}$ is open in $X_{\alpha}$, for each $\alpha \in J$. The topology generated by this basis is called the box topology.
\end{definition}


\begin{theorem}{Basis for the Box Topology}{basis_for_the_box_topology}

Suppose the topology on each space $X_{\alpha}$ is given by a basis $\mathcal{B}_{\alpha}$. The collection of all sets of the form
\[
\prod_{\alpha \in \mathcal{ J } } B_{\alpha}
\]
where $B_{\alpha} \in \mathcal{B}_{\alpha}$ for each $\alpha$, will serve as a basis for the box topology on $\prod_{\alpha \in \mathcal{ J } } X_{\alpha}$.
\end{theorem}

\begin{theorem}{Basis for the Product Topology}{basis_for_the_product_topology}

Suppose the topology on each space $X_{\alpha}$ is given by a basis $\mathcal{B}_{\alpha}$. The collection of all sets of the form
\[
\prod_{\alpha \in \mathcal{ J } } B_{\alpha}
\]

where $B_{\alpha} \in \mathcal{ B } _{\alpha}$ for finitely many indices $\alpha$ and $B_{\alpha}=X_{\alpha}$ for all the remaining indices, will serve as a basis for the product topology $\prod_{\alpha \in \mathcal{ J } } X_{\alpha}$.
\end{theorem}

\begin{definition}{R Omega}{r_omega}
$\mathbb{R}^{\omega}$, the countably infinite product of $\mathbb{R}$ with itself. Recall that
\[
    \mathbb{R}^{\omega}=\prod_{n \in \mathbb{N}} X_{n}
\]
with $ X_{ n }  =  \mathbb{R}  $ for each $ n $ 
\end{definition}


\subsection{The Metric Topology}

\begin{definition}{A metric}{metric}
A metric on a set $X$ is a function
\[
    d: X \times X \to \mathbb{R} 
\]
having the following properties:
\begin{enumerate}
       \item $d(x, y) \geq 0$ for all $x, y \in X$; equality holds if and only if $x=y$.
       \item $d(x, y)=d(y, x)$ for all $x, y \in X$.
       \item Triangle Inequality: $d(x, y)+d(y, z) \geq d(x, z)$, for all $x, y, z \in X$.
\end{enumerate}
\end{definition}

\begin{example}{Discrete Metric}{discrete_metric}
 $d: X \times X \rightarrow \mathbb{R}$ given by

\[
d(x, y)= \begin{cases}0 & x=y \\ 1 & \text { otherwise }\end{cases}
\]
\end{example}


\begin{definition}{Epsilon Ball}{epsilon_ball}
Given $\epsilon>0$, consider the set
\[
B_{d}(x, \epsilon)=\{y \mid d(x, y)<\epsilon\}
\]
of all points $y$ whose distance from $x$ is less than $\epsilon$. It is called the $\epsilon$-ball centered at $\boldsymbol{x}$. Sometimes we omit the metric $d$ from the notation and write this ball simply as $B(x, \epsilon)$, when no confusion will arise.
\end{definition}


\begin{lemma}{Epsilon Ball Contains Another}{epsilon_ball_contains_another}
    Let $ x \in X $ then for every $ B\left( x, \varepsilon  \right)  $ there is $ y \in B\left( x, \varepsilon  \right)  $ and $ \delta \in  \mathbb{R} ^{ +  }  $ such that 
    \[
    B\left( y, \delta  \right) \subseteq B\left( x, \varepsilon  \right) 
    \]
\end{lemma}
\begin{proof}
    \begin{itemize}
        \item Let $ y \in  B\left( x, \varepsilon \right)  $ we claim  $ B\left( y, \delta  \right)  $ where $ \delta =  \varepsilon -  d\left( x,y \right)  $ works.
        \item Note $ B\left( y, \delta  \right) \subseteq B\left( x, \varepsilon  \right)  $, since given any $ z \in B\left( y, \delta  \right)  $ we have that $ d\left( y, z \right) < \varepsilon  - d\left( x,y \right)$, re-arranging gives us $ d\left( x,y \right) +  d\left( y, z \right) < \varepsilon   $ and by the triangle inequality we get $ d\left( x, z  \right) < \varepsilon  $ therefore $ z \in  B\left( x, \varepsilon  \right)  $
    \end{itemize}
\end{proof}


\begin{proposition}{Epsilon Balls Form A Basis}{epsilon_balls_form_a_basis}
The collection $ \mathcal{ B }   $ of all $\epsilon$-balls $B_{d}(x, \epsilon)$, for $x \in X$ and $\epsilon>0$, is a basis for a topology on $X$
\end{proposition}
\begin{proof}
    \begin{itemize}
        \item Let $ x \in  X $ then clearly $ x \in  B _{ d }  \left( x, \varepsilon    \right)  $ works for any $ \varepsilon  \in  \mathbb{R} ^{ +  }   $ 
        \item Let $  B _{ 1 } , B _{ 2 } \in  \mathcal{ B }    $ and and let $ y \in  B _{ 1 } \cap B _{ 2 }  $, for each of $ B _{ 1 }  $ and $ B _{ 2 }  $ \hyperref[lemma:epsilon_ball_contains_another]{we have} $ \delta _{ 1 }  $ and $ \delta _{ 2 }  $ respectively with $ B\left( y, \delta _{ 1 }  \right) \subseteq B _{ 1 }  $ and $ B\left( y, \delta _{ 2 }  \right) \subseteq B _{ 2 }  $, by taking $ \delta =  \min\left( \delta _{ 1 }  , \delta _{ 2 }   \right)  $ then we have $ B\left( y, \delta  \right) \subseteq B _{ 1 } \cap  B _{ 2 }   $ as needed.
    \end{itemize}
\end{proof}



\begin{definition}{Metric Topology}{metric_topology}
If $d$ is a metric on the set $X$, then the collection of all $\epsilon$-balls $B_{d}(x, \epsilon)$, for $x \in X$ and $\epsilon>0$, is a basis for a topology on $X$, called the metric topology induced by $d$.
\end{definition}


\begin{definition}{Bounded Subset of a Metric Space}{bounded}
Let $X$ be a metric space with metric $d$. A subset $A$ of $X$ is said to be bounded if there is some number $M \in  \mathbb{R}$ such that
\[
d\left(a_{1}, a_{2}\right) \leq M
\]
for every pair of points $ a_{ 1 } , a_{ 2 } \in  A $ 
\end{definition}


\begin{example}{Function on R omega}{function_on_r_omega}
    Consider a function $ h : \mathbb{R} ^{ \omega  }  \to \mathbb{R} ^{ \omega  }  $ defined by 
    \[
    h\left( x_{1} , x_{2} , \ldots \right) = \left( \alpha _{ 1 } x _{ 1 } , \alpha _{ 2 } x _{ 2 }, \ldots  \right)
    \]
    with $ \alpha _{ 1 } , \alpha _{ 2 } , \ldots  \in  \mathbb{R} \setminus \left\{ 0 \right\}  $ 
    \begin{enumerate}
        \item Is $ h $  continuous, when $ \mathbb{R} ^{ \omega  }  $  is given the product topology?

        \item Is $ h $  continuous, when $ \mathbb{R} ^{ \omega  }  $ is given the box topology?

        \item Is $ h $  continuous, when $ \mathbb{R} ^{ \omega  }  $ is given the uniform topology?
    \end{enumerate}
\end{example}
\begin{proof}
    \begin{enumerate}
        \item 
        \begin{itemize}
            \item Take an arbitrary basis element from the product topology, that is: 
            \[
            \prod_{\alpha \in \mathcal{ J } } B_{\alpha}
            \]
            where $B_{\alpha} \in \mathcal{ B } _{\mathbb{R} }$ for finitely many indices $\alpha$ and $B_{\alpha}=\mathbb{R} $ for all the remaining indices.
        \item[$ \alpha:$]  Now note the following
            \begin{itemize}
                \item The inverse image of each of these intervals is a scaled version of itself therefore it is still an interval of $ \mathbb{R}  $, and it is therefore still open with respect to $ \mathbb{R}$ 
                \item The inverse image of $ \mathbb{R}  $ under any scaling is still $ \mathbb{R}  $ 
            \end{itemize}
            \item Therefore $ h ^{-1} \left( \prod _{ \alpha  \in \mathcal{ J }   } B _{ \alpha  }  \right)  $ is a product of finitely many intervals, and infinitely many $ \mathbb{R}  $'s and so it is open with respect to $ \mathbb{R} ^{ \omega  }  $ as needed.
        \end{itemize}
        \item 
        \begin{itemize}
            \item Given an arbitrary basis element of the box topology we have
            \[
            \prod_{\alpha \in \mathcal{ J } } B_{\alpha}
            \]
            where $B_{\alpha} \in \mathcal{B}_{ \mathbb{R} }$ for each $\alpha$
            \item Due to $ \alpha  $ we know that each of the intervals become new scaled intervals and so $ h ^{-1} \left( \prod _{ \alpha \in \mathcal{ J } B _{ \alpha  }   }  \right)  $ is a product of open intervals and is therefore open in with respect $ \mathbb{R} ^{ \omega  }  $ equipped with the box topology 
        \end{itemize}
        \item Recall that a basis for the uniform topology are epsilon balls with radius less then $ 1 $ if that's the case, then so long as every $ \alpha _{ i }  $ is greater than $ 1 $ then the inverse image scales them to be even smaller, resulting in an open set, otherwise the function $ h $ wouldn't be continuous, since all the $ \alpha _{ i }  $'s are greater than one, it is continouous.
    \end{enumerate}
\end{proof}




\section{Connectedness and Compactness}


\begin{definition}{Connected Space}{connected}
Let $X$ be a topological space. A separation of $X$ is a pair $U, V$ of disjoint nonempty open subsets of $X$ whose union is $X$. The space $X$ is said to be connected if there does not exist a separation of $X$.
\end{definition}


Notice that $ U, V $ are actually clopen, as $ X \setminus  U =  V $ and $ X \setminus  V = U $ stating that $ V $ and $ U $ are closed as well.

\begin{example}{Closed and Bounded, not Compact}{closed_and_bounded_not_compact}
A metric space $X$  and a closed and bounded subspace $Y$ of  $X$  that is not compact.
\end{example}


\begin{itemize}
    \item Consider the set $ X =  \left\{ \frac{1}{n}: n \in  \mathbb{N} ^{ +  }  \right\}  $, with the \hyperref[example:discrete_metric]{discrete metric}, it is bounded because the for any two points $ a, b \in X, d\left( a, b \right)  \le 1 $  %todo{closed and bounded proof}
    \item Let $ X $ be an infinite set and let consider the discrete metric on that set,  the metric topology which it induces (call it $ \mathcal{ T }  $)  is the discrete topology of $ X $. Therefore if we consider any subset $ Y $ of $ X $ it is closed, as $ X \setminus Y \in  \mathcal{ T }  $ (remember it's the discrete topology). But the open covering $ \left\{ \left\{ x \right\} : x \in  X \right\}  $ has no finite subcollection which also covers $ X $.
\end{itemize}

\begin{example}{R omega Connected?}{r_omega_connected?}
Consider the product, uniform, and box topologies on $\mathbb{R} ^{\omega}$. In which topologies is $\mathbb{R} ^{\omega}$ connected?
\end{example}
\begin{proof}
    \begin{enumerate}
        \item 
        \begin{itemize}
            \item Consider the product topology, we will show that $ \mathbb{R} ^{ \omega  }  $ is not connected. Consider the set $ A $ of \hyperref[definition:bounded_sequence]{bounded sequences} in $ \mathbb{R} ^{ \omega  }  $ , and the complement of $ A $, namely the unbounded sequence of $ \mathbb{R} ^{ \omega  }  $ let's label this set as $ B =  \mathbb{R}  ^{ \omega  } \setminus  A  $.
            \begin{itemize}
                \item Note that $ A $ is open in the box topology as if we fix any $ \varepsilon > 0 $ we may define for any $ \vec{x}  \in  \mathbb{R} ^{ \omega  }  $ 
                    \[
                    U _{ \vec{x}  }  \stackrel{\mathtt{D}}{=} \left( x _{ 1 } -  \varepsilon , x _{ 1 } +  \varepsilon  \right) \times \left( x _{ 2 } -  \varepsilon , x _{ 2 } +  \varepsilon  \right) \times \ldots
                    \]
                \item If $ \vec{a}  $ is a bounded sequence then $ U $ is a set of bounded sequences, as it only ever adds constants to the terms of $ \vec{a}  $, therefore $ \vec{a}  \in  U _{ \vec{a}  } $ and $ U _{ \vec{a}  }  \subseteq A $, therefore $ A $ is open.
                \item If $ \vec{b}  $ is an unbounded sequence then $ U $ is a set of unbounded sequences, as adding a constant to every element to an unbounded sequence yields an unbounded sequence. So for every $ \vec{b} \in  B  $ there we have $ \vec{b} \in  U _{ \vec{b}  }  $ and $ U _{ \vec{b}  } \subseteq B $ so $ B $ is open.
            \end{itemize}
            \item Therefore $ A $ non-trivial set which is both open and closed in $ \mathbb{R} ^{ \omega  }  $ and so it is disconnected.
        \end{itemize}
        \item Note that $ \mathbb{R} ^{ \omega  }  $ is closed with the uniform topology by following the same proof with $ \varepsilon = 1 $ 
        \item 
        \begin{itemize}
            \item Now if we consider $ \mathbb{R} ^{ \omega  }  $ with the product topology we will see that $ \mathbb{R} ^{ \omega  }  $ is connected.
            \item Let $ \mathbb{R} _{ 0 }^{ n } \stackrel{\mathtt{D}}{=} \left\{ \left( x _{ 1 } , x _{ 2 } , \ldots \right): ~\text{where for}~  i > n  ~\text{we have that}~   x _{ i } = 0   \right\}    $ 
            \item $ \mathbb{R} _{ 0 }^{ n }   $ is homeomorphic to $ \mathbb{R} ^{ n }  $ and since $ \mathbb{R}  $ is connected, and the finite cartesian product of connected spaces is connected, we get that $ \mathbb{R} _{ 0 }^{ n }  $ is connected.
            \item We now will show that the closure of $ \mathbb{R} ^{ \infty  }  $ is equal to $ \mathbb{R} ^{ \omega  }  $ and then by the fact that the closure of a connected set is closed we obtain that $ \mathbb{R} ^{  \omega  }  $ is connected.
            \item To show that $ R ^{ \omega  }  $ is equal to the closure of $ \mathbb{R} ^{ \infty  }  $ we proceed by using the fact that a point is part of the closure of a set if and only if every basis element intersects it:
            \begin{itemize}
                \item Let $ \vec{a} \in \mathbb{R} ^{ \omega  }   $ and let $ B =  \prod _{ i \in \mathbb{N}   }  B _{ i }  $ be some basis element for the product topology with $ \vec{a}  \in  U $, now we have to show this set intersects $ \mathbb{R} ^{ \infty  }  $. Since only finitely many of the $ B _{ i }  $ are equal to basis elements and the rest are equal to $ \mathbb{R}  $ that means there is some $ N \in  \mathbb{N}  $ such that $ \forall j \in  \mathbb{N} ^{ \ge N }, B _{ j } = \mathbb{R}   $, therefore the point $ \vec{b} = \left( a_{1} , a_{2} , \dotsc  , a_{N - 1} , a_{N}, 0, 0, 0, \ldots \right) \in B \cap \mathbb{R} ^{ \infty  }  $
            \end{itemize}
        \end{itemize}
    \end{enumerate}
\end{proof}



\begin{proposition}{Connected Implies Closure Connected}{connected_implies_closure_connected}
    Let $ A \subseteq X $ be a connected subspace of $ X $ then $ \overline{A}  $ is also connected.
\end{proposition}
\begin{proof}
    \begin{itemize}
        \item Suppose for the sake of contradiction that there is a separation $ B, C $  of $ \overline{A}  $, then $  B \cup  C =  \overline{A}  $ and note that that means that $ A \subseteq B \cup  C $ since $ A \subseteq \overline{A}  $  so $ \left( B \cup C \right) \cap  A =  \overline{A}  \cap  A =  A $. 
        \item Therefore $ \left( B \cap A \right) \cup  \left( C \cap  A \right) =  A$ is a separation of $ A $ noting that $ B \cap A $ and $ C \cap A $ are non-empty because ...
    \end{itemize}
\end{proof}


\begin{definition}{Totally Disconnected}{totally_disconnected}
    A topological space is totally disconnected if it's only connected subspaces are one-point sets.
\end{definition}


Consider $ \mathbb{R} _{ \ell } $ if we have $ \left\{ a \right\}  $ and $ \left\{ b \right\}  $ then the open sets $ \left( - \infty , b \right) $ and $ \left[ b, \infty  \right) $ is a separation of $ \mathbb{R} _{ \ell }  $, therefore it's disconnected. Similarly for any two points $ a, b $  in $ \mathbb{Q}  $ we have some irrational number $ r $ between the two, and thus $ \left( - \infty , r \right) _{ \mathbb{Q}  } , \left( r, \infty  \right) _{ \mathbb{Q}  }  $ is a separation thus $ \mathbb{Q}  $ is totally disconnected under this topology. 

Let's look at $ \mathbb{R}  $ with the fite complement topology. Right off the bat, we note that if a set is finite in $ \mathbb{R}  $ it's complement must be infinite therefore if $ \mathbb{R}  $ was completely disconnected it would mean for any singleton sets, we have a separation $ U, V $, but that means that $ U $ and $ V $ must be infinite, but we then get a contraditiction as $ \mathbb{R} =  U \cup V $ so $ R \setminus U = V $, now since $ U $ was open this implies that $ V $ is finite, which is a contradiction. This idea may be extended to $ \mathbb{R} ^{ 2 }  $.


\subsection{Compact Spaces}

\begin{definition}{Covering}{covering}
A collection $A$ of subsets of a space $X$ is said to cover $X$, or to be a covering of $X$, if the union of the elements of $A$ is equal to $X$. It is called an open covering of $X$ if its elements are open subsets of $X$.
\end{definition}

\begin{definition}{Compact Space}{compact_space}
A space $X$ is said to be compact if every open covering $A$ of $X$ contains a finite subcollection that also covers $X$.
\end{definition}

\begin{lemma}{Covering Yields Finite Covering if and only if Compact}{covering_yields_finite_covering_if_and_only_if_compact}
Let $Y$ be a subspace of $X$. Then $Y$ is compact if and only if every covering of $Y$ by sets open in $X$ contains a finite subcollection covering $Y$.
\end{lemma}

\begin{proposition}
{Closed Graph Yields Continuity for Compact Hausdorff
Domain}{closed_graph_yields_continuity_for_compact_hausdorff_domain}
Let \(f: X \to  Y\) be a map from a topological space \(X\) to a
compact Hausdorff space \(Y\). Show that if the graph of \(f\) is closed in \(X
\times Y\), then \(f\) is continuous.
\end{proposition}
\begin{proof}
    We know that \( \Gamma \left( f \right)  \) is closed in \( X \times Y \),
    additionally since \( X \) and \( C \) are also closed in \( X \) and \( Y
    \) respectively, we can see that \(  X \times  C \) is closed in \( X \times
    Y\), therefore the set \( \Gamma \left( f \right) \cap X \times C \) is 
    closed in \( X \times Y \). We also note that compact subsets of hausdorff
    spaces are closed, and that closed subspaces of compact spaces are compact,
    so that in a hausdorff space intersections of compact spaces are compact.
    Due to this we may deduce that  \( X \times C \) is compact, and that \(
    \Gamma \left( f \right) \cap \left( X \times C \right)   \) is compact. Now
    we consider the projection mapping \( \pi _{ X }  : X \times Y \to X  \)
    since the continuous image of a compact space compact we can see that the
    set \( \pi _{ X } \left( \Gamma \left( f \right) \land  \left( X \times C
    \right)   \right)  \) is a compact set, therefore since \( X \) is
    hausdorff it's also a closed set. Now if the following happened to be true
    the proof would be complete:
    \[
    f ^{-1} \left( C \right) = \pi _{ X } \left( \Gamma \left( C \right) \cap
    \left( X \times C \right)  \right) 
    \]
    it does happen to be true, and we'll prove it. \( \subseteq  \) Let \( x \in
    f ^{-1} \left( C \right) \) so that \( f\left( x \right) \in  C \), we must
    show that \( x \in  \pi _{ X } \left( \Gamma \left( C \right) \cap \left( X
    \times C \right)  \right) \), that is \( x =  \pi _{ X } \left( \left( a, b
    \right)  \right)  \) where \( \left( a, b \right)  \in \Gamma \left( C \right) \cap \left( X
    \times C\right)  \) so we require that \( b = f\left( a \right)  \) and
    that \( f\left( a \right) \in  C \), all in all we require \( x =  \pi _{ X } \left(
    \left( a, f\left( a \right)  \right)  \right) = a \) for some \( a \in  X
    \), this simply works if we let \( a =  x \) as we have noted that \(
    f\left( x \right) \in  C \) . \( \supseteq  \) Suppose now that \( x \in
    \Gamma \left( f \right) \cap \left( X \times C \right)  \) in this case we
    know that \( x = \pi _{ X } \left( a, f\left( a \right) \in C  \right) = a  \) for
    some \( a \in  X \) then \( f\left( x \right) = f\left( a \right) \in C \)
    as needed.
\end{proof}



\end{document}
