\begin{proposition}{Little O and Differentiability
Equivalence}{little_o_and_differentiability_equivalence}
  A function $f$ is differentiable at the point $\overline{x}$ if and only if
  there is a number $ \alpha \in \mathbb{R}$ for any $h \in \mathbb{R}$ we have 
  \[
  f\left(\overline{x}  +  h\right) = f\left(\overline{x}\right)  +  h \alpha  +  E\left(h\right)
  \]
  Where $E \in o\left(h\right) $ 
\end{proposition}
\begin{proof}
    \begin{itemize}
      \item $\Rightarrow$ 
        \begin{itemize}
          \item Assume that $f$ is differentiable, therefore we have
              \[
              f ^{ \prime } \left( \overline{x} \right) =  \lim_{ h \to 0 }
              \frac{f\left( \overline{x} +  h \right) -  f\left( x \right) }{h}
              \]
            \item Equivalently:
              \[
               \lim_{ h \to 0 } \frac{f\left( \overline{x} +  h \right) -
               f\left( x \right) }{h} -  f ^{ \prime } \left( \overline{x}
               \right) = 0
              \]
            \item Since $ \lim_{ h \to 0 } f ^{ \prime } \left( \overline{x}
            \right) = f ^{ \prime } \left( \overline{x} \right)  $ as it's a
            constant, we may use the limit law to deduce:
            \begin{gather*}
                \lim_{ h \to 0 } \left( \frac{f\left( \overline{x} + h \right) -
                f\left( x \right) }{h} - f ^{ \prime } \left( \overline{x}
                \right) \right) = 0 \\
                \Updownarrow \\
                \lim_{ h \to 0 } \frac{f\left( \overline{x} + h \right) -
                f\left( \overline{x} \right) - h f ^{ \prime } \left(
                \overline{x} \right) }{h}= 0
            \end{gather*}
            \item That is to say that $ f\left( \overline{x} + h \right) -
            f\left( \overline{x} \right) - h f ^{ \prime } \left( \overline{x}
            \right) \in o\left( h \right)  $, set $ E\left( h \right) = f\left(
            \overline{x} + h \right) - f\left( \overline{x} \right) - h f ^{
            \prime } \left( \overline{x} \right)  $ then we have that
            \[
            f\left( \overline{x} + h \right) =  f\left( \overline{x} \right) +
            h f ^{ \prime } \left( \overline{x} \right) +  E\left( h \right) 
            \]
            where we've taken $ \alpha = f ^{ \prime } \left( \overline{x}
            \right)  $ 
        \end{itemize}
      \item $\Leftarrow$ 
        \begin{itemize}
          \item Assume that there is a number $\alpha \in \mathbb{R}$ and $E \in
          o\left(h\right)$, so that for any $h \in \mathbb{R}$  we have:
            \[
            f\left(\overline{x}  +  h\right) = f\left(\overline{x}\right)  +  h
            \alpha+  E\left(h\right) \Leftrightarrow l\left(h\right) =
            f\left(\overline{x}  +  h\right)  -  f\left(\overline{x}\right)  -
            h \alpha
            \]
            thus $f\left(\overline{x}  +  h\right)  -
            f\left(\overline{x}\right)  -  h \alpha \in o\left(h\right)$.
            \item So then by definition of little-o, we get:
            \[
           \lim_{h\to0} \frac{f\left(\overline{x}  +  h\right)  -
           f\left(\overline{x}\right)  -  h \alpha }{h} = 0
            \]
            \item Since $ \lim_{ h \to 0 } \alpha = \alpha $ we may add it to
            both sides using the limit law on the left to obtain:
            \[
            \lim_{ h \to 0 } \left( \frac{f\left( \overline{x} +  h \right) -
            f\left( \overline{x} \right) -  h \alpha }{h} + \alpha  \right)=
            \alpha \Leftrightarrow \lim_{ h \to 0 } \frac{f\left( \overline{x} +
            h \right) -  f\left( \overline{x} \right) }{h} = \alpha 
            \]
            Meaning that $ \alpha = f ^{ \prime } \left( \overline{x} \right)  $
            and that $ f $ is differentiable
        \end{itemize}
    \end{itemize}
\end{proof}
