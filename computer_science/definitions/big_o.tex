\begin{definition}{Big-O}{big_o}
Let $ f, g: \mathbb{N} \to  \mathbb{R} ^{ \ge 0 }  $, then we define the set 
\[
    \mathcal{ O } \left( g \right) \stackrel{\mathtt{D}}{=}  \left\{ j: \exists c, B \in  R ^{ +  }, \forall n \in \mathbb{N} , n \ge B \rightarrow j\left( n \right) \le c g\left( n \right)    \right\} 
\]
And we say that $ f $ is in the big-O of $ g $ when $ f \in  \mathcal{ O } \left( g \right)  $ .
\end{definition}
