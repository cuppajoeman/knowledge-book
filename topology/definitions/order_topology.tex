\begin{definition}
{Order Topology}{order_topology}
Let \(X\) be a set with a simple order relation; assume \(X\) has more than one
element. Let \(\mathcal{B}\) be the collection of all sets of the following
types:
\begin{enumerate}
    \item All open intervals \((a, b)\) in \(X\).
    \item All intervals of the form \(\left[a_{0}, b\right)\), where \(a_{0}\)
    is the smallest element (if any) of \(X\).
    \item All intervals of the form \(\left(a, b_{0}\right]\), where \(b_{0}\)
    is the largest element (if any) of \(X\). The collection \(\mathcal{B}\) is
    a basis for a topology on \(X\), which is called the order topology.
\end{enumerate}
If \(X\) has no smallest element, there are no sets of type (2), and if \(X\)
has no largest element, there are no sets of type (3).
\end{definition}
