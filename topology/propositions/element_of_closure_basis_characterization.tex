\begin{proposition}{Element of Closure Basis
Characterization}{element_of_closure_basis_characterization}
Let \( A \) be a subset of a topological space \( X \), then
supposing that the topology of \( X \) is given by a basis, then \( x \in 
\overline{A}  \) if and only if every basis element \( B \) containing \( x \)
intersects \( A \) 
\end{proposition}
\begin{proof}
Since every basis element is part of the topology it generates
(\ref{corollary:basis_is_a_subset_of_the_topology_it_generates}) then this means
that every \( B \in  \mathcal{ B }   \) are open sets. Now suppose that \( x \in
\overline{A} \) then by \ref{proposition:element_of_closure_characterization}
every open set \( U \) containing \( x \) intersects \( A \), then consider
every \( B \in  \mathcal{ B }   \) where \( x \in  B \), since \( B \) is open
then we know that \( B \) and \( A \) intersect, so every basis element \( B \)
containing \( x \) intersects \( A \). \\
For the other direction we assume that every basis element that contains \( x \)
intersects \( A \). Now considering \( U \in \mathcal{ T } _{ \mathcal{ B }   }
\) by \hyperref[definition:topology_generated_by_a_basis_(basis_existance)]{the
basis existance definition of a topology generated by a basis}, we can see that
an open set \( U \) that contains \( x \) contains a basis element which
intersects \( A \) and thus \( U \) also intersects \( A \) since \( B \subseteq
U\), thus by \ref{proposition:element_of_closure_characterization} \( x \in
\overline{A}  \), as needed.
\end{proof}
