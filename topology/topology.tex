
\chapter{Topology}

\section{Topological Spaces and Continuous Functions}

\begin{definition}
{Topology on a Set}{topology_on_a_set}
Let \(X\) be a set. \(A\) collection \(\mathcal{T} \subseteq \mathcal{P}(X)\) of
subsets of \(X\) is called a topology on \(X\) provided that the
following three properties are satisfied:
\begin{enumerate}
    \item \(\emptyset \in \mathcal{T}\) and \(X \in \mathcal{T}\).
    \item \(\mathcal{T}\) is closed under finite intersections. That is, given
    any finite collection \(U_{1}, \ldots, U_{n}\) of sets in \(\mathcal{T}\),
    their common intersection \(U_{1} \cap \cdots \cap U_{n}\) is also an
    element of \(\mathcal{T}\).
    \item \(\mathcal{T}\) is closed under arbitrary unions. That is, if
    \(\left\{U_{\alpha}: \alpha \in I\right\}\) is a family of sets in
    \(\mathcal{T}\) (here \(I\) is some indexing set, which may be infinite),
    then their union \(\bigcup_{\alpha \in I} U_{\alpha}\) is also an element of
    \(\mathcal{T}\).
\end{enumerate}
\end{definition}


\begin{definition}{Open Subset}{open_subset}
The elements \( U \in  \mathcal{ T }   \) of a topology on \( X \) are called
open subsets of \( X \) or just ``open sets''.
\end{definition}


\begin{definition}
{Topological Space}{topological_space}
Given a set \(X\) and a topology \(\mathcal{T}\) on \(X\), the pair \((X,
\mathcal{T})\) is called a topological space. 
\end{definition}


\subsection{Basis}

\begin{definition}
{Basis For a Set}{basis_for_a_set}
Let \(X\) be a set. A collection of sets \(\mathcal{B} \subseteq
\mathcal{P}(X)\) is called a basis on \(X\) if the following two properties
hold:
\begin{enumerate}
    \item \(\mathcal{B}\) covers \(X\). That is: \(\forall x \in X, \exists B
    \in \mathcal{B}\) such that \(x \in B \). Or, more concisely, \(X = \bigcup
    \mathcal{B}\).
        \begin{itemize}
            \item The reason why \(X = \bigcup \mathcal{B} \) is that \(\bigcup
            _{x \in X} B_{x} \) contains every \(x \in X \) and is a
            subset of \(X \) since each \(B_{x} \) is a subset of \(X \)
            therefore \(\bigcup _{x \in X} B_{x} = X \)
        \end{itemize}
    \item \(\forall B_{1}, B_{2} \in \mathcal{B}, \forall x \in B_{1} \cap
    B_{2}, \exists B \in \mathcal{B}\) such that \(x \in B \subseteq B_{1} \cap
    B_{2}\).
    \begin{itemize}
        \item Given a point \(x\) in the intersection of two elements of the
        basis, there is some element of the basis containing \(x\) and contained
        in this intersection.
    \end{itemize}
\end{enumerate}
We will call the elements of \(\mathcal{B} \) basis elements.
\end{definition}
Note that if we are trying to prove a set is a basis for a set, then in the
second condition if \( B _{ 1 } \cap B _{ 2 } = \varnothing  \) then the second
condition holds vacuously, and therefore we may assume \( B _{ 1 } \cap  B _{ 2
} \neq \varnothing  \) when proving it.


\begin{definition}
{Topology Generated by a Basis (Basis
    Existance)}{topology_generated_by_a_basis_(basis_existance)}
    Let \(X\) be a set and \(\mathcal{B}\) a basis on \(X\)
    \[
    \mathcal{T}_{\mathcal{B}} = \left\{U \subseteq X: \forall x \in U, \exists B \in
    \mathcal{B} \text{such that} x \in B \subseteq U \right\}
    \]
\end{definition}


\begin{corollary}
{Basis is a subset of the Topology it Generates}{basis_is_a_subset_of_the_topology_it_generates}
Let \(\mathcal{B} \) be a basis for a set \(X \), then
\[
\mathcal{B} \subseteq \mathcal{T} _{\mathcal{B}}
\]
\end{corollary}
\begin{proof}
   Let \(B \in \mathcal{B} \) then we note that \(B \subseteq X \) since
   \(\mathcal{B} \subseteq \mathcal{P} \left(X\right) \). Additionally
   for every \(x \in B \) \(B \) itself is an element from \(\mathcal{B}
   \) such that \(x \in B \subseteq B \), so \(B \in \mathcal{T} _{
   \mathcal{B}} \)
\end{proof}


\begin{lemma}
{Intersection of Two Elements from The Topology Generated by a
    Basis (Basis Existance) is
Closed}{intersection_of_two_elements_from_the_topology_generated_by_a_basis_(basis_existance)_is_closed}
    We will prove that for any \(X, Y \in \mathcal{T}_{B}\), we have that
    \[
    X \cap Y \in \mathcal{T}_{\mathcal{B}}
    \]
\end{lemma}
\begin{proof}
    \begin{itemize}
        \item We know two things
            \begin{gather*}
                \forall x \in X, \exists B_{X} \in \mathcal{B} \text{such that
} x \in B_{X} \subseteq X\\
                \forall x \in Y, \exists B_{Y} \in \mathcal{B} \text{such that
} x \in B_{Y} \subseteq Y\\
            \end{gather*}
        \item And we would like to show that
            \[
                \forall x \in X \cap Y, \exists B_{XY} \in \mathcal{B} \text{
                such that} x \in B_{XY} \subseteq X \cap Y\\
            \]
        \item Let \(x \in X \cap Y\), then \(x \in X\) and \(x \in Y\), thus by the
            two facts we have \(B_{X}\) and \(B _{Y}\) from \(\mathcal{B}\).
        \begin{itemize}
            \item By clause two of the definition of basis, we can see that
                \[
                \forall x \in B_{X} \cap B_{Y}, \exists B \in \mathcal{B} \text{
                such that} x \in B \subseteq B_{X} \cap B_{Y} \subseteq X \cap
                Y
                \]
            \item So we set \(B_{XY} = B\) to finish the proof.
        \end{itemize}
    \end{itemize}
\end{proof}


\begin{proposition}
{Topology Generated By a Basis (Basis Existance) is a
Topology}{topology_generated_by_a_basis_(basis_existance)_is_a_topology}
    \begin{center}
        We claim that \(\mathcal{T}_{B}\) is a topology
    \end{center}
\end{proposition}
\begin{proof}
    \begin{itemize}
        \item Vacuously we see that \(\varnothing \in \mathcal{T} _{B}\), and
        \(X \in \mathcal{T} _{B}\) by the first clause of the definition of a
        basis
        \item We will show that \(\bigcap_{i = 1}^{n} U _{i} \in
        \mathcal{T}_{B}\) where \(\forall j \in \left[ n \right], U_{j} \in
        \mathcal{T} _{B}\) by induction
        \begin{itemize}
            \item Base Case
                \[
                \bigcap_{i = 1}^{1} U_{i} = U_{1} \in \mathcal{T}_{B}
                \left(\text{by assumption}\right)
                \]
            \item Inductive Step
            \begin{itemize}
                \item Suppose \(k \in \mathbb{N} ^{+}\) and assume it's true for
                \(k\) we'll show it holds on \(k + 1\)
                    \[
                    \bigcap_{i = 1}^{k + 1} U_{i} = \bigcap_{i = 1}^{k} U_{i}
                    \cap U_{k + 1}
                    \]
                \item Now by induction hypothesis \(\bigcap_{i = 1}^{k} U_{i}
                \in \mathcal{T} _{B}\) and also we know that \(U_{k + 1} \in
                \mathcal{T}_{B}\), therefore by the fact that this topology is
                closed under union for two elements, we know that
                    \[
                    \bigcap_{i = 1}^{k} U_{i} \cap U_{k + 1} \in \mathcal{T}_{B}
                    \]
                    as needed
            \end{itemize}
        \end{itemize}
        \item We will show that for some index set \(I\) we have
            \[
                \bigcup_{\alpha \in I} U_{\alpha} \in \mathcal{T}_{B}
            \]
        \begin{itemize}
            \item That is
                \[
                    \forall x \in \bigcup_{\alpha \in I} U_{\alpha}, \exists B
                    \in \mathcal{B} \text{such that} x \in B \subseteq
                    \bigcup_{\alpha \in I} U_{\alpha}
                \]
            \item So let \(x \in \bigcup_{\alpha \in I} U_{\alpha}\), therefore
            we know that \(x \in U_{\beta}\) for some \(\beta \in I\), but since
            \(U _{\beta} \in \mathcal{T}_{B}\) we get \(B_{U} \in \mathcal{B}\)
            such that \(x \in B_{U} \subseteq U_{\beta}\)
            \item Take \(B = B_{U}\) and note that
                \[
                x \in B = B_{U} \subseteq U_{\beta} \subseteq \bigcup_{\alpha
                \in I} U _{\alpha}
                \]
                as needed.
        \end{itemize}
    \end{itemize}
\end{proof}


\begin{definition}
{Topology Generated by a Basis
    (Union)}{topology_generated_by_a_basis_(union)}
    Let \(X\) be a set and \(\mathcal{B}\) a basis on \(X\), we define:
    \[
    \mathcal{T}_{\mathcal{B}} = \left\{\bigcup \mathcal{C}: \mathcal{C}
    \subseteq \mathcal{B} \right\}
    \]
    and say that \(\mathcal{T} _{B}\) is called the topology generated by \(
    \mathcal{B}\), note that \(\mathcal{C} \)'s elements are subsets of \(
    X \) (basis elements)
\end{definition}


\begin{proposition}
{Topology Generated By a Basis (Union) is a
Topology}{topology_generated_by_a_basis_(union)_is_a_topology}
    \[
    \mathcal{T}_{\mathcal{B}} \text{is a topology}
    \]
\end{proposition}
\begin{proof}
    \begin{itemize}
        \item \(\varnothing \in \mathcal{T}_{\mathcal{B}} \) is true because
        \(\varnothing \subseteq \mathcal{B} \) and \(\bigcup \varnothing =
        \varnothing \). It's also clear that \(X \in \mathcal{T}_{B} \) since
        from the definition of basis we know that \(X = \bigcup \mathcal{B} \)
        \item We'll show that \(\mathcal{T} _{\mathcal{B}} \) is closed under
        arbitrary unions
        \begin{itemize}
            \item Let \(V_{\alpha}, \alpha \in I \) be elements of \(\mathcal{T}
            _{\mathcal{B}} \) for some indexing set \(I \), one must show that
                \[
                \bigcup _{\alpha \in I} V_{\alpha} \in \mathcal{T}
                _{\mathcal{B}}
                \]
            \item By the definition of \(\mathcal{T} _{\mathcal{B}} \) for each
            \(\alpha \) we have \(\mathcal{C}_ {\alpha} \) such that
            \(V_{\alpha} = \bigcup \mathcal{C} _{\alpha} \), therefore
            \begin{align*}
                \bigcup _{\alpha \in I} V_{\alpha} &= \bigcup _{\alpha \in I}
                \left(\bigcup \mathcal{C} _{\alpha}\right) \\
                &=  \bigcup _{\alpha \in I} \left(\bigcup _{X \in \mathcal{C}
                _{\alpha}} X\right) \\
                &=  \bigcup _{X \in \left(\bigcup _{\alpha \in I} \mathcal{C}
                _{\alpha}\right)} X \\
                &= \bigcup \left(\bigcup _{\alpha \in I} \mathcal{C} _{\alpha}\right)
            \end{align*}
        \end{itemize}
        \item \(\mathcal{T} _{\mathcal{B}} \) is closed under finite
        intersections, we'll show it's closed under pairwise intersections and
        the general result may be proven inductively.
        \begin{itemize}
            \item Suppose \(U = \bigcup \mathcal{A} \) and \(V = \bigcup
            \mathcal{C} \) are elements of \(\mathcal{T} _{\mathcal{B}}
            \) we must show that \(U \cap V \in \mathcal{T} _{\mathcal{B}
            } \)
                \[
                U \cap V = \left(\bigcup \mathcal{A}\right) \cap \left(
                \bigcup \mathcal{C}\right) = \bigcup \left\{A \cap C:
                \enspace \text{where} \enspace A \in \mathcal{A}, C \in
                \mathcal{C} \right\}
                \]
                where the last equality comes from the fact that if a point is
                in the intersection of \(\bigcup \mathcal{A} \) and \(\bigcup
                \mathcal{C} \) then it must be in the intersection of some \(A
                \in \mathcal{A} \) and some \(C \in \mathcal{C} \)
            \item To show this is an element of \(\mathcal{T} _{\mathcal{B}
            } \) all we have to do is to show that the set \(\left\{A \cap C:
            \text{where} A \in \mathcal{A}, C \in \mathcal{C}
            \right\} \subseteq \mathcal{T} _{\mathcal{B}} \) and then use
            the fact that \(\mathcal{T} _{\mathcal{B}} \) is closed under
            arbitrary unions.
            \begin{itemize}
                \item Let \(J \cap K \) be from the set under consideration and
                now our goal is to show that \(J \cap K \in \mathcal{T} _{
                \mathcal{B}} \), that is we have some \(\mathcal{R} \) such
                that \(J \cap K = \bigcup \mathcal{R} \)
                \item Recall that \(J \in \mathcal{A} \subseteq \mathcal{B}
                \subseteq P\left(X\right) \) therefore \(J \subseteq X \) and
                by the same logic \(K \subseteq X \) therefore \(J \cap K
                \subseteq X \).
                \item Thus for every \(x \in J \cap K \) we have \(B_{x} \)
                such that \(x \in B_{x} \subseteq J \cap K \) from the
                definition of basis for a set, so we have
                    \[
                    J \cap K \subseteq \left[ \bigcup _{x \in J \cap K} B_{x
                    } \right] \subseteq J \cap K
                    \]
                    Since each \(B_{x} \subseteq J \cap K \)
                \item In other words
                    \[
                    J \cap K = \left[ \bigcup _{x \in J \cap K} B_{x}
                    \right] = \bigcup \left\{B_{x} : x \in J \cap K \right\}
                    \]
                \item And finally noting that \(\left\{B_{x} : x \in J \cap K
                \right\} \subseteq \mathcal{B} \)
            \end{itemize}
        \end{itemize}
    \end{itemize}
\end{proof}


\begin{proposition}
{Generating by Basis Existance or Union Yields Same
Topology}{generating_by_basis_existance_or_union_yields_same_topology}
\[
\left\{U \subseteq X: \forall x \in U, \exists B \in \mathcal{B}
~\text{such that} x \in B \subseteq U \right\} = \left\{\bigcup \mathcal{C} :
\mathcal{C} \subseteq \mathcal{B} \right\}
\]
\end{proposition}
\begin{proof}
    \begin{itemize}
        \item \(\subseteq \) Let \(U \) be an element of the left hand side,
        we'd like to show it's an element of the right hand side. For each point
        \( x \in  U \) we have \( B _{ x } \in \mathcal{ B }   \) such that \( x
        \in B _{ x } \subseteq U\), therefore \( U = \bigcup _{ x \in  U } B _{
        x}  \) and so \( U \) is an element of the right hand side.
        \item \(\supseteq \) Suppose that \(\mathcal{C} \subseteq \mathcal{B} \)
        then for any \(C \in \mathcal{C} \) we know that \(C \in \mathcal{
        B} \) (it's a basis element) and therefore we know that \(C \) is an
        element of the left hand side since we can take \(B = C \). Then since
        the left hand side is a topology \(\bigcup \mathcal{C} \) is also a
        part of the left hand side as it's an arbitrary union.
    \end{itemize}
\end{proof}


\begin{lemma}
{Basis Criterion}{basis_criterion}
Let \(X\) be a topological space. Suppose that \(\mathcal{C}\) is a collection
of open sets of \(X\) such that for each open set \(U\) of \(X\) and each \(x\)
in \(U\), there is an element \(C\) of \(\mathcal{C}\) such that \(x \in C
\subset U\). Then \(\mathcal{C}\) is a basis for the topology of \(X\).
\end{lemma}
\begin{proof}
    \begin{itemize}
        \item We'll show it's a basis, so let \(x \in X \) now since \(X \)
        is open we know that there is \(C \in \mathcal{C} \) such that
        \(x \in C \subseteq X \), as needed.
        \item We continue to the second condition, so we take \(C _{1}, C _{
        2} \in \mathcal{C} \) and then let \(x \in C _{1} \cap C _{2}
        \) (we are allowed to do that, because if \(C _{1} \cap C _{2} =
        \varnothing \) then the statement holds vacuously). Since \(\mathcal{C
} \) is a collection of open sets we know that \(C _{1} \cap C _{
        2} \) is also open with respect to \(X \), thus by assumption we have \(C
        _{3} \) such that \(x \in C _{3} \subseteq C _{1} \cap C _{2} \)
        \item The statement also claims that \(\mathcal{T} _{\mathcal{C}
} = \mathcal{T} \)where \(\mathcal{T} \) is the topology of \(X \)
        \begin{itemize}
            \item \(\subseteq \) Suppose \(W \) is an element of the topology
            generated by \(\mathcal{C} \), then \(W = \bigcup C \) where \(C
            \subseteq \mathcal{C} \), but note that every element of \(C \)
            is an element of \(\mathcal{T} \) since \(\mathcal{C} \)
            is a collection of open subsets of \(X \), therefore \(W \) being
            an arbitrary union is also open with respect to \(X \), that is \(
            W \in \mathcal{T} \).
            \item \(\supseteq \) Suppose \(U \in \mathcal{T} \) then if
            \(x \in U \) we have \(C \in \mathcal{C} \) such that \(x
            \in C \subseteq U\), then by the basis existance definition of a
            generated topology we can see that \(U \in \mathcal{T} _{\mathcal{
            C}} \)
        \end{itemize}
    \end{itemize}
\end{proof}



\begin{proposition}
{Finer is Equivalent to Basis Containment}{finer_is_equivalent_to_basis_containment}
Let \(\mathcal{B}_{1}\) and \(\mathcal{B}_{2}\) be two bases on a set \(X\),
then \(\mathcal{T}_{\mathcal{B}_{1}} \subseteq \mathcal{T}_{\mathcal{B}_{2}}\)
if and only if for every \(x \in X\) and \(B_{1} \in \mathcal{B}_{1}\)
containing \(x\), there is a \(B_{2} \in \mathcal{B}_{2}\) such that \(x \in
B_{2} \subseteq B_{1}\)
\end{proposition}
\begin{proof}
\begin{itemize}
    \item \( \Rightarrow  \) 
    \begin{itemize}
        \item Suppose \( \mathcal{ T } _{ \mathcal{ B } _{ 1 }    } \subseteq
        \mathcal{ T } _{ \mathcal{ B } _{ 2 }   }    \). Now let \( x \in  X \)
        and \( B _{ 1 } \in  \mathcal{ B } _{ 1 }   \) where \( x \in B _{ 1 }  \)
        (Note that we can do this because any basis covers \( X \)),
        \hyperref[corollary:basis_is_a_subset_of_the_topology_it_generates]{additionally
        we have that every basis is a subset of the topology it generates}
        therefore \( B _{ 1 } \in \mathcal{ T } _{ \mathcal{ B } _{ 1 } }     \) and
        so by assumption we have that \( B _{ 1 } \in \mathcal{ T } _{ \mathcal{
        B} _{ 2 }   }   \) which by definition means that \( \forall x \in B _{
        1} , \exists B _{ 2 } \in  \mathcal{ B } _{ 2 }, \) such that \( x \in B
        _{ 2 } \subseteq B _{ 1 } \) which is exactly what we wanted to show.
    \end{itemize}
    \item \( \Leftarrow  \) 
    \begin{itemize}
        \item Suppose the reverse, so let \( U \in \mathcal{ T } _{ \mathcal{ B
        } _{ 1 }   }   \) we must show that \( U \in  \mathcal{ T } _{ \mathcal{
        B } _{ 2 } }  \). Since \( U \in \mathcal{ T } _{ \mathcal{ B } _{ 1 }
        }    \)  this means that \( \forall x \in  U, \exists B _{ 1 } \in
        \mathcal{ B } _{ 1 }    \) such that \( x \in  B _{ 1 } \subseteq U \)
        is a true statement. Recall that we'd like to prove that \( U \in
        \mathcal{ T } _{ \mathcal{ B } _{ 2 }   }   \), namely that \( \forall x
        \in U\) we have \( B _{ 2 } \in \mathcal{ B } _{ 2 }   \) such that \( x
        \in  B _{ 2 }  \subseteq U\) 
        \item Therefore let \( x \in  U \) by the fact that \( U \in  \mathcal{
        T} _{ \mathcal{ B } _{ 1 }   }   \) we have \( B _{ 1  } \in \mathcal{ B
        } _{ 1 }   \) such that \( x \in \mathcal{ B } _{ 1 } \subseteq U  \). 
        \item By our original assumption (which is \( \forall x \in  X \) and \(
        B _{ 1 }  \in \mathcal{ B } _{ 1 }  \) containing \( x \) we have  \( B
        _{ 2 } \in  \mathcal{ B } _{ 2 }  \) with \( x \in B _{ 2 } \subseteq B
        _{ 1 } \)), we get \( B _{ 2 } \in  \mathcal{ B } _{ 2 }  \) such that
        \( x \in  B _{ 2 } \subseteq B _{ 1 } \) and recall that \( B _{ 1 }
        \subseteq U \) so we have \( x \in B _{ 2 } \subseteq U \) as needed,
        thus \( U \in  \mathcal{ T } _{ \mathcal{ B } _{ 2 }   }   \) 
    \end{itemize}
\end{itemize} 
\end{proof}


\subsection{The Subspace Topology}

\begin{definition}
{Subspace Topology}{subspace_topology}
    Let \(X\) be a topological space with topology \(\mathcal{T}\). If \(Y\) is a subset of \(X\), the collection
    \[
    \mathcal{T}_{Y} = \{Y \cap U \mid U \in \mathcal{T}\}
    \]
    is a topology on \(Y\), called the subspace topology. With this topology,
    \(Y\) is called a subspace of \(X\); its open sets consist of all
    intersections of open sets of \(X\) with \(Y\). We will say that \( U \) is
    open in \( Y \) if it belongs to the topology of \( Y \) and open in \( X \)
    if it belongs to the topology of \( X \).
\end{definition}


\subsection{The Product Topology}


\begin{definition}{Product Topology}{product_topology}
    Let $\mathcal{S}_{\beta}$ denote the collection
    \[
    \mathcal{ S } _{\beta}=\left\{\pi_{\beta}^{-1}\left(U_{\beta}\right) \mid U_{\beta} \text { open in } X_{\beta}\right\}
    \]
    and let $\mathcal{ S } $ denote the union of these collections,
    \[
    \mathcal{S}=\bigcup_{\beta \in J} \mathcal{S}_{\beta}
    \]
    The topology generated by the subbasis $\mathcal{ S } $ is called the product topology. In this topology $\prod_{\alpha \in J} X_{\alpha}$ is called a product space.
\end{definition}


\input{topology/definitions/box_topology}

\begin{theorem}{Basis for the Box Topology}{basis_for_the_box_topology}

Suppose the topology on each space $X_{\alpha}$ is given by a basis $\mathcal{B}_{\alpha}$. The collection of all sets of the form
\[
\prod_{\alpha \in \mathcal{ J } } B_{\alpha}
\]
where $B_{\alpha} \in \mathcal{B}_{\alpha}$ for each $\alpha$, will serve as a basis for the box topology on $\prod_{\alpha \in \mathcal{ J } } X_{\alpha}$.
\end{theorem}

\begin{theorem}{Basis for the Product Topology}{basis_for_the_product_topology}

Suppose the topology on each space $X_{\alpha}$ is given by a basis $\mathcal{B}_{\alpha}$. The collection of all sets of the form
\[
\prod_{\alpha \in \mathcal{ J } } B_{\alpha}
\]

where $B_{\alpha} \in \mathcal{ B } _{\alpha}$ for finitely many indices $\alpha$ and $B_{\alpha}=X_{\alpha}$ for all the remaining indices, will serve as a basis for the product topology $\prod_{\alpha \in \mathcal{ J } } X_{\alpha}$.
\end{theorem}

\begin{definition}{R Omega}{r_omega}
$\mathbb{R}^{\omega}$, the countably infinite product of $\mathbb{R}$ with itself. Recall that
\[
    \mathbb{R}^{\omega}=\prod_{n \in \mathbb{N}} X_{n}
\]
with $ X_{ n }  =  \mathbb{R}  $ for each $ n $ 
\end{definition}


\subsection{The Metric Topology}

\begin{definition}{A metric}{metric}
A metric on a set $X$ is a function
\[
    d: X \times X \to \mathbb{R} 
\]
having the following properties:
\begin{enumerate}
       \item $d(x, y) \geq 0$ for all $x, y \in X$; equality holds if and only if $x=y$.
       \item $d(x, y)=d(y, x)$ for all $x, y \in X$.
       \item Triangle Inequality: $d(x, y)+d(y, z) \geq d(x, z)$, for all $x, y, z \in X$.
\end{enumerate}
\end{definition}

\begin{example}{Discrete Metric}{discrete_metric}
 $d: X \times X \rightarrow \mathbb{R}$ given by

\[
d(x, y)= \begin{cases}0 & x=y \\ 1 & \text { otherwise }\end{cases}
\]
\end{example}


\begin{definition}{Epsilon Ball}{epsilon_ball}
Given $\epsilon>0$, consider the set
\[
B_{d}(x, \epsilon)=\{y \mid d(x, y)<\epsilon\}
\]
of all points $y$ whose distance from $x$ is less than $\epsilon$. It is called the $\epsilon$-ball centered at $\boldsymbol{x}$. Sometimes we omit the metric $d$ from the notation and write this ball simply as $B(x, \epsilon)$, when no confusion will arise.
\end{definition}


\input{topology/lemmas/epsilon_ball_contains_another}

\input{topology/propositions/epsilon_balls_form_a_basis}

\input{topology/definitions/metric_topology}

\begin{definition}{Bounded Subset of a Metric Space}{bounded}
Let $X$ be a metric space with metric $d$. A subset $A$ of $X$ is said to be bounded if there is some number $M \in  \mathbb{R}$ such that
\[
d\left(a_{1}, a_{2}\right) \leq M
\]
for every pair of points $ a_{ 1 } , a_{ 2 } \in  A $ 
\end{definition}


\input{topology/examples/function_on_R_omega}



\section{Connectedness and Compactness}


\input{topology/definitions/connected}

Notice that $ U, V $ are actually clopen, as $ X \setminus  U =  V $ and $ X \setminus  V = U $ stating that $ V $ and $ U $ are closed as well.

\begin{example}{Closed and Bounded, not Compact}{closed_and_bounded_not_compact}
A metric space $X$  and a closed and bounded subspace $Y$ of  $X$  that is not compact.
\end{example}


\begin{itemize}
    \item Consider the set $ X =  \left\{ \frac{1}{n}: n \in  \mathbb{N} ^{ +  }  \right\}  $, with the \hyperref[example:discrete_metric]{discrete metric}, it is bounded because the for any two points $ a, b \in X, d\left( a, b \right)  \le 1 $  %todo{closed and bounded proof}
    \item Let $ X $ be an infinite set and let consider the discrete metric on that set,  the metric topology which it induces (call it $ \mathcal{ T }  $)  is the discrete topology of $ X $. Therefore if we consider any subset $ Y $ of $ X $ it is closed, as $ X \setminus Y \in  \mathcal{ T }  $ (remember it's the discrete topology). But the open covering $ \left\{ \left\{ x \right\} : x \in  X \right\}  $ has no finite subcollection which also covers $ X $.
\end{itemize}

\input{topology/examples/R_omega_connected_in_diff_topos}

\input{topology/propositions/connected_implies_closure_connected}

\input{topology/definitions/totally_disconnected}

Consider $ \mathbb{R} _{ \ell } $ if we have $ \left\{ a \right\}  $ and $ \left\{ b \right\}  $ then the open sets $ \left( - \infty , b \right) $ and $ \left[ b, \infty  \right) $ is a separation of $ \mathbb{R} _{ \ell }  $, therefore it's disconnected. Similarly for any two points $ a, b $  in $ \mathbb{Q}  $ we have some irrational number $ r $ between the two, and thus $ \left( - \infty , r \right) _{ \mathbb{Q}  } , \left( r, \infty  \right) _{ \mathbb{Q}  }  $ is a separation thus $ \mathbb{Q}  $ is totally disconnected under this topology. 

Let's look at $ \mathbb{R}  $ with the fite complement topology. Right off the bat, we note that if a set is finite in $ \mathbb{R}  $ it's complement must be infinite therefore if $ \mathbb{R}  $ was completely disconnected it would mean for any singleton sets, we have a separation $ U, V $, but that means that $ U $ and $ V $ must be infinite, but we then get a contraditiction as $ \mathbb{R} =  U \cup V $ so $ R \setminus U = V $, now since $ U $ was open this implies that $ V $ is finite, which is a contradiction. This idea may be extended to $ \mathbb{R} ^{ 2 }  $.


\subsection{Compact Spaces}

\begin{definition}{Covering}{covering}
A collection $A$ of subsets of a space $X$ is said to cover $X$, or to be a covering of $X$, if the union of the elements of $A$ is equal to $X$. It is called an open covering of $X$ if its elements are open subsets of $X$.
\end{definition}

\begin{definition}{Compact Space}{compact_space}
A space $X$ is said to be compact if every open covering $A$ of $X$ contains a finite subcollection that also covers $X$.
\end{definition}

\begin{lemma}{Covering Yields Finite Covering if and only if Compact}{covering_yields_finite_covering_if_and_only_if_compact}
Let $Y$ be a subspace of $X$. Then $Y$ is compact if and only if every covering of $Y$ by sets open in $X$ contains a finite subcollection covering $Y$.
\end{lemma}
