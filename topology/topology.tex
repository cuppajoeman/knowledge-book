
\chapter{Topology}

\section{Topological Spaces and Continuous Functions}

\begin{definition}
{Topology on a Set}{topology_on_a_set}
Let \(X\) be a set. \(A\) collection \(\mathcal{T} \subseteq \mathcal{P}(X)\) of
subsets of \(X\) is called a topology on \(X\) provided that the
following three properties are satisfied:
\begin{enumerate}
    \item \(\emptyset \in \mathcal{T}\) and \(X \in \mathcal{T}\).
    \item \(\mathcal{T}\) is closed under finite intersections. That is, given
    any finite collection \(U_{1}, \ldots, U_{n}\) of sets in \(\mathcal{T}\),
    their common intersection \(U_{1} \cap \cdots \cap U_{n}\) is also an
    element of \(\mathcal{T}\).
    \item \(\mathcal{T}\) is closed under arbitrary unions. That is, if
    \(\left\{U_{\alpha}: \alpha \in I\right\}\) is a family of sets in
    \(\mathcal{T}\) (here \(I\) is some indexing set, which may be infinite),
    then their union \(\bigcup_{\alpha \in I} U_{\alpha}\) is also an element of
    \(\mathcal{T}\).
\end{enumerate}
\end{definition}


\begin{definition}{Open Subset}{open_subset}
The elements \( U \in  \mathcal{ T }   \) of a topology on \( X \) are called
open subsets of \( X \) or just ``open sets''.
\end{definition}


\begin{definition}
{Topological Space}{topological_space}
Given a set \(X\) and a topology \(\mathcal{T}\) on \(X\), the pair \((X,
\mathcal{T})\) is called a topological space. 
\end{definition}


\begin{proposition}{Open iff every point is in another Open
Set}{open_iff_every_point_is_in_another_open_set}
Given a set \( X \) and a topology \( \mathcal{ T } _{ X }   \), \( U \in
\mathcal{ T } _{ X }   \) if and only if 
\[
\forall x \in  U, \exists V \in  \mathcal{ T } _{ X } \enspace \text{such that}
\enspace x \in  V \subseteq U
\]
\end{proposition}
\begin{proof}
\( \Rightarrow  \) Assuming that \( U \) is open, we let \( x \in  U \) and take
\( V = U \) and surely \( x \in U \subseteq U \), as needed.\\
\( \Leftarrow  \) Suppose the converse, we'd like to show that \( U \) is open,
but note that for each \( x \in  U \) we have \( V _{ x } \in  \mathcal{ T } _{
X}   \) such that \( x \in V _{ x } \subseteq U \), thus \( \bigcup _{ x \in U }
V _{ x } \subseteq U \) but additionally every \( x \in U \) is also in the
union, therefore \( U = \bigcup _{ x \in  U } V _{ x }  \), that is to say that
\( U \) is an arbitrary union of open sets, thus it is open as well by the
definition of a topology.
\end{proof}


\subsection{Basis}

\begin{definition}
{Basis For a Set}{basis_for_a_set}
Let \(X\) be a set. A collection of sets \(\mathcal{B} \subseteq
\mathcal{P}(X)\) is called a basis on \(X\) if the following two properties
hold:
\begin{enumerate}
    \item \(\mathcal{B}\) covers \(X\). That is: \(\forall x \in X, \exists B
    \in \mathcal{B}\) such that \(x \in B \). Or, more concisely, \(X = \bigcup
    \mathcal{B}\).
        \begin{itemize}
            \item The reason why \(X = \bigcup \mathcal{B} \) is that \(\bigcup
            _{x \in X} B_{x} \) contains every \(x \in X \) and is a
            subset of \(X \) since each \(B_{x} \) is a subset of \(X \)
            therefore \(\bigcup _{x \in X} B_{x} = X \)
        \end{itemize}
    \item \(\forall B_{1}, B_{2} \in \mathcal{B}, \forall x \in B_{1} \cap
    B_{2}, \exists B \in \mathcal{B}\) such that \(x \in B \subseteq B_{1} \cap
    B_{2}\).
    \begin{itemize}
        \item Given a point \(x\) in the intersection of two elements of the
        basis, there is some element of the basis containing \(x\) and contained
        in this intersection.
    \end{itemize}
\end{enumerate}
We will call the elements of \(\mathcal{B} \) basis elements.
\end{definition}
Note that if we are trying to prove a set is a basis for a set, then in the
second condition if \( B _{ 1 } \cap B _{ 2 } = \varnothing  \) then the second
condition holds vacuously, and therefore we may assume \( B _{ 1 } \cap  B _{ 2
} \neq \varnothing  \) when proving it.


\begin{definition}
{Topology Generated by a Basis (Basis
    Existance)}{topology_generated_by_a_basis_(basis_existance)}
    Let \(X\) be a set and \(\mathcal{B}\) a basis on \(X\)
    \[
    \mathcal{T}_{\mathcal{B}} = \left\{U \subseteq X: \forall x \in U, \exists B \in
    \mathcal{B} \text{such that} x \in B \subseteq U \right\}
    \]
\end{definition}


\begin{corollary}
{Basis is a subset of the Topology it Generates}{basis_is_a_subset_of_the_topology_it_generates}
Let \(\mathcal{B} \) be a basis for a set \(X \), then
\[
\mathcal{B} \subseteq \mathcal{T} _{\mathcal{B}}
\]
\end{corollary}
\begin{proof}
   Let \(B \in \mathcal{B} \) then we note that \(B \subseteq X \) since
   \(\mathcal{B} \subseteq \mathcal{P} \left(X\right) \). Additionally
   for every \(x \in B \) \(B \) itself is an element from \(\mathcal{B}
   \) such that \(x \in B \subseteq B \), so \(B \in \mathcal{T} _{
   \mathcal{B}} \)
\end{proof}


\begin{lemma}
{Intersection of Two Elements from The Topology Generated by a
    Basis (Basis Existance) is
Closed}{intersection_of_two_elements_from_the_topology_generated_by_a_basis_(basis_existance)_is_closed}
    We will prove that for any \(X, Y \in \mathcal{T}_{B}\), we have that
    \[
    X \cap Y \in \mathcal{T}_{\mathcal{B}}
    \]
\end{lemma}
\begin{proof}
    \begin{itemize}
        \item We know two things
            \begin{gather*}
                \forall x \in X, \exists B_{X} \in \mathcal{B} \text{such that
} x \in B_{X} \subseteq X\\
                \forall x \in Y, \exists B_{Y} \in \mathcal{B} \text{such that
} x \in B_{Y} \subseteq Y\\
            \end{gather*}
        \item And we would like to show that
            \[
                \forall x \in X \cap Y, \exists B_{XY} \in \mathcal{B} \text{
                such that} x \in B_{XY} \subseteq X \cap Y\\
            \]
        \item Let \(x \in X \cap Y\), then \(x \in X\) and \(x \in Y\), thus by the
            two facts we have \(B_{X}\) and \(B _{Y}\) from \(\mathcal{B}\).
        \begin{itemize}
            \item By clause two of the definition of basis, we can see that
                \[
                \forall x \in B_{X} \cap B_{Y}, \exists B \in \mathcal{B} \text{
                such that} x \in B \subseteq B_{X} \cap B_{Y} \subseteq X \cap
                Y
                \]
            \item So we set \(B_{XY} = B\) to finish the proof.
        \end{itemize}
    \end{itemize}
\end{proof}


\begin{proposition}
{Topology Generated By a Basis (Basis Existance) is a
Topology}{topology_generated_by_a_basis_(basis_existance)_is_a_topology}
    \begin{center}
        We claim that \(\mathcal{T}_{B}\) is a topology
    \end{center}
\end{proposition}
\begin{proof}
    \begin{itemize}
        \item Vacuously we see that \(\varnothing \in \mathcal{T} _{B}\), and
        \(X \in \mathcal{T} _{B}\) by the first clause of the definition of a
        basis
        \item We will show that \(\bigcap_{i = 1}^{n} U _{i} \in
        \mathcal{T}_{B}\) where \(\forall j \in \left[ n \right], U_{j} \in
        \mathcal{T} _{B}\) by induction
        \begin{itemize}
            \item Base Case
                \[
                \bigcap_{i = 1}^{1} U_{i} = U_{1} \in \mathcal{T}_{B}
                \left(\text{by assumption}\right)
                \]
            \item Inductive Step
            \begin{itemize}
                \item Suppose \(k \in \mathbb{N} ^{+}\) and assume it's true for
                \(k\) we'll show it holds on \(k + 1\)
                    \[
                    \bigcap_{i = 1}^{k + 1} U_{i} = \bigcap_{i = 1}^{k} U_{i}
                    \cap U_{k + 1}
                    \]
                \item Now by induction hypothesis \(\bigcap_{i = 1}^{k} U_{i}
                \in \mathcal{T} _{B}\) and also we know that \(U_{k + 1} \in
                \mathcal{T}_{B}\), therefore by the fact that this topology is
                closed under union for two elements, we know that
                    \[
                    \bigcap_{i = 1}^{k} U_{i} \cap U_{k + 1} \in \mathcal{T}_{B}
                    \]
                    as needed
            \end{itemize}
        \end{itemize}
        \item We will show that for some index set \(I\) we have
            \[
                \bigcup_{\alpha \in I} U_{\alpha} \in \mathcal{T}_{B}
            \]
        \begin{itemize}
            \item That is
                \[
                    \forall x \in \bigcup_{\alpha \in I} U_{\alpha}, \exists B
                    \in \mathcal{B} \text{such that} x \in B \subseteq
                    \bigcup_{\alpha \in I} U_{\alpha}
                \]
            \item So let \(x \in \bigcup_{\alpha \in I} U_{\alpha}\), therefore
            we know that \(x \in U_{\beta}\) for some \(\beta \in I\), but since
            \(U _{\beta} \in \mathcal{T}_{B}\) we get \(B_{U} \in \mathcal{B}\)
            such that \(x \in B_{U} \subseteq U_{\beta}\)
            \item Take \(B = B_{U}\) and note that
                \[
                x \in B = B_{U} \subseteq U_{\beta} \subseteq \bigcup_{\alpha
                \in I} U _{\alpha}
                \]
                as needed.
        \end{itemize}
    \end{itemize}
\end{proof}


\begin{definition}
{Topology Generated by a Basis
    (Union)}{topology_generated_by_a_basis_(union)}
    Let \(X\) be a set and \(\mathcal{B}\) a basis on \(X\), we define:
    \[
    \mathcal{T}_{\mathcal{B}} = \left\{\bigcup \mathcal{C}: \mathcal{C}
    \subseteq \mathcal{B} \right\}
    \]
    and say that \(\mathcal{T} _{B}\) is called the topology generated by \(
    \mathcal{B}\), note that \(\mathcal{C} \)'s elements are subsets of \(
    X \) (basis elements)
\end{definition}


\begin{proposition}
{Topology Generated By a Basis (Union) is a
Topology}{topology_generated_by_a_basis_(union)_is_a_topology}
    \[
    \mathcal{T}_{\mathcal{B}} \text{is a topology}
    \]
\end{proposition}
\begin{proof}
    \begin{itemize}
        \item \(\varnothing \in \mathcal{T}_{\mathcal{B}} \) is true because
        \(\varnothing \subseteq \mathcal{B} \) and \(\bigcup \varnothing =
        \varnothing \). It's also clear that \(X \in \mathcal{T}_{B} \) since
        from the definition of basis we know that \(X = \bigcup \mathcal{B} \)
        \item We'll show that \(\mathcal{T} _{\mathcal{B}} \) is closed under
        arbitrary unions
        \begin{itemize}
            \item Let \(V_{\alpha}, \alpha \in I \) be elements of \(\mathcal{T}
            _{\mathcal{B}} \) for some indexing set \(I \), one must show that
                \[
                \bigcup _{\alpha \in I} V_{\alpha} \in \mathcal{T}
                _{\mathcal{B}}
                \]
            \item By the definition of \(\mathcal{T} _{\mathcal{B}} \) for each
            \(\alpha \) we have \(\mathcal{C}_ {\alpha} \) such that
            \(V_{\alpha} = \bigcup \mathcal{C} _{\alpha} \), therefore
            \begin{align*}
                \bigcup _{\alpha \in I} V_{\alpha} &= \bigcup _{\alpha \in I}
                \left(\bigcup \mathcal{C} _{\alpha}\right) \\
                &=  \bigcup _{\alpha \in I} \left(\bigcup _{X \in \mathcal{C}
                _{\alpha}} X\right) \\
                &=  \bigcup _{X \in \left(\bigcup _{\alpha \in I} \mathcal{C}
                _{\alpha}\right)} X \\
                &= \bigcup \left(\bigcup _{\alpha \in I} \mathcal{C} _{\alpha}\right)
            \end{align*}
        \end{itemize}
        \item \(\mathcal{T} _{\mathcal{B}} \) is closed under finite
        intersections, we'll show it's closed under pairwise intersections and
        the general result may be proven inductively.
        \begin{itemize}
            \item Suppose \(U = \bigcup \mathcal{A} \) and \(V = \bigcup
            \mathcal{C} \) are elements of \(\mathcal{T} _{\mathcal{B}}
            \) we must show that \(U \cap V \in \mathcal{T} _{\mathcal{B}
            } \)
                \[
                U \cap V = \left(\bigcup \mathcal{A}\right) \cap \left(
                \bigcup \mathcal{C}\right) = \bigcup \left\{A \cap C:
                \enspace \text{where} \enspace A \in \mathcal{A}, C \in
                \mathcal{C} \right\}
                \]
                where the last equality comes from the fact that if a point is
                in the intersection of \(\bigcup \mathcal{A} \) and \(\bigcup
                \mathcal{C} \) then it must be in the intersection of some \(A
                \in \mathcal{A} \) and some \(C \in \mathcal{C} \)
            \item To show this is an element of \(\mathcal{T} _{\mathcal{B}
            } \) all we have to do is to show that the set \(\left\{A \cap C:
            \text{where} A \in \mathcal{A}, C \in \mathcal{C}
            \right\} \subseteq \mathcal{T} _{\mathcal{B}} \) and then use
            the fact that \(\mathcal{T} _{\mathcal{B}} \) is closed under
            arbitrary unions.
            \begin{itemize}
                \item Let \(J \cap K \) be from the set under consideration and
                now our goal is to show that \(J \cap K \in \mathcal{T} _{
                \mathcal{B}} \), that is we have some \(\mathcal{R} \) such
                that \(J \cap K = \bigcup \mathcal{R} \)
                \item Recall that \(J \in \mathcal{A} \subseteq \mathcal{B}
                \subseteq P\left(X\right) \) therefore \(J \subseteq X \) and
                by the same logic \(K \subseteq X \) therefore \(J \cap K
                \subseteq X \).
                \item Thus for every \(x \in J \cap K \) we have \(B_{x} \)
                such that \(x \in B_{x} \subseteq J \cap K \) from the
                definition of basis for a set, so we have
                    \[
                    J \cap K \subseteq \left[ \bigcup _{x \in J \cap K} B_{x
                    } \right] \subseteq J \cap K
                    \]
                    Since each \(B_{x} \subseteq J \cap K \)
                \item In other words
                    \[
                    J \cap K = \left[ \bigcup _{x \in J \cap K} B_{x}
                    \right] = \bigcup \left\{B_{x} : x \in J \cap K \right\}
                    \]
                \item And finally noting that \(\left\{B_{x} : x \in J \cap K
                \right\} \subseteq \mathcal{B} \)
            \end{itemize}
        \end{itemize}
    \end{itemize}
\end{proof}


\begin{proposition}
{Generating by Basis Existance or Union Yields Same
Topology}{generating_by_basis_existance_or_union_yields_same_topology}
\[
\left\{U \subseteq X: \forall x \in U, \exists B \in \mathcal{B}
~\text{such that} x \in B \subseteq U \right\} = \left\{\bigcup \mathcal{C} :
\mathcal{C} \subseteq \mathcal{B} \right\}
\]
\end{proposition}
\begin{proof}
    \begin{itemize}
        \item \(\subseteq \) Let \(U \) be an element of the left hand side,
        we'd like to show it's an element of the right hand side. For each point
        \( x \in  U \) we have \( B _{ x } \in \mathcal{ B }   \) such that \( x
        \in B _{ x } \subseteq U\), therefore \( U = \bigcup _{ x \in  U } B _{
        x}  \) and so \( U \) is an element of the right hand side.
        \item \(\supseteq \) Suppose that \(\mathcal{C} \subseteq \mathcal{B} \)
        then for any \(C \in \mathcal{C} \) we know that \(C \in \mathcal{
        B} \) (it's a basis element) and therefore we know that \(C \) is an
        element of the left hand side since we can take \(B = C \). Then since
        the left hand side is a topology \(\bigcup \mathcal{C} \) is also a
        part of the left hand side as it's an arbitrary union.
    \end{itemize}
\end{proof}


\begin{lemma}
{Basis Criterion}{basis_criterion}
Let \(X\) be a topological space. Suppose that \(\mathcal{C}\) is a collection
of open sets of \(X\) such that for each open set \(U\) of \(X\) and each \(x\)
in \(U\), there is an element \(C\) of \(\mathcal{C}\) such that \(x \in C
\subset U\). Then \(\mathcal{C}\) is a basis for the topology of \(X\).
\end{lemma}
\begin{proof}
    \begin{itemize}
        \item We'll show it's a basis, so let \(x \in X \) now since \(X \)
        is open we know that there is \(C \in \mathcal{C} \) such that
        \(x \in C \subseteq X \), as needed.
        \item We continue to the second condition, so we take \(C _{1}, C _{
        2} \in \mathcal{C} \) and then let \(x \in C _{1} \cap C _{2}
        \) (we are allowed to do that, because if \(C _{1} \cap C _{2} =
        \varnothing \) then the statement holds vacuously). Since \(\mathcal{C
} \) is a collection of open sets we know that \(C _{1} \cap C _{
        2} \) is also open with respect to \(X \), thus by assumption we have \(C
        _{3} \) such that \(x \in C _{3} \subseteq C _{1} \cap C _{2} \)
        \item The statement also claims that \(\mathcal{T} _{\mathcal{C}
} = \mathcal{T} \)where \(\mathcal{T} \) is the topology of \(X \)
        \begin{itemize}
            \item \(\subseteq \) Suppose \(W \) is an element of the topology
            generated by \(\mathcal{C} \), then \(W = \bigcup C \) where \(C
            \subseteq \mathcal{C} \), but note that every element of \(C \)
            is an element of \(\mathcal{T} \) since \(\mathcal{C} \)
            is a collection of open subsets of \(X \), therefore \(W \) being
            an arbitrary union is also open with respect to \(X \), that is \(
            W \in \mathcal{T} \).
            \item \(\supseteq \) Suppose \(U \in \mathcal{T} \) then if
            \(x \in U \) we have \(C \in \mathcal{C} \) such that \(x
            \in C \subseteq U\), then by the basis existance definition of a
            generated topology we can see that \(U \in \mathcal{T} _{\mathcal{
            C}} \)
        \end{itemize}
    \end{itemize}
\end{proof}



\begin{proposition}
{Finer is Equivalent to Basis Containment}{finer_is_equivalent_to_basis_containment}
Let \(\mathcal{B}_{1}\) and \(\mathcal{B}_{2}\) be two bases on a set \(X\),
then \(\mathcal{T}_{\mathcal{B}_{1}} \subseteq \mathcal{T}_{\mathcal{B}_{2}}\)
if and only if for every \(x \in X\) and \(B_{1} \in \mathcal{B}_{1}\)
containing \(x\), there is a \(B_{2} \in \mathcal{B}_{2}\) such that \(x \in
B_{2} \subseteq B_{1}\)
\end{proposition}
\begin{proof}
\begin{itemize}
    \item \( \Rightarrow  \) 
    \begin{itemize}
        \item Suppose \( \mathcal{ T } _{ \mathcal{ B } _{ 1 }    } \subseteq
        \mathcal{ T } _{ \mathcal{ B } _{ 2 }   }    \). Now let \( x \in  X \)
        and \( B _{ 1 } \in  \mathcal{ B } _{ 1 }   \) where \( x \in B _{ 1 }  \)
        (Note that we can do this because any basis covers \( X \)),
        \hyperref[corollary:basis_is_a_subset_of_the_topology_it_generates]{additionally
        we have that every basis is a subset of the topology it generates}
        therefore \( B _{ 1 } \in \mathcal{ T } _{ \mathcal{ B } _{ 1 } }     \) and
        so by assumption we have that \( B _{ 1 } \in \mathcal{ T } _{ \mathcal{
        B} _{ 2 }   }   \) which by definition means that \( \forall x \in B _{
        1} , \exists B _{ 2 } \in  \mathcal{ B } _{ 2 }, \) such that \( x \in B
        _{ 2 } \subseteq B _{ 1 } \) which is exactly what we wanted to show.
    \end{itemize}
    \item \( \Leftarrow  \) 
    \begin{itemize}
        \item Suppose the reverse, so let \( U \in \mathcal{ T } _{ \mathcal{ B
        } _{ 1 }   }   \) we must show that \( U \in  \mathcal{ T } _{ \mathcal{
        B } _{ 2 } }  \). Since \( U \in \mathcal{ T } _{ \mathcal{ B } _{ 1 }
        }    \)  this means that \( \forall x \in  U, \exists B _{ 1 } \in
        \mathcal{ B } _{ 1 }    \) such that \( x \in  B _{ 1 } \subseteq U \)
        is a true statement. Recall that we'd like to prove that \( U \in
        \mathcal{ T } _{ \mathcal{ B } _{ 2 }   }   \), namely that \( \forall x
        \in U\) we have \( B _{ 2 } \in \mathcal{ B } _{ 2 }   \) such that \( x
        \in  B _{ 2 }  \subseteq U\) 
        \item Therefore let \( x \in  U \) by the fact that \( U \in  \mathcal{
        T} _{ \mathcal{ B } _{ 1 }   }   \) we have \( B _{ 1  } \in \mathcal{ B
        } _{ 1 }   \) such that \( x \in \mathcal{ B } _{ 1 } \subseteq U  \). 
        \item By our original assumption (which is \( \forall x \in  X \) and \(
        B _{ 1 }  \in \mathcal{ B } _{ 1 }  \) containing \( x \) we have  \( B
        _{ 2 } \in  \mathcal{ B } _{ 2 }  \) with \( x \in B _{ 2 } \subseteq B
        _{ 1 } \)), we get \( B _{ 2 } \in  \mathcal{ B } _{ 2 }  \) such that
        \( x \in  B _{ 2 } \subseteq B _{ 1 } \) and recall that \( B _{ 1 }
        \subseteq U \) so we have \( x \in B _{ 2 } \subseteq U \) as needed,
        thus \( U \in  \mathcal{ T } _{ \mathcal{ B } _{ 2 }   }   \) 
    \end{itemize}
\end{itemize} 
\end{proof}


\begin{definition}
{Standard Topology on the Real Line}{standard_topology_on_the_real_line}
    If \(\mathcal{B} \) is the collection of all open intervals on the real
    line, that is:
    \[
    \left(a, b\right) = \left\{x : a < x < b \right\}
    \]
    then the topology generated by \(\mathcal{B} \) is called the standard
    topology on the real line
\end{definition}


\begin{definition}
{Lower Limit Topology}{lower_limit_topology}
     The topology generated by \(\mathcal{B} = \left\{\left[ a, b
\right) \subseteq \mathbb{R}: a < b \right\} \) is defined as the lower
     limit topology on \(\mathbb{R} \)
\end{definition}
The corresponding topological space \(\left(\mathbb{R}, \mathcal{S}\right)
\) is called the Sorgenfrey line.


\begin{definition}{K-Topology on R}{k-topology_on_r}
    Define $ \mathcal{ K }  = \left\{ \frac{1}{n}: n \in  \mathbb{N} ^{ + }
    \right\}  $, and $ \mathcal{ B } =  \left\{ \left( a, b \right): a < b
    \right\} \cup \left\{ \left( a, b \right) \setminus \mathcal{ K } : a < b
    \right\}$ then the topology generated by $ \mathcal{ B }  $ is called the
    \( \mathcal{ K }   \)-Topology
\end{definition}


\begin{proposition}
{Lower Limit and K-Topology are Strictly Finer than the
Standard Topology on
R}{lower_limit_and_k-topology_are_strictly_finer_than_the_standard_topology_on_r}
Let \(\mathcal{T} _{\ell} \) be the lower limit topology, \(\mathcal{T
} _{\mathcal{K}} \) be the \(\mathcal{K} \)-topology and \(
\mathcal{T} \) the standard topology on \(\mathbb{R} \), then
\[
\mathcal{T} \subset \mathcal{T} _{\ell} \enspace \text{and} \enspace
 \mathcal{T} \subset \mathcal{T} _{\mathcal{K}}
\]
\end{proposition}
\begin{proof}
    We show this by using
    \hyperref[proposition:finer_is_equivalent_to_basis_containment]{this}. So
    let \(x \in X \) and \(\left(a, b\right) \) be a basis element for
    \(\mathcal{T} \) then \(\left[ x, b\right) \) is a basis element of
    \(\mathcal{T} _{\ell} \) with \(\left[ x, b\right) \subseteq \left(
    a, b\right) \). Now see that \(\left[ x, d\right) \) is an element of \(
    \mathcal{T} _{\ell} \) (it is a basis element) for which there is no
    basis element of \(\mathcal{T} \) that contains \(x \) and is a subset
    of \(\left[ x, d\right) \) because they are open intervals, therefore \(
    \mathcal{T} \subset \mathcal{T} _{\ell} \).\\
    As for the \(\mathcal{K} \) topology we can see that for any basis
    element \(\left(a, b\right) \) of \(\mathcal{T} \) we can use that
    same basis element in \(\mathcal{T} _{\mathcal{K}} \) and thus we
    trivially get that \(\mathcal{T} \subseteq \mathcal{T} _{\mathcal{K}} \), we
    also note that for the basis element \( \left( -1, 1 \right) \setminus
    \mathcal{ K }   \) of \( \mathcal{ T } _{ \mathcal{ K }   }   \) and \( x =
    0\) there is no basis element of \( \mathcal{ T }   \) that contains \( 0 \)
    and is a subset of \( \left( -1, 1  \right) \setminus \mathcal{ K }   \) 
\end{proof}


\begin{definition}
{Subbasis}{subbasis}
A subbasis \(\mathcal{S}\) for a topology on \(X\) is a collection of subsets of
\(X\) whose union equals \(X \). 
\end{definition}


\begin{definition}
{Topology Generated by a
Subbasis}{topology_generated_by_a_subbasis}
 The topology generated by the subbasis \(\mathcal{S}\) is defined to be the
 collection \(\mathcal{T}\) of all unions of finite intersections of elements of
 \(\mathcal{S}\). In another light we may say that this topology is generated by
 the basis which is contstructed of finite intersections of elements of \(
 \mathcal{S} \)
\end{definition}


\begin{proposition}
{Topology Generated by a Subbasis is a
Topology}{topology_generated_by_a_subbasis_is_a_topology}
Let \(\mathcal{S} \) be a subbasis, the topology generated by \(\mathcal{
S} \) is a topology
\end{proposition}
\begin{proof}
To show that this is true we will use first show that that the set of finite
intersections of elements of \(\mathcal{S} \) (let's call this \(\mathcal{
B} _{\mathcal{S}} \enspace \Winkey \)) is a basis, then we know that \(\left\{
\bigcup B : B \subseteq \mathcal{B} _{\mathcal{S}} \right\} \) is a
topology by~\ref{proposition:topology_generated_by_a_basis_(union)_is_a_topology}\\
Let \(x \in X \) then since \(\bigcup \mathcal{S} = X \) we know that \(x
\in S\) for some \(S \in \mathcal{S} \), note that we say that a single
set is a finite intersection, and therefore \(S \in \mathcal{B} _{\mathcal{
S}} \) so we've found a basis element of \(\mathcal{B} _{\mathcal{S}
} \) which contains \(x \). Now for the second condition let \(\mathcal{I},
\mathcal{J} \) be finite index sets, then set \(B _{1} = \bigcap _{ \alpha \in
\mathcal{I}} S _{\alpha} \) and \(B _{2} = \bigcap _{ \alpha \in \mathcal{J}} S
_{\alpha} \), but then their intersection is \( \bigcap _{ \alpha \in \mathcal{
I} \cup \mathcal{ J }    } S _{ \alpha  }  \) and is therefore still a finite
union of elements of \( \mathcal{ S }   \), and so this set satisfies the second
condition for being a basis.
\end{proof}


\begin{definition}
{Order Topology}{order_topology}
Let \(X\) be a set with a simple order relation; assume \(X\) has more than one
element. Let \(\mathcal{B}\) be the collection of all sets of the following
types:
\begin{enumerate}
    \item All open intervals \((a, b)\) in \(X\).
    \item All intervals of the form \(\left[a_{0}, b\right)\), where \(a_{0}\)
    is the smallest element (if any) of \(X\).
    \item All intervals of the form \(\left(a, b_{0}\right]\), where \(b_{0}\)
    is the largest element (if any) of \(X\). The collection \(\mathcal{B}\) is
    a basis for a topology on \(X\), which is called the order topology.
\end{enumerate}
If \(X\) has no smallest element, there are no sets of type (2), and if \(X\)
has no largest element, there are no sets of type (3).
\end{definition}


\begin{proposition}
{Order Topology Basis}{order_topology_basis}
The set \(\mathcal{B} \) specified by~\ref{definition:order_topology} is a
basis.
\end{proposition}
\begin{proof}
Let \(x \in X \) if \(x \) is the smallest or largest element of \(X \) then
sets \(\left[ x, x + \varepsilon\right) \) or \(\left(x - \varepsilon,
x \right] \) respectively will work. Otherwise \(\left(x - \varepsilon, x
+ \varepsilon\right) \) will work.\\
Now we take two elements from \(\mathcal{B} \) and find a third contained
within both.
\begin{itemize}
    \item \(\left(a, b\right) \cap \left(c, d\right) = \left(x, y\right)
    \) for some \(x, y \)
    \item \(\left(a, M \right] \cap \left(b, M \right] = \left(\max\left(a, b\right),M \right] \)
    \item \(\left[m, a\right) \cap \left[m, b\right) = \left[m, \min\left(
    a, b\right)\right) \)
    \item \(\left(a, b\right) \cap \left(d, M \right] = \left(\max\left(a, d\right), b\right) \) and \(\left(a, b\right) \cap \left[ m, d
\right) = \left(a, \min\left(b, d\right)\right) \)
\end{itemize}
Therefore the second condition of a basis is satisfied so \(\mathcal{B} \)
is a basis.
\end{proof}


\begin{definition}
{Product Topology of two Topological Spaces}{product_topology_of_two_topological_spaces}
Let \(X\) and \(Y\) be topological spaces. The product topology on \(X \times
Y\) is the topology having as basis the collection \(\mathcal{B}\) of all sets
of the form \(U \times V\), where \(U\) is an open subset of \(X\) and \(V\) is
an open subset of \(Y\).
\end{definition}


\begin{proposition}{Open Cartesian Products form a
Basis}{open_cartesian_products_form_a_basis}
The set \( \mathcal{ B } \) specified by
\ref{definition:product_topology_of_two_topological_spaces} is a basis.
\end{proposition}
\begin{proof}
    Let \( x \in  X \) then the set \( X \times Y \) is a basis element since \(
    X\) and \( Y \) are open with respect to themselves. \\
    Let \( U _{ 1 } \times V _{ 1 }  \) and \( U _{ 2 } \times V _{ 2 }  \) be
    two elements of \( \mathcal{ B }   \) then  by
    \ref{proposition:commutivity_of_cartesian_product_and_(union,intersection)}
    we can see
    \[
        \left( U _{ 1 } \times V _{ 1 }  \right) \cap \left( U _{ 2 } \times V
        _{ 2 }  \right) = \left( U _{ 1 } \cap U _{ 2 }  \right) \times \left( V
        _{ 1 } \cap  V _{ 2 } \right) 
    \]
    and thus for the second condition of a basis we can trivially choose \( B =
    B\).
\end{proof}


\begin{proposition}
{Basis For the Product Topology of Two Topological
Spaces}{basis_for_the_product_topology_of_two_topological_spaces}
Let \(X, Y \) be topological spaces generated by the basis \(\mathcal{B},
\mathcal{C} \) then
\[
\mathcal{D} = \left\{B \times C: B \in \mathcal{B} \enspace \text{and}
\enspace C \in \mathcal{C} \right\}
\]
is a basis for the topology \(X \times Y \).
\end{proposition}
\begin{proof}
   We will prove it using the \hyperref[lemma:basis_criterion]{basis
   criterion}. Let \(W \) be an open set of \(X \times Y \), and let \(
   \left(x, y\right) \in W \) since \(\left\{U \times V: U \enspace
   \text{open in} \enspace X \enspace \text{and} \enspace \text{open
   in} \enspace Y \right\} \) is a basis for the topology \(X \times Y \)
   (and a basis covers the set \(X \times Y \)) we have some \(U \times V \)
   such that \(\left(x, y\right) \in U \times V \subseteq W \).
   Additionally, since \(\mathcal{B}, \mathcal{C} \) are bases for \(X, Y \) we
   get \(B \) and \(C \) such that \(x \in B \subseteq U \) (since \(U \) is
   open in \(X \) and
   using~\ref{definition:topology_generated_by_a_basis_(basis_existance)}) and
   \(y \in C \subseteq V \) so therefore \(\left(x, y\right) \in B \times C
   \subseteq W \) thus we've found an element of the basis in question which is
   a subset of of the open set, meaning we've satisfied the basis criterion and
   thus \(\mathcal{D} \) is a basis.
\end{proof}


\begin{proposition}
{Subbasis for the Product Topology of two Topological
Spaces}{subbasis_for_the_product_topology_of_two_topological_spaces}
The set
\[
\mathcal{S} = \left\{\pi _{1} ^{-1} \left(U\right) : U \enspace
\text{open in} \enspace X \right\} \cup \left\{\pi _{2} ^{-1} \left(V
\right) : V \enspace \text{open in} \enspace Y \right\}
\]
is a subbasis for the product topology on \(X \times Y \).
\end{proposition}
\begin{proof}
    We will show \(\mathcal{T} _ \mathcal{S} \subseteq \mathcal{T} _{X
    \times Y} \) directly and to show that \(\mathcal{T} _{X \times Y}
    \subseteq \mathcal{T} _ \mathcal{S} \) we use~\ref{proposition:finer_is_equivalent_to_basis_containment}. Recall that
    from~\ref{proposition:inverse_image_of_the_projection_mapping},
    \begin{gather*}
    \left\{\pi _{1} ^{-1} \left(U\right) : U \enspace
    \text{open in} \enspace X \right\} \cup \left\{\pi _{2} ^{-1} \left(V
\right) : V \enspace \text{open in} \enspace Y \right\}\\
    \verteq\\
    \left\{U \times Y: U \text{open in} \enspace X \right\} \cup \left\{X
    \times V: V \enspace \text{open in} \enspace Y \right\}
    \end{gather*}
    So any element of \(\mathcal{S} \) is an element of the topology on \(
    X \times Y\) because every element of \(\mathcal{S} \) is of the form \(A
    \times B \) where \(A \) is open in \(X \) and \(B \) is open in \(Y \)
    so that they (any element of \(\mathcal{S} \)) are basis elements of the
    basis which generates \(X \times Y \), therefore arbitrary unions of finite
    intersections of elements of \(\mathcal{S} \) are still part of the
    topology (by the definition of topology), thus we have directly shown that
    \(\mathcal{T} _{\mathcal{S}} \subseteq \mathcal{T} _{X \times Y
} \). \\
    Now let \(U \times V \) be a basis element of \(\mathcal{T} _{X \times
    Y} \) but note that
    \begin{align*}
        U \times V &= \left(U \times Y\right) \cap \left(X \times V\right)
        \\
                   &= \pi _{1} ^{-1} \left(U\right) \cap \pi _{2} ^{-1}
                   \left(V\right)
    \end{align*}
    that is to say that \(U \times V \) is equal to a basis element of \(
    \mathcal{T} _{\mathcal{S}} \) and therefore
    from~\ref{proposition:finer_is_equivalent_to_basis_containment} it follows
    trivally that \(\mathcal{T} _{X \times Y} \subseteq \mathcal{T} _{
    \mathcal{S}} \), hence \( \mathcal{ T } _{ \mathcal{ S }   } = \mathcal{ T }
    _{ X \times Y } \) 
\end{proof}


\subsection{The Subspace Topology}

\begin{definition}
{Subspace Topology}{subspace_topology}
    Let \(X\) be a topological space with topology \(\mathcal{T}\). If \(Y\) is a subset of \(X\), the collection
    \[
    \mathcal{T}_{Y} = \{Y \cap U \mid U \in \mathcal{T}\}
    \]
    is a topology on \(Y\), called the subspace topology. With this topology,
    \(Y\) is called a subspace of \(X\); its open sets consist of all
    intersections of open sets of \(X\) with \(Y\). We will say that \( U \) is
    open in \( Y \) if it belongs to the topology of \( Y \) and open in \( X \)
    if it belongs to the topology of \( X \).
\end{definition}


\begin{proposition}{The Subspace Topology is a
Topology}{the_subspace_topology_is_a_topology}
\( \mathcal{ T } _{ Y }   \) is a topology
\end{proposition}
\begin{proof}
    Note that \( \varnothing = Y \cap \varnothing \in  \mathcal{ T } _{ Y }
    \) and that \( X = Y \cap  X \in  \mathcal{ T } _{ Y } \). Now consider an
    arbitrary union: \( \bigcup _{ \alpha \in \mathcal{ J }   } \left( U _{
    \alpha } \cap Y \right)  \) then by the fact that
    \hyperref[proposition:intersection_factors]{intersection factors} we can see
    that it equals \( \left( \bigcup _{ \alpha \in \mathcal{ J }   } U _{ \alpha
    } \right) \cap Y \) which is an element of the subspace topology. Again
    using the fact that the intersection factors, we can see that for the finite
    intersection we have \( \bigcap _{ i \in  \left[ n \right]  } \left( U _{ i
    } \cap  Y \right) = \left( \bigcap _{ i \in  \left[ n \right]  } U _{ i }
    \right) \cap  Y\) again an element of the subspace topology, thus it is
    indeed a topology.
\end{proof}


\begin{proposition}{Basis for the Subspace
Topology}{basis_for_the_subspace_topology}
If \( \mathcal{ B }   \) is a basis for the topology of \( X \), then 
\[
\mathcal{ B } _{ Y } = \left\{ B \cap Y: B \in  \mathcal{ B }   \right\}  
\]
is a basis for the subspace topology on \( Y \).
\end{proposition}
\begin{proof}
    We will proceed by attempting to satisfy the
    \hyperref[lemma:basis_criterion]{basis criterion}, so let \( U \) be an open
    set of \( X \) and consider \( y \in U \cap  Y \) since \( y \in U \) and \(
    \mathcal{ B }  \) generates the topology on \( X \), then by the basis
    existance definition, we get \( B \in \mathcal{ B }   \) such that \( y \in
    B \subseteq U\) since \( y \in Y \) we can say the following \( y \in B \cap
    Y \subseteq U \cap Y\) therefore we've found an element of \( \mathcal{ B }
    _{ Y } \)  contained within our open set \( U \cap Y \) as needed, thus \(
    \mathcal{ B } _{ Y }   \) is a basis for \( \mathcal{ T } _{ Y }   \) .
\end{proof}


\begin{proposition}{Subspace Topology
Transitivity}{subspace_topology_transitivity}
Let \( Y \) be a subspace of \( X \). If \( U \) is open in \( Y \) and \( Y \)
is open in \( X \) , then \( U \) is open in \( X \) 
\end{proposition}
\begin{proof}
   If \( U  \) is open in \( Y \) then \( U =  Y \cap  V \)  where \( V \) is
   open in \( X \) since \( Y \)  is also open in \( X \) then their
   intersection is open in \( X \), in other words \( U  \) is open in \( X \) 
\end{proof}


\begin{proposition}{Subspace Product is same as Normal
Product}{subspace_product_is_same_as_normal_product}
If \( A \) is a subspace of \( X \) and \( B \) is a subspace of \( Y \), then
the product topology on \( A \times B \) is the same as the subspace topology \(
A \times B\) inherits as a subspace of \( X \times Y \).
\end{proposition}
\begin{proof}
\begin{itemize}
    \item As a subspace
    \begin{itemize}
        \item The set of elements \( U \times V \) where \( U \) is open in \( X
        \) and \( V \) is open in \( Y \)  is a basis for \( X \times Y
        \)
        \item Therefore \( \left\{ \left( U \times  V \right) \cap \left( A \times B
        \right): U \in \mathcal{ T } _{ X } ~\&~ V \in \mathcal{ T } _{ Y }
        \right\}  \) is a basis for the subspace \( A \times B \) with
        respect to \( X \times Y \) by
        \ref{proposition:basis_for_the_subspace_topology}
    \end{itemize}
    \item As a product
    \begin{itemize}
        \item \( A \times B \) is generated by the basis \( \left\{ J \times
        K: J \enspace \text{open in} \enspace A, K \enspace \text{open in}
        \enspace B \right\}  \) 
        \item Since \( A \) is a subspace of \( X \) and \( B \) is a subspace
        of \( Y \) we know that
        \[
        \mathcal{ T } _{ A } = \left\{ U \cap A: U \enspace \text{open in}
        \enspace X \right\}  \enspace \text{and} \enspace  
        \mathcal{ T } _{ B } = \left\{ V \cap B: V \enspace \text{open in}
        \enspace B \right\} 
        \]
    \end{itemize}

    \item With all that context in mind, observe the following :
    \begin{gather*}
        \left\{ J \times K: J \enspace \text{open in} \enspace A, K \enspace
        \text{open in} \enspace B \right\} \\
        \verteq\\
        \left\{ \left( U \cap A \right) \times \left( V \cap  B \right):
        U \in \mathcal{ T } _{ X } \enspace \& \enspace V \in \mathcal{ T }
        _{ Y } \right\} \\
        \verteq \enspace \text{by
        \ref{proposition:commutivity_of_cartesian_product_and_(union,intersection)}}
        \enspace  \\
        \left\{ \left(  U \times V \right) \cap \left( A \times B \right) :
        U \in \mathcal{ T } _{ X } \enspace \& \enspace V \in  \mathcal{ T }
        _{ Y } \right\} 
    \end{gather*}
    \item Therefore the two basis are identical, and thus the topology they
    generate are identical, as needed.
\end{itemize}
\end{proof}


\subsection{Closed Sets and Limit Points}

\begin{definition}{Closed Set}{closed_set}
A subset \( A \) of a topological space \( X \) is closed if and only if \( X
\setminus A \) is open.
\end{definition}


\begin{proposition}{Properties of Closure}{properties_of_closure}
Suppose \( X \) is a topological space, then the following holds:
\begin{enumerate}
    \item \( \varnothing , X \) are closed
    \item Arbitrary intersections of closed sets are closed
    \item Finite unions of closed sets are closed
\end{enumerate}
\end{proposition}
\begin{proof}
    \begin{enumerate}
        \item \( X \setminus \varnothing = X \) and \( X \setminus X =
        \varnothing  \) and since \( X \) is a topology we know that \(
        \varnothing , X \in  \mathcal{ T }   _{ X }  \) 
        \item Suppose \( \left\{ A _{ \alpha  }  \right\} _{ \alpha \in
        \mathcal{ J }   }  \) are closed sets, then from
        \hyperref[theorem:demorgan's_laws]{DeMorgan's Laws} we know that 
        \[
        X \setminus \bigcap _{ \alpha \in \mathcal{ J }   } = \bigcup _{ \alpha
        \in \mathcal{ J }  } \left( X -  A _{ \alpha  }  \right) 
        \]
        which is an aribtrary union of open sets and is therefore open.
        \item Suppose that \( A _{ i }  \) is closed for \( i \in  \left[ n
        \right] \) then again from DeMorgan we have
        \[
        X \setminus  \bigcup _{ i = 1 } ^{ n } A _{ i } = \bigcap _{ i = 1 }^{ n
        } \left( X \setminus A _{ i }  \right) 
        \]
        thus it's open as it's equal to a finite intersection of open sets.
    \end{enumerate}
\end{proof}


\begin{proposition}{Closed in a Subspace}{closed_in_a_subspace}
Suppose \( Y \) is a subspace of \( X \), then \( A \) is closed in \( Y \) if
and only if it equals the intersection of a closed set of \( X \) with \( Y \) 
\end{proposition}
\begin{proof}
    \( \Leftarrow  \), assume that \( A =  C \cap  Y \) with \( C \) closed in
    \( X \), then \( X \setminus C \) is open in \( X \). Thus \( \left( X - C
    \right) \cap Y \in \mathcal{ T } _{ Y }   \) by
    \ref{definition:subspace_topology} by
    \ref{proposition:associativity_of_intersection_and_set_difference} we get
    that \( \left( X \setminus C \right) \cap  Y = Y \setminus A \) so \( Y
    \setminus  \) is open in \( Y \) which means \( A \) is closed in \( Y \)
    .\\
    \( \Rightarrow  \)  Assuming that \( A \) is closed in \( Y \) so then \( Y
    \setminus A\) is open in \( Y \) by the definition of the subspace topology
    that means that \( Y \setminus A = U \cap Y \) where \( U  \) is open in \(
    X\), setting \( A =  Y \cap  \left( X \setminus U \right)  \) satisfies the
    given equation by \ref{proposition:intersection_as_set_difference} and thus
    \( A \) is equal to the intersection a closed set of \( X \) with \( Y \)
    since \( U  \) was open in \( X \) making \( X \setminus U \) closed.
\end{proof}


\begin{definition}{Interior of a Set}{interior_of_a_set}
Given a subset \( A \) of a topological space \( X \), the interior of \( A \)
is the union of all open sets contained in \( A \), and is written as \(
\operatorname{ int } \left( A \right)  \).Note as a union of open sets, it's
open itself.
\end{definition}


\begin{proposition}{Interior is Itself}{interior_is_itself}
Suppose that \( A \) is open if and only if  \( \operatorname{ int } \left( A \right) = A  \) 
\end{proposition}
\begin{proof}
If \( A \) is open then \( A \) is an open set set containing \( A \) therefore
as \( \operatorname{int} \left( A \right)  \) is the union of all open sets
containing \( A \), we can see that \( \operatorname{int} \left( A \right) = A
\) since it is a union of sets \( X \subseteq A \) with \( A \) itself. \\
Suppose \( \operatorname{int} \left( A \right) = A \) since \(
\operatorname{int} \left( A \right)  \) is an arbitrary union of open sets then
it is still open, thus \( A \) is open.
\end{proof}


\begin{definition}{Closure of a Set}{closure_of_a_set}
Given a subset \( A \) of a topological space, the closure of \( A \) is the
intersection of all closed sets containing \( A \), and is denoted by \(
\overline{A}  \). Note that from \ref{proposition:properties_of_closure} that \(
\overline{A} \) is closed.
\end{definition}


\begin{proposition}{Closure is Itself}{closure_is_itself}
Suppose that \( A \) is closed if and only if \( A = \overline{A}  \) 
\end{proposition}
\begin{proof}
    Recall that \( \overline{A}  \) is defined as the intersection of all closed
    sets which contain \( A \), thus the smallest set which could contain \( A
    \) is \( A \) itself, since \( A \) was assumed to be closed, the
    intersection is equal to \( A \).\\ 
    Supposing that \( \overline{A} = A \) then since \( \overline{A}  \) is an
    arbitrary intersection of closed sets, then by
    \ref{proposition:properties_of_closure} \( A \) is closed as well.
\end{proof}


\begin{proposition}{Subset Relationship between Interior and
Closure}{subset_relationship_between_interior_and_closure}
Suppose that \( A \) is a subset of a topological space \( X \), then 
\[
\operatorname{ int } \left( A \right)  \subseteq A \subseteq \overline{A} 
\]
\end{proposition}
\begin{proof}
    By definition \( \operatorname{ int } \left( A \right)  \) is the union of
    sets which are subsets of \( A \) thus it is a subset of \( A \), on the
    otherhand \( \overline{A}  \) is an intersection of sets which all contain
    \( A \) thus \( A \subseteq \overline{A}  \), so we've deduced that \(
    \operatorname{ int } \left( A \right) \subseteq A \subseteq \overline{A}
    \) as needed.
\end{proof}


\begin{proposition}{Closure in a Subspace}{closure_in_a_subspace}
Let \( Y \) be a subspace of \( X \) and let \( A \) be a subset of \( Y \)
then with \( \overline{A}  \) being the closure of \( A \) in \( X \), then the
closure of \( A \)  in \( Y \) is \( \overline{A}  \cap Y \) 
\end{proposition}
\begin{proof}
    Let \( B \) be the closure of \( A \) in \( Y \). Since \( \overline{A}  \)
    is closed in \( X \) then \( \overline{A} \cap Y \) is closed in \( Y \) as
    it an intersection of a closed set of \( X \) with \( Y \) (see
    \ref{proposition:closed_in_a_subspace}). By the definition of closure \( B
    \) is the intersection of all closed subsets of \( Y \) which contain \( A
    \) and thus \( B \subseteq \left( \overline{A} \cap Y \right)  \). Also
    since \( B \) is closed in \( Y \) then it equals \( C \cap  Y \) for some
    set \( C \) which is closed in \( X \) (by
    \ref{proposition:closed_in_a_subspace} again), recall that \( \overline{A}
    \) is closed in \( X \) and thus by it's definition we get that \(
    \overline{A} \subseteq C \) which yields \( \overline{A} \cap Y \subseteq C
    \cap Y = B\) so we've shown that \( \overline{A} \cap Y \subseteq B
    \subseteq \overline{A} \cap Y\) and so \( B = \overline{A} \cap Y \) .
\end{proof}


\begin{proposition}{Element of Closure
Characterization}{element_of_closure_characterization}
Let \( A \) be a subset of a topological space \( X \), then
\( x \in  \overline{A}  \) if and only if every open set \( U \) containing \( x
\) intersects \( A \) 
\end{proposition}
\begin{proof}
    To prove \( A \Leftrightarrow B \) one may equivalently prove \( \neg A
    \Leftrightarrow \neg B \), we perform the latter.\\
    Suppose \( x \not\in \overline{A}  \) let us show that there is some open
    set \( U \) containing \( x \) which doesn't intersect \( A \). By
    considering the set \( U = X \setminus \overline{A}  \) then \( x \in  U \)
    and \( A \cap  U = \varnothing  \) (since \( A \subseteq \overline{A}  \))
    so they do not interesect.\\
    Conversely, if we have an open set \( U \) containing \( x \) which doesn't
    intersect with \( A\), then \( X \setminus U \) is a closed set of \( X \)
    which contains \( A \), since \( \overline{A}  \) is the intersection of all
    closed sets which contain \( A \) then we can see that \( \overline{A}
    \subseteq X \setminus U \) now since \( x \in U \) then \( x \not\in X
    \setminus U \) and since \( \overline{A} \subseteq X \setminus U \) then
    also \( x \not\in \overline{A}  \) which is what we needed to prove.
\end{proof}



\begin{proposition}{Element of Closure Basis
Characterization}{element_of_closure_basis_characterization}
Let \( A \) be a subset of a topological space \( X \), then
supposing that the topology of \( X \) is given by a basis, then \( x \in 
\overline{A}  \) if and only if every basis element \( B \) containing \( x \)
intersects \( A \) 
\end{proposition}
\begin{proof}
Since every basis element is part of the topology it generates
(\ref{corollary:basis_is_a_subset_of_the_topology_it_generates}) then this means
that every \( B \in  \mathcal{ B }   \) are open sets. Now suppose that \( x \in
\overline{A} \) then by \ref{proposition:element_of_closure_characterization}
every open set \( U \) containing \( x \) intersects \( A \), then consider
every \( B \in  \mathcal{ B }   \) where \( x \in  B \), since \( B \) is open
then we know that \( B \) and \( A \) intersect, so every basis element \( B \)
containing \( x \) intersects \( A \). \\
For the other direction we assume that every basis element that contains \( x \)
intersects \( A \). Now considering \( U \in \mathcal{ T } _{ \mathcal{ B }   }
\) by \hyperref[definition:topology_generated_by_a_basis_(basis_existance)]{the
basis existance definition of a topology generated by a basis}, we can see that
an open set \( U \) that contains \( x \) contains a basis element which
intersects \( A \) and thus \( U \) also intersects \( A \) since \( B \subseteq
U\), thus by \ref{proposition:element_of_closure_characterization} \( x \in
\overline{A}  \), as needed.
\end{proof}


\begin{definition}{Neighborhood of a Point}{neighborhood_of_a_point}
A neighborhood \( U \) of a point \( x \) is an open set \( U \) which contains
\( x \).
\end{definition}


\begin{definition}{Limit Point}{limit_point}
If \( A \) is a subset of a topological space \( X \) we say that \( x \) is a
limit point of \( A \) if and only if every neighborhood of \( x \) intersects
\( A \) in some point other than itself. Note that \( x \) need not be an
element of \( A \).
\end{definition}


\begin{proposition}{Closure as Union of Original Set and Limit
Points}{closure_as_union_of_original_set_and_limit_points}
Let \( A \) be a subset of a topological space \( X \) and let \( A ^{ \prime  }
\) denote the limit points of \( A \) then: 
\[
\overline{A} = A \cup  A ^{ \prime  } 
\]
\end{proposition}
\begin{proof}
    If \( x \in  A ^{ \prime  } \) then by definition every neighborhood \( U \)
    of \( x \) has that \( U  \cap \left( A \setminus \left\{ x \right\}
    \right) \neq \varnothing \) thus by
    \ref{proposition:element_of_closure_characterization} we can see that \( x
    \in  \overline{A}  \), that is \( A ^{ \prime  } \subseteq \overline{A}  \),
    now one should not forget that \( A \subseteq \overline{A}  \)
    (\ref{proposition:subset_relationship_between_interior_and_closure}) thus
    having both in tandem yields \( A ^{ \prime  }  \cup  A \subseteq
    \overline{A} \).\\
    For the reverse inclusion, let \( x \in  \overline{A}  \) and we'll prove
    that \( x \in  A ^{ \prime  } \cup  A \), but if \( x \in A \) then surely
    it's true, thus we may assume that \( x \not\in A \). Due to the fact that
    \( x \in \overline{A}  \) we know that every neighborhood \( U \) of \( x \)
    intersects \( A \), but \( x \not\in A \) so the intersection point is not
    \( x \), in other words we have that \( U \cap  \left( A \setminus \left\{ x
    \right\}  \right) \neq \varnothing  \) and so \( x \in A ^{ \prime  }  \)
    which tells us \( x \in A \cup A ^{ \prime  }  \).
\end{proof}


\begin{proposition}{Closed if it Contains it's Limit Points}{closed_if_it_contains_it's_limit_points}
A subset \( A \) of of a topological space \( X \) is closed if and only if it
contains all of it's limit points, that is 
\[
A ^{ \prime  } \subseteq A
\]
\end{proposition}
\begin{proof}
    \( A \) is closed if and only if \( A = \overline{A}  \), by
    \ref{proposition:closure_as_union_of_original_set_and_limit_points} we see
    that \( A =  A \cup  A ^{ \prime  } \), this holds if and only if \( A ^{
    \prime  } \subseteq A \) because if \( A ^{ \prime  }  \) contains an
    element which is not an element of \( A \) we get a contradiction since \( A
    =  A ^{ \prime  } \cup  A\)  therefore every element of \( A ^{ \prime  }
    \) is an element of \( A \), namely \( A ^{ \prime  } \subseteq A \) 
\end{proof}


\begin{definition}{Convergence}{convergence}
Given a topological space \( X \) we say that a sequence \( x _{ 1 } , x _{  2 }
, \ldots  \) converges to a point \( x \in  X \) if for each neighborhood \( U
\) of \( x \) there is an \( n \in \mathbb{N} ^{ + }  \) such that for all \( n
\ge N\) we have \( x _{ n } \in  U \) 
\end{definition}


\begin{definition}{T1 Space}{t1_space}
A topological space \( X \) is said to be \( T _{ 1 }  \) if for any \( x, y
\din X \), there are neighborhoods \( U, V \) of \( x, y \) respectively such
that \( y \not\in U \) and \( x \not\in V \) 
\end{definition}


\begin{definition}{Hausdorff Space}{hausdorff_space}
A topological space \( X \) is called a Hausdorff space if for each pair \( x, y
\din X\) there exists disjoint neighborhoods \( U, V \) of \( x, y \). A space
with this property is said to be \( T _{ 2 }  \) .
\end{definition}


\begin{proposition}{Every finite point set in a T1 space is
Closed}{every_finite_point_set_in_a_t1_space_is_closed}
Let \( S \) be a finite subset of a topological space \( X \) given then \( T _{
1}  \) property, then \( S \) is closed.
\end{proposition}
\begin{proof}
Since finite unions of closed sets are closed
(\ref{proposition:properties_of_closure}), then we can see that we may
equivalently prove that every one point set is closed.\\
Let \( x \in  X \), we will show \( \left\{ x \right\}  \) is closed in \( X \),
so we will show that \( X \setminus \left\{ x \right\}  \) is open via
\ref{proposition:open_iff_every_point_is_in_another_open_set}. Let \( a \in  X
\setminus \left\{ x \right\}  \) then since \( a \neq x \) by the \( T _{ 1 }
\) axiom, we get a neighborhood \( U _{ a }  \) of \( a \) such that \( x
\not\in U _{ a }  \), thus by
\ref{proposition:open_iff_every_point_is_in_another_open_set}, \(  X \setminus
\left\{ x \right\}  \) is open and therefore \( \left\{ x \right\}  \) is
closed.
\end{proof}



\begin{proposition}{T1 Space, Limit Point iff Every Neighborhood contains
Infinitely Many
Points}{t1_space,_limit_point_iff_every_neighborhood_contains_infinitely_many_points}
Let \( X \) be a space satisfying the \( T _{ 1 }  \) axiom, and \( A \) a
subset of \( X \), then the point \( x \) is a limit point of \( A \) if and
only if every neighborhood of \( x \) contains infinitely many points of \( A \) 
\end{proposition}
\begin{proof}
    \( \Leftarrow  \) If every neighborhood of \( x \) contains infinitely many
    points of \( A \), then it intersects \( A \) in some point other than \( x
    \) so therefore \( x \in A ^{ \prime  } \) by definition.\\
    \( \Rightarrow  \) Supposing that \( x \in A ^{ \prime  }  \) and for the
    sake of contradiction that some neighborhood \( U \) of \( x \) intersects
    \( A \) in only finitely many points, then also \( U \setminus \left\{ x
    \right\}  \) intersects \( A \) in only finitely many points, so let \(
    \left( U \setminus \left\{ x \right\} \cap A \right) = \left\{ x_{1} , x_{2}
    , \dotsc , x_{n} \right\}  \), but then since  \( \left\{ x_{1} , x_{2}
    , \dotsc , x_{n} \right\} \) is a finite point set in a \( T _{ 1 }  \)
    space, we know that it is closed so that \( X \setminus \left\{ x_{1} ,
    x_{2} , \dotsc , x_{n} \right\}  \) is open (and note that \( x \in  X
    \setminus \left\{ x_{1} , x_{2} , \dotsc , x_{n} \right\}  \) since \( x
    \not\in U \cap \left( A \setminus \left\{ x \right\}  \right) = \left\{
    x_{1} , x_{2} , \dotsc , x_{n} \right\} \)). Thus \( U \cap X \setminus
    \left\{ x_{1} , x_{2} , \dotsc , x_{n} \right\}  \) is an open set
    containing \( x \) and therefore a neighborhood of \( x \), but note that
    this set doesn't intersect \( A \setminus \left\{ x \right\}  \) because if
    \( k \in U \cap \left( X \setminus \left\{ x_{1} , x_{2} , \dotsc , x_{n}
    \right\} \right)    \) then \( k \not\in \left\{ x_{1} , x_{2} , \dotsc ,
    x_{n} \right\}  \) so \( k \not\in A \setminus \left\{ x \right\}  \), this
    is a contradiction with the fact that \( x \) was a limit point, because
    when that's true every neighborhood of \( x \) intersects \( A \setminus
    \left\{ x \right\}  \). Thus we conclude that every neighborhood of \( x \)
    contains infinitely many points of \( A \).
\end{proof}



\begin{proposition}{Hausdorff Yields Unique
Convergence}{hausdorff_yields_unique_convergence}
If \( X \) is a Hausdorff space then a sequence of points \( X \) converges to
at most one point of \( X \) 
\end{proposition}
\begin{proof}
    Suppose \( x _{ n }  \) is a sequence of points of \( X \) which converges
    to \( x \). Let \( y \in  X \) such that \( y \neq x \), then we have two
    disjoint neighborhoods \( U _{ x } , V _{ y }  \) since \( U _{ x }  \)
    contains all the points of \( x _{ n }  \) except for finitely many, then \(
    V _{ y } \) can only contain finitely many points of \( x _{ n }  \) so that
    \( x _{ n }  \) cannot converge to \( y \) 
\end{proof}


\subsection{Continuous Functions}

\begin{definition}{Continuous Function}{continuous_function}
Let \( X, Y\) be topological spaces, then a function \( f : X \to Y  \) is said
to be continuous if for each open subset \( V \) of \( Y \) the set \( f ^{-1}
\left( V \right)  \) is an open subset of \( X \) 
\end{definition}


\begin{proposition}{Continuous Function in terms of Basis
Elements}{continuous_function_in_terms_of_basis_elements}
Suppose \( X, Y \) are topological spaces where \( Y \) is generated by the
basis \( \mathcal{ B }   \) then \( f \) is continuous if the inverse image of
every basis element is open in \( X \).
\end{proposition}
\begin{proof}
    From \ref{definition:topology_generated_by_a_basis_(union)} we know that
    given \( V \in  \mathcal{ T } _{ Y }    \) that \( V =  \bigcup _{ \alpha
    \in \mathcal{ J }   } B _{ \alpha  }  \) thus by
    \ref{proposition:inverse_image_respects_set_operations} we have
    \[
    f ^{-1} \left( V \right) = \bigcup _{ \alpha \in \mathcal{ J }   } f ^{-1}
    \left( B _{ \alpha  }  \right) 
    \]
    So then \( f ^{-1} \left( V \right)  \) is open with respect to \( X \) so
    long as \( f ^{-1} \left( B _{ \alpha  }  \right)  \) is for each \( \alpha
    \in  \mathcal{ J }  \) 
\end{proof}


\begin{proposition}{Continuity Equivalences}{continuity_equivalences}
Let \( X, Y \) be topological spaces; let \( f : X \to Y \) then the following
are equivalent:
\begin{enumerate}
    \item \( f \) is continuous
    \item For every \( A \subseteq X \) we have that \( f\left( \overline{A}
    \right) \subseteq \overline{f\left( A \right) }  \) 
    \item For every closed set \( B \subseteq Y  \), \( f ^{-1} \left( B \right)
    \) is a closed in \( X \) 
    \item For each \( x \in  X \) and neighborhood \( V \) of \( f\left( x
    \right)  \), there exists a neighborhood \( U \) of \( x \) such that \(
    f\left( U \right) \subseteq V \) 
\end{enumerate}
\end{proposition}
\begin{proof}
    To show that they are all equivalent we will prove that \( 1 \Rightarrow 2
    \) , \( 2 \Rightarrow 3 \) , \( 3 \Rightarrow 1 \) and \( 1 \Rightarrow 4
    \), \( 4 \Rightarrow 1 \), this will show all possible bi-implications as
    all bi-implications using \( 1, 2, 3 \) are proving using the cycle \( 1
    \Rightarrow 2 \Rightarrow 3 \Rightarrow 1 \) and for any bi-implication
    which uses \( 4 \), say \( a \Rightarrow 4 \) then it may be obtained by
    getting starting from \( a \) getting to \( 1 \) (always possible as \( a \)
    is in the cycle which contains \( 1 \)) which leads to \( 4 \), to show \( 4
    \Rightarrow a\) we see it's possible by entering the cycle from \( 4 \)
    through \( 1 \) .\\
    \( 1 \Rightarrow 2 \) Suppose that \( f \) is continuous and let \( A
    \subseteq X \) we must show that if \( k \in  f\left( \overline{A}  \right)
    \) then we have that \( k \in  \overline{f\left( A \right) }  \), to do that
    we will use
    \ref{proposition:element_of_closure_characterization}, supposing that \( k
    \in  f\left( \overline{A}  \right)  \) then we know that \( k = f\left( x
    \right)  \) where \( x \in  \overline{A}  \) we want to show that \( k \in
    \overline{f\left( A \right) } \) so as mentioned let \( U \) be an open set
    which contains \( k \), we must show that \( U \) intersects \( f\left( A
    \right)  \) since \( U \) is open in \( Y \) we know that \( f ^{-1} \left(
    U\right)  \) is open in \( X \) since we assumed that \( f \) is continuous.
    Recall that \( k \in  U \) and that \( k =  f\left( x \right)  \) so then \(
     x \in f ^{-1} \left( U \right) \) so \( f ^{-1} \left( U \right)  \) is an
     open set of \( X \) which contains \( x \in \overline{A}  \) thus since
     \ref{proposition:element_of_closure_characterization} is an if and only if
     we know that \( f ^{-1} \left( U \right)  \) must intersect \( \overline{A}
     \) at some point \( y \) so that \( U  \) intersects \( f\left( A \right)
     \) at the point \( f\left( y \right)  \) since \( y \in  A \cap f ^{-1}
     \left( U \right)  \) , therefore by
     \ref{proposition:element_of_closure_characterization} we know that \( k \in
     \overline{f\left( A \right) } \) as needed.\\
     \( 2 \Rightarrow 3 \) Suppose \( 2  \) holds and let \( B \subseteq Y \) be
     a closed set, we will show that \( f ^{-1} \left( B \right)  \) is closed
     in \( X \), for convience set \( A =  f ^{-1}  \left( B \right)  \) then
     we will show that \( A = \overline{A}  \) to satisfy
     \ref{proposition:closure_is_itself}. We are already aware that \( A
     \subseteq \overline{A}  \) by
     \ref{proposition:subset_relationship_between_interior_and_closure} so
     therefore we just have to show that \( \overline{A} \subseteq  A \) to do
     this we will show that we will take \( x \in  \overline{A}  \) and therefore we know  
     \( f\left( x \right) \in  f\left( \overline{A}  \right)   \) from \( 2 \)
     yielding \( f\left( x \right) \in  f\left( \overline{A}  \right) \subseteq
     \overline{f\left( A \right) } \subseteq B\)  where the final equality
     follows from \ref{proposition:image_of_the_inverse_image} and that fact
     that \( B \) is closed, specifically \( \overline{f\left( A \right) } =
     \overline{f\left( f ^{-1} \left( B \right)  \right) } \subseteq
     \overline{B} = B  \), so it holds \( f\left( x \right) \in  B  \)
     equivalently \( x \in  f ^{-1} \left( B \right) = A \) so \( \overline{A}
     \subseteq A \) yielding \( \overline{A} = A \). \\
     \( 3 \Rightarrow 1 \) Let \( V \) be an open set of \( Y \), and let \( B
     = Y \setminus  V\) therefore by
     \ref{proposition:inverse_image_respects_set_operations} we can see that 
     \[
     f ^{-1} \left( B \right) =  f ^{-1} \left( Y \right) \setminus f ^{-1}
     \left( V \right) = X \setminus f ^{-1} \left( V \right) 
     \]
     where the second equality is being justified by
     \ref{proposition:inverse_image_of_codomain}. Since \( f ^{-1} \left( B
     \right)  \) is a closed set of \( X \) due to this one can see that \( X
     \setminus f ^{-1} \left( V \right)  \) is also closed meaning that \(
     X \setminus \left( X \setminus f ^{-1} \left( V \right)  \right)   \) is an
     open set of \( X \) and we note that \( X \setminus \left( X \setminus f
     ^{-1} \left( V \right)  \right) =  f ^{-1} \left( V \right) \), namely that
     \( f ^{-1} \left( V \right)  \) is an open set of \( X \) as needed.\\
     \( 1 \Rightarrow 4 \) Let \( x \in  X \) and suppose \( v \) is a
     neighborhood of \( f\left( x \right)  \) then \( U = f ^{-1} \left( V
     \right)  \) is an open set containing \( x \) and thus a neighborhood of \(
     x\) and by \ref{proposition:image_of_the_inverse_image} one can see that \(
     f\left( U \right) =  f\left( f ^{-1} \left( V \right)  \right) \subseteq V
     \) as needed.
     \( 4 \Rightarrow 1 \) Now suppose the converse so to prove that \( f \) is
     continuous let \( V \) be an open set of \( Y \) and let \( x \) be a point
     of \( f ^{-1} \left( V \right)  \), so that \( V \) becomes a neighborhood
     of \( f\left( x \right)  \) 
\end{proof}


\subsection{The Product Topology}


\begin{definition}{Product Topology}{product_topology}
    Let $\mathcal{S}_{\beta}$ denote the collection
    \[
    \mathcal{ S } _{\beta}=\left\{\pi_{\beta}^{-1}\left(U_{\beta}\right) \mid U_{\beta} \text { open in } X_{\beta}\right\}
    \]
    and let $\mathcal{ S } $ denote the union of these collections,
    \[
    \mathcal{S}=\bigcup_{\beta \in J} \mathcal{S}_{\beta}
    \]
    The topology generated by the subbasis $\mathcal{ S } $ is called the product topology. In this topology $\prod_{\alpha \in J} X_{\alpha}$ is called a product space.
\end{definition}


\input{topology/definitions/box_topology}

\begin{theorem}{Basis for the Box Topology}{basis_for_the_box_topology}

Suppose the topology on each space $X_{\alpha}$ is given by a basis $\mathcal{B}_{\alpha}$. The collection of all sets of the form
\[
\prod_{\alpha \in \mathcal{ J } } B_{\alpha}
\]
where $B_{\alpha} \in \mathcal{B}_{\alpha}$ for each $\alpha$, will serve as a basis for the box topology on $\prod_{\alpha \in \mathcal{ J } } X_{\alpha}$.
\end{theorem}

\begin{theorem}{Basis for the Product Topology}{basis_for_the_product_topology}

Suppose the topology on each space $X_{\alpha}$ is given by a basis $\mathcal{B}_{\alpha}$. The collection of all sets of the form
\[
\prod_{\alpha \in \mathcal{ J } } B_{\alpha}
\]

where $B_{\alpha} \in \mathcal{ B } _{\alpha}$ for finitely many indices $\alpha$ and $B_{\alpha}=X_{\alpha}$ for all the remaining indices, will serve as a basis for the product topology $\prod_{\alpha \in \mathcal{ J } } X_{\alpha}$.
\end{theorem}

\begin{definition}{R Omega}{r_omega}
$\mathbb{R}^{\omega}$, the countably infinite product of $\mathbb{R}$ with itself. Recall that
\[
    \mathbb{R}^{\omega}=\prod_{n \in \mathbb{N}} X_{n}
\]
with $ X_{ n }  =  \mathbb{R}  $ for each $ n $ 
\end{definition}


\subsection{The Metric Topology}

\begin{definition}{A metric}{metric}
A metric on a set $X$ is a function
\[
    d: X \times X \to \mathbb{R} 
\]
having the following properties:
\begin{enumerate}
       \item $d(x, y) \geq 0$ for all $x, y \in X$; equality holds if and only if $x=y$.
       \item $d(x, y)=d(y, x)$ for all $x, y \in X$.
       \item Triangle Inequality: $d(x, y)+d(y, z) \geq d(x, z)$, for all $x, y, z \in X$.
\end{enumerate}
\end{definition}

\begin{example}{Discrete Metric}{discrete_metric}
 $d: X \times X \rightarrow \mathbb{R}$ given by

\[
d(x, y)= \begin{cases}0 & x=y \\ 1 & \text { otherwise }\end{cases}
\]
\end{example}


\begin{definition}{Epsilon Ball}{epsilon_ball}
Given $\epsilon>0$, consider the set
\[
B_{d}(x, \epsilon)=\{y \mid d(x, y)<\epsilon\}
\]
of all points $y$ whose distance from $x$ is less than $\epsilon$. It is called the $\epsilon$-ball centered at $\boldsymbol{x}$. Sometimes we omit the metric $d$ from the notation and write this ball simply as $B(x, \epsilon)$, when no confusion will arise.
\end{definition}


\input{topology/lemmas/epsilon_ball_contains_another}

\input{topology/propositions/epsilon_balls_form_a_basis}

\input{topology/definitions/metric_topology}

\begin{definition}{Bounded Subset of a Metric Space}{bounded}
Let $X$ be a metric space with metric $d$. A subset $A$ of $X$ is said to be bounded if there is some number $M \in  \mathbb{R}$ such that
\[
d\left(a_{1}, a_{2}\right) \leq M
\]
for every pair of points $ a_{ 1 } , a_{ 2 } \in  A $ 
\end{definition}


\input{topology/examples/function_on_R_omega}



\section{Connectedness and Compactness}


\input{topology/definitions/connected}

Notice that $ U, V $ are actually clopen, as $ X \setminus  U =  V $ and $ X \setminus  V = U $ stating that $ V $ and $ U $ are closed as well.

\begin{example}{Closed and Bounded, not Compact}{closed_and_bounded_not_compact}
A metric space $X$  and a closed and bounded subspace $Y$ of  $X$  that is not compact.
\end{example}


\begin{itemize}
    \item Consider the set $ X =  \left\{ \frac{1}{n}: n \in  \mathbb{N} ^{ +  }  \right\}  $, with the \hyperref[example:discrete_metric]{discrete metric}, it is bounded because the for any two points $ a, b \in X, d\left( a, b \right)  \le 1 $  %todo{closed and bounded proof}
    \item Let $ X $ be an infinite set and let consider the discrete metric on that set,  the metric topology which it induces (call it $ \mathcal{ T }  $)  is the discrete topology of $ X $. Therefore if we consider any subset $ Y $ of $ X $ it is closed, as $ X \setminus Y \in  \mathcal{ T }  $ (remember it's the discrete topology). But the open covering $ \left\{ \left\{ x \right\} : x \in  X \right\}  $ has no finite subcollection which also covers $ X $.
\end{itemize}

\input{topology/examples/R_omega_connected_in_diff_topos}

\input{topology/propositions/connected_implies_closure_connected}

\input{topology/definitions/totally_disconnected}

Consider $ \mathbb{R} _{ \ell } $ if we have $ \left\{ a \right\}  $ and $ \left\{ b \right\}  $ then the open sets $ \left( - \infty , b \right) $ and $ \left[ b, \infty  \right) $ is a separation of $ \mathbb{R} _{ \ell }  $, therefore it's disconnected. Similarly for any two points $ a, b $  in $ \mathbb{Q}  $ we have some irrational number $ r $ between the two, and thus $ \left( - \infty , r \right) _{ \mathbb{Q}  } , \left( r, \infty  \right) _{ \mathbb{Q}  }  $ is a separation thus $ \mathbb{Q}  $ is totally disconnected under this topology. 

Let's look at $ \mathbb{R}  $ with the fite complement topology. Right off the bat, we note that if a set is finite in $ \mathbb{R}  $ it's complement must be infinite therefore if $ \mathbb{R}  $ was completely disconnected it would mean for any singleton sets, we have a separation $ U, V $, but that means that $ U $ and $ V $ must be infinite, but we then get a contraditiction as $ \mathbb{R} =  U \cup V $ so $ R \setminus U = V $, now since $ U $ was open this implies that $ V $ is finite, which is a contradiction. This idea may be extended to $ \mathbb{R} ^{ 2 }  $.


\subsection{Compact Spaces}

\begin{definition}{Covering}{covering}
A collection $A$ of subsets of a space $X$ is said to cover $X$, or to be a covering of $X$, if the union of the elements of $A$ is equal to $X$. It is called an open covering of $X$ if its elements are open subsets of $X$.
\end{definition}

\begin{definition}{Compact Space}{compact_space}
A space $X$ is said to be compact if every open covering $A$ of $X$ contains a finite subcollection that also covers $X$.
\end{definition}

\begin{lemma}{Covering Yields Finite Covering if and only if Compact}{covering_yields_finite_covering_if_and_only_if_compact}
Let $Y$ be a subspace of $X$. Then $Y$ is compact if and only if every covering of $Y$ by sets open in $X$ contains a finite subcollection covering $Y$.
\end{lemma}

\begin{proposition}
{Closed Graph Yields Continuity for Compact Hausdorff
Domain}{closed_graph_yields_continuity_for_compact_hausdorff_domain}
Let \(f: X \to  Y\) be a map from a topological space \(X\) to a
compact Hausdorff space \(Y\). Show that if the graph of \(f\) is closed in \(X
\times Y\), then \(f\) is continuous.
\end{proposition}
\begin{proof}
    We know that \( \Gamma \left( f \right)  \) is closed in \( X \times Y \),
    additionally since \( X \) and \( C \) are also closed in \( X \) and \( Y
    \) respectively, we can see that \(  X \times  C \) is closed in \( X \times
    Y\), therefore the set \( \Gamma \left( f \right) \cap X \times C \) is 
    closed in \( X \times Y \). We also note that compact subsets of hausdorff
    spaces are closed, and that closed subspaces of compact spaces are compact,
    so that in a hausdorff space intersections of compact spaces are compact.
    Due to this we may deduce that  \( X \times C \) is compact, and that \(
    \Gamma \left( f \right) \cap \left( X \times C \right)   \) is compact. Now
    we consider the projection mapping \( \pi _{ X }  : X \times Y \to X  \)
    since the continuous image of a compact space compact we can see that the
    set \( \pi _{ X } \left( \Gamma \left( f \right) \land  \left( X \times C
    \right)   \right)  \) is a compact set, therefore since \( X \) is
    hausdorff it's also a closed set. Now if the following happened to be true
    the proof would be complete:
    \[
    f ^{-1} \left( C \right) = \pi _{ X } \left( \Gamma \left( C \right) \cap
    \left( X \times C \right)  \right) 
    \]
    it does happen to be true, and we'll prove it. \( \subseteq  \) Let \( x \in
    f ^{-1} \left( C \right) \) so that \( f\left( x \right) \in  C \), we must
    show that \( x \in  \pi _{ X } \left( \Gamma \left( C \right) \cap \left( X
    \times C \right)  \right) \), that is \( x =  \pi _{ X } \left( \left( a, b
    \right)  \right)  \) where \( \left( a, b \right)  \in \Gamma \left( C \right) \cap \left( X
    \times C\right)  \) so we require that \( b = f\left( a \right)  \) and
    that \( f\left( a \right) \in  C \), all in all we require \( x =  \pi _{ X } \left(
    \left( a, f\left( a \right)  \right)  \right) = a \) for some \( a \in  X
    \), this simply works if we let \( a =  x \) as we have noted that \(
    f\left( x \right) \in  C \) . \( \supseteq  \) Suppose now that \( x \in
    \Gamma \left( f \right) \cap \left( X \times C \right)  \) in this case we
    know that \( x = \pi _{ X } \left( a, f\left( a \right) \in C  \right) = a  \) for
    some \( a \in  X \) then \( f\left( x \right) = f\left( a \right) \in C \)
    as needed.
\end{proof}

