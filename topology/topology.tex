
\chapter{Topology}

\section{Topological Spaces and Continuous Functions}

\subsection{Basis for a Topology}

\begin{definition}{Basis}{basis}
    If $X$ is a set, a basis for a topology on $X$ is a collection $\mathcal{B}$ of subsets of $X$ (called basis elements) such that
    \begin{enumerate}
        \item For each $x \in X$, there is at least one basis element $B$ containing $x$.
        \item If $x$ belongs to the intersection of two basis elements $B_{1}$ and $B_{2}$, then there is a basis element $B_{3}$ containing $x$ such that $B_{3} \subset B_{1} \cap B_{2}$.
    \end{enumerate}
    If $\mathcal{B}$ satisfies these two conditions, then we define the topology $\mathcal{T}$ generated by $\mathcal{B}$ as follows: $A$ subset $U$ of $X$ is said to be open in $X$ (that is, to be an element of $\mathcal{T}$) if for, each $x \in U$, there is a basis element $B \in \mathcal{B}$ such that $x \in B$ and $B \subset U$. Note that each basis element is itself an element of $\mathcal{T}$.
\end{definition}


\subsection{The Subspace Topology}

\begin{definition}{Subspace Topology}{subspace_topology}
    Let $X$ be a topological space with topology $\mathcal{T}$. If $Y$ is a subset of $X$, the collection
    \[
    \mathcal{T}_{Y}=\{Y \cap U \mid U \in \mathcal{T}\}
    \]
    is a topology on $Y$, called the subspace topology. With this topology, $Y$ is called a subspace of $X$; its open sets consist of all intersections of open sets of $X$ with $Y$.
\end{definition}


\subsection{The Product Topology}

\begin{definition}{Product Topology}{product_topology}
    Let $\mathcal{S}_{\beta}$ denote the collection
    $$
    \mathcal{ S } _{\beta}=\left\{\pi_{\beta}^{-1}\left(U_{\beta}\right) \mid U_{\beta} \text { open in } X_{\beta}\right\}
    $$
    and let $\mathcal{ S } $ denote the union of these collections,
    $$
    \mathcal{S}=\bigcup_{\beta \in J} \mathcal{S}_{\beta}
    $$
    The topology generated by the subbasis $\mathcal{ S } $ is called the product topology. In this topology $\prod_{\alpha \in J} X_{\alpha}$ is called a product space.
\end{definition}


\begin{theorem}{Basis for the Box Topology}{basis_for_the_box_topology}

Suppose the topology on each space $X_{\alpha}$ is given by a basis $\mathcal{B}_{\alpha}$. The collection of all sets of the form
$$
\prod_{\alpha \in \mathcal{ J } } B_{\alpha}
$$
where $B_{\alpha} \in \mathcal{B}_{\alpha}$ for each $\alpha$, will serve as a basis for the box topology on $\prod_{\alpha \in \mathcal{ J } } X_{\alpha}$.
\end{theorem}

\begin{theorem}{Basis for the Product Topology}{basis_for_the_product_topology}

Suppose the topology on each space $X_{\alpha}$ is given by a basis $\mathcal{B}_{\alpha}$. The collection of all sets of the form
\[
\prod_{\alpha \in \mathcal{ J } } B_{\alpha}
\]

where $B_{\alpha} \in \mathcal{ B } _{\alpha}$ for finitely many indices $\alpha$ and $B_{\alpha}=X_{\alpha}$ for all the remaining indices, will serve as a basis for the product topology $\prod_{\alpha \in \mathcal{ J } } X_{\alpha}$.
\end{theorem}

\begin{definition}{R Omega}{r_omega}
$\mathbb{R}^{\omega}$, the countably infinite product of $\mathbb{R}$ with itself. Recall that
$$
\mathbb{R}^{\omega}=\prod_{n \in \mathbb{N}} X_{n}
$$
with $ X_{ n }  =  \mathbb{R}  $ for each $ n $ 
\end{definition}


\subsection{The Metric Topology}

\begin{definition}{A metric}{metric}
A metric on a set $X$ is a function
$$
d: X \times X \longrightarrow \mathbb{R} 
$$
having the following properties:
\begin{enumerate}
       \item $d(x, y) \geq 0$ for all $x, y \in X$; equality holds if and only if $x=y$.
       \item $d(x, y)=d(y, x)$ for all $x, y \in X$.
       \item Triangle Inequality: $d(x, y)+d(y, z) \geq d(x, z)$, for all $x, y, z \in X$.
\end{enumerate}
\end{definition}

\begin{example}{Discrete Metric}{discrete_metric}
 $d: X \times X \rightarrow \mathbb{R}$ given by
$$
d(x, y)= \begin{cases}0 & x=y \\ 1 & \text { otherwise }\end{cases}
$$
\end{example}


\begin{definition}{Epsilon Ball}{epsilon_ball}
Given $\epsilon>0$, consider the set
$$
B_{d}(x, \epsilon)=\{y \mid d(x, y)<\epsilon\}
$$
of all points $y$ whose distance from $x$ is less than $\epsilon$. It is called the $\epsilon$-ball centered at $\boldsymbol{x}$. Sometimes we omit the metric $d$ from the notation and write this ball simply as $B(x, \epsilon)$, when no confusion will arise.
\end{definition}

\begin{definition}{Metric Topology}{metric_topology}
If $d$ is a metric on the set $X$, then the collection of all $\epsilon$-balls $B_{d}(x, \epsilon)$, for $x \in X$ and $\epsilon>0$, is a basis for a topology on $X$, called the metric topology induced by $d$.
\end{definition}


\begin{definition}{Bounded Subset of a Metric Space}{bounded}
Let $X$ be a metric space with metric $d$. A subset $A$ of $X$ is said to be bounded if there is some number $M \in  \mathbb{R}$ such that
$$
d\left(a_{1}, a_{2}\right) \leq M
$$
for every pair of points $ a_{ 1 } , a_{ 2 } \in  A $ 
\end{definition}


\section{Connectedness and Compactness}

\begin{example}{Closed and Bounded, not Compact}{}
A metric space $X$  and a closed and bounded subspace $Y$ of  $X$  that is not compact.
\end{example}

\begin{itemize}
    \item Consider the set $ X =  \left\{ \frac{1}{n}: n \in  \mathbb{N} ^{ +  }  \right\}  $, with the \hyperref[example:discrete_metric]{discrete metric}, it is bounded because the for any two points $ a, b \in X, d\left( a, b \right)  \le 1 $  \todo{closed and bounded proof}
    \item Let $ X $ be an infinite set and let consider the discrete metric on that set,  the metric topology which it induces (call it $ \mathcal{ T }  $)  is the discrete topology of $ X $. Therefore if we consider any subset $ Y $ of $ X $ it is closed, as $ X \setminus Y \in  \mathcal{ T }  $ (remember it's the discrete topology). But the open covering $ \left\{ \left\{ x \right\} : x \in  X \right\}  $ has no finite subcollection which also covers $ X $ .
\end{itemize}


\subsection{Compact Spaces}

\begin{definition}{Covering}{covering}
A collection $A$ of subsets of a space $X$ is said to cover $X$, or to be a covering of $X$, if the union of the elements of $A$ is equal to $X$. It is called an open covering of $X$ if its elements are open subsets of $X$.
\end{definition}

\begin{definition}{Compact Space}{compact_space}
A space $X$ is said to be compact if every open covering $A$ of $X$ contains a finite subcollection that also covers $X$.
\end{definition}

\begin{lemma}{Covering Yields Finite Covering if and only if Compact}{covering_yields_finite_covering_if_and_only_if_compact}
Let $Y$ be a subspace of $X$. Then $Y$ is compact if and only if every covering of $Y$ by sets open in $X$ contains a finite subcollection covering $Y$.
\end{lemma}
