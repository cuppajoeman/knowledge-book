\chapter{First Order Logic}

\section{Languages}

\begin{definition}{First-Order Language}{first_order_language}
	A first order language $\mathcal{L}$ is an infinite collection of distinct symbols, no one of which is properly contained in another, sparated into the following cateogories:
	\begin{enumerate}
		\item Parentheses: (,).
		\item Connectives: $\lor, \neg$.
		\item Quantifier: $\forall$.
		\item Variables, one for each positive integer n: $v_{1}, v_{2}, \ldots, v_{n}, \ldots$ The set of variable symbols will be denoted $ \mathcal{ V }   $ .
		\item Equality symbol: $=$.
		\item Constant symbols: Some set of zero or more symbols. This set will be denoted as $ \mathcal{ C }   $ 
		\item Function symbols: For each positive integer $n$, some set of zero or more $n$-ary function symbols as $ \mathcal{ F }  $ .
		\item Relation symbols: For each positive integer $n$, some set of zero or more $n$-ary relation symbols as $ \mathcal{ R }  $ .
	\end{enumerate}
	We may denote a language $ \mathcal{ L }   $ as $ \left( \mathcal{ V } ,\mathcal{ C } , \mathcal{ F } , \mathcal{ R }      \right)  $ as these are the only parts of a the language which may vary.
\end{definition}


\begin{definition}{Language of Number Theory}{language_of_number_theory}
We define
\[
    \mathcal{ L } _{ NT }  \stackrel{\mathtt{D}}{=} \left\{ \mathcal{ C } = \left\{ 0 \right\} , \mathcal{ F } =  \left\{ S, +, \cdot , E \right\} , \mathcal{ R } = \left\{ < \right\}     \right\} 
\]
\end{definition}


\begin{definition}{Term}{term}
If $\mathcal{L}$ is a language, a term of $\mathcal{L}$ is a nonempty finite string $t$ of symbols from $\mathcal{L}$ such that either:
\begin{enumerate}
    \item $t$ is a variable, or
    \item $t$ is a constant symbol, or
    \item $t: \equiv f t_{1} t_{2} \ldots t_{n}$, where $f$ is an $n$-ary function symbol of $\mathcal{L}$ and each of the $t_{i}$ is a term of $\mathcal{L}$.
\end{enumerate}
\end{definition}


\begin{definition}{Terms of a Language}{terms_of_a_language}
We denote the set of all terms of a language as $ \mathfrak{ T } \left( \mathcal{ L }   \right)   $ 
\end{definition}


\begin{definition}{Formula}{formula}
If $\mathcal{L}$ is a first-order language, a formula of $\mathcal{L}$ is a nonempty finite string $\phi$ of symbols from $\mathcal{L}$ such that either:
\begin{enumerate}
    \item $\phi: \equiv=t_{1} t_{2}$, where $t_{1}$ and $t_{2}$ are terms of $\mathcal{L}$, or
    \item $\phi: \equiv R t_{1} t_{2} \ldots t_{n}$, where $R$ is an $n$-ary relation symbol of $\mathcal{L}$ and $t_{1}, t_{2}$, $\ldots, t_{n}$ are all terms of $\mathcal{L}$, or
    \item $\phi: \equiv(\neg \alpha)$, where $\alpha$ is a formula of $\mathcal{L}$, or
    \item $\phi: \equiv(\alpha \vee \beta)$, where $\alpha$ and $\beta$ are formulas of $\mathcal{L}$, or
    \item $\phi: \equiv(\forall v)(\alpha)$, where $v$ is a variable and $\alpha$ is a formula of $\mathcal{L}$.
\end{enumerate}
\end{definition}


\begin{definition}{Formulas of a Language}{formulas_of_a_language}
We denote the set of all formula of a langauge $ \mathcal{ L }   $ as $ \mathfrak{ F }  \left( \mathcal{ L }   \right)  $ 
\end{definition}


\begin{definition}{Subformula}{subformula}
A formula $ \phi  $ is a subformula of a formula $ \psi  $ if $ \phi  $ appears as a substring of $ \psi  $ 
\end{definition}


\begin{definition}{Scope}{scope}
If $(\forall v)(\alpha)$ is a subformula of $ \phi  $ , we will say that the scope of the quantifier $\forall$ is $\alpha$.  If there are multiple $ \forall  $ in $ \phi  $ then each occurrence of the quantifier will have its own scope. 
\end{definition}


\input{logic/definitions/a_term's_variable_replaced_by_a_term}

\begin{definition}{A Formulas Variable Replaced by a Term}{a_formulas_variable_replaced_by_a_term}
Suppose that $\phi$ is an $\mathcal{L}$-formula, $t$ is a term, and $x$ is a variable. We define the formula $\phi_{t}^{x}$ (read " $\phi$ with $x$ replaced by $t$ ") as follows:
\begin{enumerate}
    \item If $\phi: \equiv=u_{1} u_{2}$, then $\phi_{t}^{x}$ is $=\left(u_{1}\right)_{t}^{x}\left(u_{2}\right)_{t}^{x}$.
    \item If $\phi: \equiv R u_{1} u_{2} \ldots u_{n}$, then $\phi_{t}^{x}$ is $R\left(u_{1}\right)_{t}^{x}\left(u_{2}\right)_{t}^{x} \ldots\left(u_{n}\right)_{t}^{x}$.
    \item If $\phi: \equiv \neg(\alpha)$, then $\phi_{t}^{x}$ is $\neg\left(\alpha_{t}^{x}\right)$.
    \item If $\phi: \equiv(\alpha \vee \beta)$, then $\phi_{t}^{x}$ is $\left(\alpha_{t}^{x} \vee \beta_{t}^{x}\right)$.
    \item If $\phi: \equiv(\forall y)(\alpha)$, then
\end{enumerate}
$$
\phi_{t}^{x}= \begin{cases}\phi & \text { if } x \text { is } y \\ (\forall y)\left(\alpha_{t}^{x}\right) & \text { otherwise }\end{cases}
$$
\end{definition}


\begin{itemize}
    \item  In the fourth clause of the definition above and in the first two clauses of the next definition, the parentheses are not really there. Because $u_{1}{ }_{t}^{x}$ is hard to read so the parentheses have been added in the interest of readability.
    \item If we let $t$ be $g(c)$ and we let $u$ be $f(x, y)+h(z, x, g(x))$, then $u_{t}^{x}$ is
        \[
        f(g(c), y)+h(z, g(c), g(g(c)))
        \]
\end{itemize}


\input{logic/definitions/a_term_is_substitutable_for_a_variable.tex}

To understand the motivation behind the fourth clause, consider the formula:
\[
\phi :\equiv \left( \forall x \right) \left( \forall y \right) x = y
\]

Then one might want to say that $ \left( \left( \forall y \right) x = y \right) _{ t }^{ x }   $  where $ t $ is any term

\input{logic/definitions/free_variable_in_a_formula.tex}

\subsubsection*{Examples}
\begin{itemize}
    \item Thus, if we look at the formula
    $$
    \forall v_{2} \neg\left(\forall v_{3}\right)\left(v_{1}=S\left(v_{2}\right) \vee v_{3}=v_{2}\right)
    $$
    the variable $v_{1}$ is free whereas the variables $v_{2}$ and $v_{3}$ are not free. 
    \item A slightly more complicated example is
        \[
        \left(\forall v_{1} \forall v_{2}\left(v_{1}+v_{2}=0\right)\right) \vee v_{1}=S(0)
        \]
        \begin{itemize}
            \item In this formula, $v_{1}$ is free whereas $v_{2}$ is not free. Especially when a formula is presented informally, you must be careful about the scope of the quantifiers and the placement of parentheses.
        \end{itemize}
\end{itemize}

\subsubsection*{Notes}
\begin{itemize}
    \item We will have occasion to use the informal notation $\forall x \phi(x)$. This will mean that $\phi$ is a formula and $x$ is among the free variables of $\phi$. 
    \item If we then write $\phi(t)$, where $t$ is an $\mathcal{L}$-term, that will denote the formula obtained by taking $\phi$ and replacing each occurrence of the variable $x$ with the term $t$. 
\end{itemize}


\begin{definition}{Variable Assignment Function}{variable_assignment_function}
If $\mathfrak{A}$ is an $\mathcal{L}$-structure, a variable assignment function $ s \in  \operatorname{ func } \left( \mathcal{ V } , A  \right)  $ assigns variables to elements of the universe. 
\end{definition}


\begin{definition}{Term Assignment Function}{term_assignment_function}
Suppose that $\mathfrak{A}$ is an $\mathcal{L}$-structure and $s$ is a variable assignment function into $\mathfrak{A}$. The function $\bar{s}$, called the term assignment function generated by $s$, is the function with domain consisting of the set of $\mathcal{L}$-terms and codomain $A$ defined recursively as follows:
\begin{enumerate}
    \item If $t$ is a variable, $\bar{s}(t)=s(t)$.
    \item If $t$ is a constant symbol $c$, then $\bar{s}(t)=c^{\mathfrak{A}}$.
    \item If $t: \equiv f t_{1} t_{2} \ldots t_{n}$, then $\bar{s}(t)=f^{\mathfrak{A}}\left(\bar{s}\left(t_{1}\right), \bar{s}\left(t_{2}\right), \ldots, \bar{s}\left(t_{n}\right)\right)$.
\end{enumerate}
\end{definition}


\begin{definition}{Interpretation of a Term}{interpretation_of_a_term}
    Let $ t \in  \mathfrak{ T } \left( \mathcal{ L }   \right)    $, $ \mathfrak{ A }   $ an $\mathcal{L}$-Structure, and  $ s \in  \operatorname{ func } \left( \mathcal{ V } , A  \right)  $, then the interpretation of $ t $ is $ \overline{s} \left( t \right)  $ 
\end{definition}


\begin{definition}{Satisfaction}{satisfaction}
Suppose that $\mathfrak{A}$ is an $\mathcal{L}$-structure, $\phi$ is an $\mathcal{L}$-formula, and $s:$ Vars $\rightarrow A$ is an assignment function. We will say that $\mathfrak{A}$ satisfies $\phi$ with assignment $s$, and write $\mathfrak{A} \models \phi[s]$, in the following circumstances:
\begin{enumerate}
    \item If $\phi: \equiv=t_{1} t_{2}$ and $\bar{s}\left(t_{1}\right)$ is the same element of the universe $A$ as $\bar{s}\left(t_{2}\right)$, or
    \item If $\phi: \equiv R t_{1} t_{2} \ldots t_{n}$ and $\left(\bar{s}\left(t_{1}\right), \bar{s}\left(t_{2}\right), \ldots, \bar{s}\left(t_{n}\right)\right) \in R^{\mathfrak{A}}$, or
    \item If $\phi: \equiv(\neg \alpha)$ and it's false that $\mathfrak{A} \models \alpha[s]$ or,
    \item If $\phi: \equiv(\alpha \vee \beta)$ and $\mathfrak{A} \models \alpha[s]$, or $\mathfrak{A} \models \beta[s]$ (or both), or
    \item If $\phi: \equiv(\forall x)(\alpha)$ and, for each element $a$ of $A, \mathfrak{A} \models \alpha[s(x \mid a)]$.
\end{enumerate}

If $\Gamma$ is a set of $\mathcal{L}$-formulas, we say that $\mathfrak{A}$ satisfies $\Gamma$ with assignment $s$, and write $\mathfrak{A} \models \Gamma[s]$ if for each $\gamma \in \Gamma, \mathfrak{A} \models \gamma[s]$.
\end{definition}


\begin{definition}{L-Structure}{l-structure}
Fix a language $\mathcal{L}$. An $\mathcal{L}$-structure $\mathfrak{A}$ is a nonempty set $A$, called the universe of $\mathfrak{A}$, together with:
\begin{enumerate}
    \item For each constant symbol $c$ of $\mathcal{L}$, an element $c^{\mathfrak{A}}$ of $A$,
    \item For each $n$-ary function symbol $f$ of $\mathcal{L}$, a function $f^{\mathfrak{A}}: A^{n} \rightarrow A$, and
    \item For each $n$-ary relation symbol $R$ of $\mathcal{L}$, an $n$-ary relation $R^{\mathfrak{A}}$ on $A$ (i.e., a subset of $A^{n}$ ).
\end{enumerate}
We may denote an $\mathcal{L}$-Structure as follows
\[
\left( A, \mathcal{ C } ^{  \mathfrak{ A }   }  \stackrel{\mathtt{D}}{=} \left\{ c ^{ \mathfrak{ A }   } : c \in \mathcal{ C }   \right\}, \mathcal{ F } ^{ \mathfrak{ A }   } = \left\{ f ^{ \mathfrak{ A }   } : f \in  \mathcal{ F }   \right\}, \mathcal{ R } ^{ \mathfrak{ A }   } = \left\{ R ^{ \mathfrak{ A }   } : R \in  \mathcal{ R }   \right\}        \right) 
\]
\end{definition}


\begin{definition}{Equivalent Formulas}{equivalent_formulas}
We say that two formulas $ \phi  $ and $ \psi  $ are equivalent when for any $\mathcal{L}$-Structure $ \mathfrak{ A }   $ and assignment function $ s $ we have that
\[
\mathfrak{ A } \models \phi  \left[ s  \right] \Leftrightarrow \mathfrak{ A } \models \psi \left[ s \right]  
\]
\end{definition}


\begin{proposition}{Double Negation Equivalence}{double_negation_equivalence}
Let $ \phi  $ be a formula, then $ \phi  $ is equivalent to $ \neg \left( \neg \phi  \right)  $ 
\end{proposition}
\begin{proof}
    Let $ \mathfrak{ A }   $ be $\mathcal{L}$-Structure, and let $ s $ be an assignment function, let's consider what it means for $ \mathfrak{ A } \models \neg \left( \neg \phi  \right)   $ 
    \begin{align*}
        \mathfrak{ A } \models \neg \left( \neg \phi  \right)  &\Leftrightarrow  ~\text{it's false that}~  \mathfrak{ A } \models \neg \phi  \\
                                                               &\Leftrightarrow ~\text{it's false that, it's false that}~ \mathfrak{ A } \models \phi  \\
                                                               &\Leftrightarrow ~\text{it's true that}~  \mathfrak{ A } \models \phi \\
                                                               &\Leftrightarrow  \mathfrak{ A } \models \phi  
    \end{align*}
    And we know that if it's false that it's false, then it's true, because something is either true or false. Thus we've shown that $ \mathfrak{ A } \models \neg \left( \neg \phi  \right)   $ if and only if $ \mathfrak{ A } \models \phi   $ .
\end{proof}


\begin{definition}{A Literal Structure}{a_literal_structure}
Let $ \mathcal{ L } = \left( \mathcal{ V } , \mathcal{ C } , \mathcal{ F } , \mathcal{ R }      \right)    $   be a language,  and $ X \subseteq \mathfrak{ F }   \left( \mathcal{ L }   \right)$, then a literal structure is:
\[
\mathfrak{ L } _{ X }  \stackrel{\mathtt{D}}{=} \left( X , \mathcal{ C } , \mathcal{ F } , \mathcal{ R }     \right)  
\]
That is to say, our interpretation of any constant, function or relation symbol is that symbol itself.
\end{definition}


\begin{definition}{Restricted L-Structure}{restricted_l-structure}
    Suppose $ \mathcal{ L } =  \left( \mathcal{ V } , \mathcal{ C } , \mathcal{ F } , \mathcal{ R }       \right)    $ and $ \mathcal{ L } ^{ \prime  } =  \left( \mathcal{ V } ^{ \prime  }  , \mathcal{ C } ^{ \prime  }  , \mathcal{ F } ^{  \prime  }  , \mathcal{ R } ^{ \prime  }        \right)   $  are languages and $ \mathfrak{A} ^{ \prime  } =\left(A ^{ \prime  } , \mathcal{ C ^{ \prime  }  } ^{ \mathfrak{ A } ^{\prime} }  , \mathcal{ F ^{ \prime  }  } ^{ \mathfrak{ A } ^{\prime}   },\mathcal{ R ^{ \prime  }  } ^{ \mathfrak{ A } ^{\prime}   }         \right)   $ be an $\mathcal{L} ^{  \prime  } $-Structure, then we define 
    \[
        \mathfrak{A} ^{ \prime  } \upharpoonright_{\mathcal{L}} \stackrel{\mathtt{D}}{=} \left( A ^{ \prime  } ,  \left\{ c ^{ \mathfrak{ A } ^{ \prime  }    } : c \in  \mathcal{ C }   \right\}, \left\{ f ^{ \mathfrak{ A } ^{ \prime  }    } : f \in  \mathcal{ F }   \right\}, \left\{ R ^{ \mathfrak{ A } ^{ \prime  }    } : R \in  \mathcal{ R }   \right\}     \right)  
    \]
    In other words we have just created an $\mathcal{L}$-Structure $ \mathfrak{ A } ^{ \prime  } \upharpoonright_{\mathcal{L}} $ which intereprets everything the same was as $ \mathfrak{ A } ^{ \prime  }   $ but it is now defined for $ \mathcal{ L }   $ .
\end{definition}


\section{Deductions}
\input{logic/definitions/deduction_of_a_formula}

\begin{definition}{Logical Axioms}{logical_axioms}
 Given a first order language $ \mathcal{ L }   $ the set $\Lambda$ of logical
 axioms is the collection of all $ \mathcal{ L }-$formulas that fall into one of
 the following categories:
\begin{gather*}
    x=x  ~\text{for each variable}~ x \\
    \left[ \left( x _{ 1 } = y _{ 2 }  \right) \land \left( x _{ 2 } = y _{
    2}\right) \land \ldots \land \left( x _{ n } =  y _{ n }    \right)
    \right] \rightarrow \\
    \qquad \qquad \qquad \left( f\left( x_{1} , x_{2} , \dotsc , x_{n} \right) = f\left( y_{1} ,
    y_{2} , \dotsc , y_{n} \right)  \right) \\
    \left[\left(x_{1}=y_{1}\right) \wedge\left(x_{2}=y_{2}\right) \wedge \cdots
    \left(x_{n}=y_{n}\right)\right] \rightarrow \\
    \qquad \qquad \qquad \left(R\left(x_{1}, x_{2}, \ldots, x_{n}\right) \rightarrow
    R\left(y_{1}, y_{2}, \ldots, y_{n}\right)\right)\\
    \left( \forall x \phi  \right) \rightarrow \phi _{ t }^{ x } ~\text{if $ t $
    is substitutable for $ x $ in $ \phi  $ }~  \\
    \phi _{ t }^{ x } \to \left( \exists x \phi \right) ~\text{if $ t $ is
    substitutable for $ x $ in $ \phi  $ }~  
\end{gather*}
To refer to them easily we label them by  moving down the above list E1, E2, E3,
Q1, Q2 
\end{definition}


\begin{lemma}{Universal connection to Variable Assignment Function}{forall vaf}
    \[
    \Sigma \vdash \theta \text { if and only if } \Sigma \vdash \forall x \theta
    \]
\end{lemma}

%Note this lemma might seem quite strange, but note it actually makes sense, %todo{finish why}

\begin{proof}
    \begin{itemize}
        \item $ \Rightarrow $ 
        \begin{itemize}
            \item Suppose that $ \Sigma \vdash \theta$, therefore we have a deduction $ \mathcal{D}$ of $ \theta$, then the proof
                \begin{gather*}
                    \mathcal{D}\\
                    \left[ \left( \forall y \left( y =  y \right) \right) \land  \neg \left( \forall y \left( y =  y \right) \right) \right] \rightarrow \theta \tag{taut. PC}\\
                    \left[ \left( \forall y \left( y =  y \right) \right) \land  \neg \left( \forall y \left( y =  y \right) \right) \right] \rightarrow \left(  \forall  x \right)\theta \tag{QR}\\
                    \left(  \forall x \right) \theta \tag{PC}
                \end{gather*}
        \end{itemize}
        \item $ \Leftarrow $
        \begin{itemize}
            \item Suppose that $ \Sigma \vdash \forall x \theta$, so we have a deduction of it, call it  $\mathcal{D}$, then the following deduction suffices
            \begin{gather*}
               \mathcal{D} \\
               \forall x \theta\\
               \forall x \theta \rightarrow \theta _{x}^{x}\\
                \theta _{x}^{x}
            \end{gather*}
        \end{itemize}
    \end{itemize}
\end{proof}


\section{Completeness}

\input{logic/definitions/consistent_set_of_l_formulas}

\input{logic/propositions/contradiction_has_no_model}

\begin{proposition}
{Implying a Contradiction Means no Model}{implying_a_contradiction_means_no_model}
Let \(\Sigma \) be a set of sentences, then
\[
\Sigma \models \bot
\]
if and only if \(\Sigma \) has no model
\end{proposition}
\begin{proof}
    \begin{itemize}
        \item Recall that \(\Sigma \models \bot \) means that for any
            \(\mathcal{L}\)-Structure \(\mathfrak{A} \) we have that
            \[
            \mathfrak{A} \models \Sigma \enspace \text{implies} \enspace
            \mathfrak{A} \models \bot
            \]
            but recall that \(\mathfrak{A} \models \bot \) is always false,
            therefore for the implication to hold we require that \(\mathfrak{A}
            \models \Sigma \) to be false, which means \(\Sigma \) has no model
            as \(\mathfrak{A} \) was arbitrary.
        \item Suppose that \(\Sigma \) has no model, then that means for any
        \(\mathcal{L}\)-Structure \(\mathfrak{A} \) we have that \(\mathfrak{A}
        \not \models \Sigma \), so \( \Sigma \models \bot  \) vacuously holds.
    \end{itemize}
\end{proof}


\input{logic/propositions/logical_contradiction}

\begin{lemma}{Constant Replacement Removes It}{constant_replacement_removes_it}
Let $ \phi  $ be an $ \mathcal{ L }   $ formula, then for any $ c \in  \phi \cap  \mathcal{ C }  $ and any $ t \in  \mathfrak{ T } \left( \mathcal{ L }   \right)   $ where $ c \not\in t $ then 
\[
 c \not\in \phi _{ t }^{ c } 
\]
\end{lemma}
\begin{proof}
    The proof is by structural induction on $ \phi  $. Since formulas are constructed from terms, it must be shown on terms first:
    \begin{itemize}
        \item Base Case
        \begin{itemize}
            \item Suppose $ u :\equiv x $ where $ x $ is some variable, then $ x _{ u }^{ c } :\equiv x $ and so $ c \not\in x $, because $ \mathcal{ C } \cap \mathcal{ V } = \varnothing    $ 
            \item If $ u $ is a constant but $ c \not\in u $ then the subtitution doesn't change anything as above, but if $ u :\equiv  c $ then $ u _{ t }^{ c } :\equiv t   $ and so $ c \not\in u _{ t }^{ c }  $ 
        \end{itemize}
        \item Induction Step
        \begin{itemize}
            \item Let $ t_{1} , t_{2} , \dotsc  , t_{n - 1} , t_{n} $ be terms where the claim holds, we will show the claim holds on 
            \[
            f  t_{1} , t_{2} , \dotsc  , t_{n - 1} , t_{n} 
            \]
            \item Note that  $ \left(  f  t_{1} , t_{2} , \dotsc  , t_{n - 1} , t_{n}   \right) _{ t }^{ c } :\equiv f \left( t_{1} \right) _{ t }^{ c }, \left( t_{2} \right) _{ t }^{ c }, \ldots, \left( t_{n-1} \right) _{ t }^{ c }, \left( t_{n} \right) _{ t }^{ c }   $, and that $ c \not\in f $, therefore $ c \not\in f \left( t_{1} \right) _{ t }^{ c }, \left( t_{2} \right) _{ t }^{ c }, \ldots, \left( t_{n-1} \right) _{ t }^{ c }, \left( t_{n} \right) _{ t }^{ c }  :\equiv  f \left( t_{1} \right) _{ t }^{ c }, \left( t_{2} \right) _{ t }^{ c }, \ldots, \left( t_{n-1} \right) _{ t }^{ c }, \left( t_{n} \right) _{ t }^{ c }     $ as needed.
        \end{itemize}
    \end{itemize}
    Now for the formulas
    \begin{itemize}
        \item Base Case
        \begin{itemize}
            \item For atomic formulas, we know that they are created out of terms, and symbols which don't change under replacement (namely $ = $ and relation symbols)   therefore since the claim holds on terms, and that the newly added symbols don't change and our induction hypothesis holds for $ \alpha  $ , $ \beta  $ the claim holds.
        \end{itemize}
        \item Induction Step
        \begin{itemize}
            \item Suppose that the claim holds on formulas $ \alpha, \beta  $ then it also holds for $ \neg \alpha  $ and $ \alpha \lor \beta  $ since the new characters added to the strings are $ \lor  $ and $ \neg  $ which don't change under replacement
            \item When we have $ \left( \forall x \right) \phi  $ we can be sure that $ x  :\not\equiv c  $ as $ x \in  \mathcal{ V }   $ and $ c \in  \mathcal{ C }   $ and therefore by the definition of replacement we have that $ \left( \left( \forall x \right) \alpha \right) _{ t }^{ c }  :\equiv \left( \forall x \right) \alpha _{ t }^{ c }  $  and then by the induction hypothesis we have that $ c \not\in \left( \forall x \right) \alpha  $ 
        \end{itemize}
    \end{itemize}
\end{proof}


\begin{lemma}{Constant Extension still Consistent}{constant_extension_still_consistent}
If $\Sigma$ is a consistent set of $\mathcal{L}$-sentences and $\mathcal{L}^{\prime}$ is an extension by constants of $\mathcal{L}$, then $\Sigma$ is consistent when viewed as a set of $\mathcal{L}^{\prime}$-sentences.
\end{lemma}
\begin{proof}
    \begin{itemize}
        \item Suppose for the sake of contradiction that $ \Sigma  $ is not consistent when viewed as a set of $\mathcal{L}^{\prime}$-sentences, so $ \Sigma \vdash \bot $, considering the set of all deductions of $ \bot  $ from $ \Sigma  $ we may find a deduction $ \mathcal{ D }   $  which uses the least number of the newly added constants by the well ordering principle, let this number be $ n \in  \mathbb{N} $ and that $ n > 0 $ or else we would have a deduction from $ \mathcal{ L }   $ of $ \bot  $ which would be a contradiction.
        \item Let $ v $ be a variable that isn't used in $ \mathcal{ D }   $, note that we can do this because there are infinitely many variables, but only finitely many of them can be used in a deduction and let $ c $ be one of the newly added constants which is used in $ \mathcal{ D }   $ and let $ \mathcal{ D } _{ v } ^{ c }    $ be created where for each line $ \phi \in  \mathcal{ D }  $ we replace it with $ \phi _{ v }^{ c } $ , and note that the last line of $ \mathcal{ D } _{ v }   $ is still $ \bot  $, at this point we don't know if $ \mathcal{ D } _{ v }   $ is a deduction, so we have to check that it is.
        \begin{itemize}
            \item Note that if it is a deduction, then we've arrived at a contradiction since we have a deduction of $ \bot  $ with a number of newly added constants which is less than $ n $ (and $ n $ is minimal).
        \end{itemize}
        \item If $ \phi  $ is an equality axiom or an element of $ \Sigma $ then $ \phi _{ v } :\equiv \phi  $ because equality axioms only contain variables and not constants, also $ \Sigma  $ is a set of $ \mathcal{ L }   $ sentences and so it can't contain any of the new constants. Therefore equality axioms and any sentence from $ \Sigma  $ is not modified by this procedure, leaving them as valid steps in the deduction.
        \item If $ \phi  $ is $ \left( \forall x \right) \theta \rightarrow \theta _{ t }^{ x }  $ then $ \phi _{ v }  $ is $ \left( \forall x \right)\theta _{ v } \rightarrow \left( \theta _{ v }  \right) _{ t _{ v }  }^{ x  }  $, to see why we use $ t _{ v }  $ as well as $ \theta _{ v }  $ try $ \theta :\equiv c =  x $ and $ t :\equiv   c + 3 $ 
    \end{itemize}
\end{proof}



Notice that if we write $ \Sigma \models \bot $ it means that for any $\mathcal{L}$-Structure $ \mathfrak{ A }   $ if $ \mathfrak{ A } \models \Sigma   $ then $ \mathfrak{ A } \models \bot  $ by the above discussion that forces $ \mathfrak{ A } \models \Sigma   $ to be false, and therefore $ \Sigma  $ has no model.


\begin{theorem}{Completeness Theorem}{completeness}
Suppose that $\Sigma$ is a set of $\mathcal{L}$-formulas, where $ \mathcal{L}$ is a countable langauge  and $\phi$ is an $\mathcal{L}$-formula. If $\Sigma \models \phi$, then $\Sigma \vdash \phi$.

\section*{Setup}

\begin{itemize}
    \item We start by assuming that $ \Sigma \models \phi$, we must show that $ \Sigma \vdash \phi$.
    \item If $ \phi$ is not a sentence then we can always prove $ \phi'$ which is the same as $ \phi$ with all of it's variables bound
    \begin{itemize}
        \item We can do that by appending $ \left( \forall  x _{f}  \right)$ where each $ x_{f}$  is a free varaible of $ \phi$ to the front of $ \phi$ 
    \end{itemize}
\item Therefore we will prove it for all sentences $ \phi$ %\todo[inline]{justify why this is equivalent}
\end{itemize}

\end{theorem}

\begin{proposition}{Finite Subsets having a Model implies Original Set
does}{finite_subsets_having_a_model_implies_original_set_does}
Let \( \Sigma  \) be a set of sentences, then if every finite subset of \(
\Sigma  \) has a model, then \( \Sigma  \) has a model
\end{proposition}
\begin{proof}
    Suppose for the sake of contradiction that every finite subset of \(
    \Sigma  \)  has a model, but \( \Sigma  \) doesn't have a model, by
    \hyperref[propositions:implying_a_contradiction_means_no_model]{the fact
    that \( \Sigma \models \bot  \) if and only if \( \Sigma  \) has no model},
    we can see that \( \Sigma \models \bot  \), by completeness \( \Sigma \vdash
    \bot \) and thus there is a finite deduction \( \mathcal{ D }   \) of \(
    \bot  \), consider \( \Sigma _{ \mathcal{ D }   }  \) the set of all
    elements of \( \Sigma \) which are used in the deduction of \( \bot  \), we
    can see that \( \Sigma _{ \mathcal{ D }   } \vdash \bot \) thus by soundness
    \( \Sigma _{ \mathcal{ D}   } \models \bot  \) which means that \( \Sigma _{
    \mathcal{ D }   }  \) has no model, but \( \Sigma _{ \mathcal{ D }   }  \)
    is a finite subset of \( \Sigma  \) and by assumption must have a model,
    thus we've arrived at a contradiction and so \( \Sigma  \) must have a
    model.
\end{proof}


\begin{proposition}{Sentences which force an Infinite
    Model}{sentences_which_force_an_infinite_model}
    We define: 
    \[
     \phi _{ n }  :\equiv \left( \exists x _{ 1 }   \right) \ldots \left(
     \exists x _{ n }   \right) \left[ \bigwedge _{ i, j \din \left[ n \right] } x _{ i
     } \neq x _{ j } \right]  \quad \text{and} \quad \Phi \stackrel{\mathtt{D}}{=}
     \left\{ \phi  _{ n } : n \in  \mathbb{N} \right\} 
    \]
    Then any model of \( \Phi  \) has an infinite universe.
\end{proposition}
\begin{proof}
\begin{itemize}
    \item Let \( \mathfrak{ A }   \) be a model of \( \Phi  \), let's first
    figure out what it means for \( \mathfrak{ A } \models \phi _{ k }   \) for
    some \( k \in  \mathbb{N}  \). 
    \( \mathfrak{ A } \models \phi _{ k }   \) if and only if for each there
    exists \( a_{1} , a_{2} , \dotsc , a_{n} \in  A \) such that for every \( i,
    j \din \left[ k \right] \) we have \( \mathfrak{ A } \models x _{ i } \neq x
    _{ j } \left[ s \left[ x _{ i }  \mid a _{ i }  \right] \left[ x _{ j }
   \mid a _{ j }  \right]   \right] \) (Note that in reality the modified
   assignment function would be \( \left[ s \left[ x _{ 1 } \mid a _{ 1 }
   \right] \ldots \left[ x _{ k }  \mid a _{ k }  \right]   \right]  \) but it
   only matters what it does to \( x _{ i }  \) and \( x _{ j }  \)). 
   \item Now recall that due to the modification of the assignment function \(
   \mathfrak{ A } \models x _{ i } \neq x _{ j } \left[ s \left[ x _{ i }  \mid
   a _{ i } \right] \left[ x _{ j } \mid a _{ j }  \right]  \right]   \) simply
   says that \( a _{ i }  \) is not the same element of \( A \) as \( a _{ j }
   \), and in total it claims that any two elements are different, thus they are
   all distinct, and so any model of \( \phi _{ k }  \) has at least \( k \)
   elements. Thus any model of \( \Phi  \) has more elements than any natural
   number, so that any model has a universe of infintely many elements.
\end{itemize}
\end{proof}






\section{Incompleteness}

\begin{definition}{Bounded Quantifier Abbreviations}{bounded_quantifier_abbreviations}
If $x$ is a variable that does not occur in the term $t$, let us agree to use the following abbreviations:
\begin{gather*}
(\forall x<t) \phi \text { means } \forall x(x<t \rightarrow \phi) \\
(\forall x \leq t) \phi \text { means } \forall x((x<t \vee x=t) \rightarrow \phi) \\
(\exists x<t) \phi \text { means } \exists x(x<t \wedge \phi) \\
(\exists x \leq t) \phi \text { means } \exists x((x<t \vee x=t) \wedge \phi)
\end{gather*}
These abbreviations will constitute the set of bounded quantifiers.
\end{definition}


\begin{definition}{Sigma Formula}{sigma_formula}
The collection of $\Sigma$-formulas is defined as the smallest set of $\mathcal{L}_{N T}$ formulas such that:
\begin{enumerate}
    \item Every atomic formula is a $\Sigma$-formula.
    \item Every negation of an atomic formula is a $\Sigma$-formula.
    \item If $\alpha$ and $\beta$ are $\Sigma$-formulas, then $\alpha \wedge \beta$ and $\alpha \vee \beta$ are both $\Sigma$-formulas.
    \item If $\alpha$ is a $\Sigma$-formula, and $x$ is a variable that does not occur in the term $t$, then the following are $\Sigma$-formulas: $(\forall x<t) \alpha,(\forall x \leq t) \alpha$, $(\exists x<t) \alpha,(\exists x \leq t) \alpha$
    \item If $\alpha$ is a $\Sigma$-formula and $x$ is a variable, then $(\exists x) \alpha$ is a $\Sigma$-formula.
\end{enumerate}
\end{definition}


\begin{definition}{Pi Formula}{pi_formula}
The collection of $\Pi$-formulas is the smallest set of $\mathcal{L}_{N T}$ formulas such that:
\begin{enumerate}
    \item Every atomic formula is a $\Pi$-formula.
    \item Every negation of an atomic formula is a $\Pi-$formula.
    \item If $\alpha$ and $\beta$ are $\Pi-$formulas, then $\alpha \wedge \beta$ and $\alpha \vee \beta$ are both $\Pi-$formulas.
    \item If $\alpha$ is a $\Pi$-formula, and $x$ is a variable that does not occur in the term $t$, then the following are $\Pi-$formulas: $(\forall x<t) \alpha,(\forall x \leq t) \alpha$ $(\exists x<t) \alpha,(\exists x \leq t) \alpha$
    \item If $\alpha$ is a $\Pi$-formula and $x$ is a variable, then $(\forall x) \alpha$ is a $\Pi$-formula.
\end{enumerate}
\end{definition}


\begin{proposition}
{Negation of a Sigma Formula is a Pi Formula}{negation_of_a_sigma_formula_is_a_pi_formula}
    Suppose \(\sigma \) is a \(\Sigma \)-formula, then \(\neg \sigma \) is equivalent
a \(\Pi \)-formula
\end{proposition}
\begin{proof}
We will proceed by induction on the complexity of the formula, in other words,
we will do structural induction
    \begin{itemize}
        \item Base Case
        \begin{itemize}
            \item Suppose \(\phi \) is an atomic \(\Sigma \)-formula, that means
            that it is an atomic formula in the sense of a normal formula,
            therefore it's negation is the negation of an atomic formula and is
            therefore a \(\Pi \)-formula
            \item Suppose \(\phi :\equiv \neg \alpha \) where \(\alpha \) is an
            atomic formula then by
            \hyperref[proposition:double_negation_equivalence]{the fact that
            double negation ``cancels''} we see that \(\phi \) is equivalent to
            \(\alpha \) which is an atomic formula, and therefore \(\alpha \) is
            a \(\Pi \)-formula
        \end{itemize}
        \item Induction Step
        \begin{itemize}
\item Suppose \(\alpha \) and \(\beta \) are \(\Sigma
        \)-formulas where the claim holds
            \begin{itemize}
                \item \(\phi :\equiv \alpha \lor \beta \), then \(\neg \phi \) is
                equivalent to \(\neg \alpha \land \neg \beta \) and therefore by
                our induction hypothesis we know that \(\neg \alpha \) and \(
                \neg \beta \) are equivent to some \(\Pi \)-formulas \(\pi _{
                \alpha}, \pi _{\beta} \) then \(\phi \) is equivalent to
                \(\pi _{\alpha} \land \pi _{\beta} \) which is a \(\Pi
                \)-formula by the third clause. \item \(\phi :\equiv \alpha
                \land \beta \) then \(\neg \phi \) is equivalent to \(\neg \alpha
                \lor \neg \beta \) and by the same argument above, we can see
                that this is a \(\Pi \)-formula \item \(\phi :\equiv \left(
                \forall x < t\right) \alpha \), then it's negation is \(\left(
                \exists x\right) \left(x < t \land \neg \alpha\right) \) by
                our induction hypothesis we know that \(\neg \alpha \) is
                equivalent to some \(\Pi \)-formula \(\pi _{\neg \alpha} \) so \(
                \left(\exists x\right) \left(x < t \land \neg \alpha \right) \)is
                equivalent to 
                \[
                \left(\exists x\right) \left(x < t \land \pi
                _{\neg \alpha}\right) :\equiv \left(\exists x < t\right) \pi
                _{\neg \alpha} 
                \]
                and therefore we may say that \(\phi \) is
                equivalent to a \(\Pi \)-formula.
                \item Suppose \(\phi :\equiv \left(\forall x \le t\right) \)
                then \(\neg \phi \) is equivalent to 
                \[
                    \left(\exists x\right)
                    \neg \left(\left(x < t \lor x = t\right) \rightarrow \alpha
                    \right) :\equiv \left(\exists x\right) \neg \left(\neg
                    \left(x < t \lor x = t\right) \lor \alpha\right) 
                \]
                which
                is equivalent to \(\left(\exists x\right) \left(\left(x < t
                \lor x = t\right) \land \neg \alpha\right) :\equiv \left(
                \exists x \le t\right) \neg \alpha \) and since \(\neg
                \alpha \) is equivalent to some \(\Pi \)-formula, \(\pi _{\neg
                \alpha} \) by our inductive hypothesis, we know that \(\left(
                \exists x \le t\right) \pi _{\neg \alpha} \) is a \(\Pi
                \)-formula.
                \item Suppose \(\phi :\equiv \left(\exists x < t\right)
                \alpha \) then the negation is equivalent to \(\left(\forall x
                \right) \neg \left(\neg \left(x < t\right) \lor \neg \alpha
                \right) :\equiv \left(\forall x\right) \left(x < t
                \rightarrow \neg \alpha\right) :\equiv \left(\forall x < t
                \right) \neg \alpha \), and following similar logic as above
                we can see that \(\left(\forall x < t\right) \neg \alpha \) is
                a \(\Pi \)-formula
                \item \(\phi :\equiv \left(\exists x \le t\right) \alpha \)
                the negation is equivalent to 
                \[
                \left(\forall x\right) \left(
                \neg \left(\left(x < t\right) \lor \left(x = t\right) \lor
                \neg \alpha\right)\right) 
                \]
                which is the same as
                \[
                \left(\forall x\right)
                \left(\left(x < t\right) \lor \left(x = t\right)\right)
                \rightarrow \neg \alpha :\equiv \left(\forall x \le t\right)
                \neg \alpha 
                \] as needed.
            \end{itemize}
        \end{itemize}
    \end{itemize}
\end{proof}


\begin{definition}
{Representable Set}{representable_set}
A set \(A \subseteq \mathbb{N}^{k}\) is said to be representable (in \(N\)) if
there is an \(\mathcal{L}_{N T}\)-formula \(\phi(\undervec{x})\) such that
\[
\begin{array}
{ll}
\forall \undervec{a} \in A & N \vdash \phi(\overline{\undervec{a}}) \\
\forall \undervec{b} \notin A & N \vdash \neg \phi(\overline{\undervec{b}})
\end{array}
\]
In this case we will say that the formula \(\phi\) represents the set \(A\).
\end{definition}

\input{logic/definitions/weakly_representable_set}
\input{logic/definitions/total_function}
\input{logic/definitions/partial_function}
\begin{definition}
{Representable Function}{representable_function}
Suppose that \(f: \mathbb{N}^{k} \rightarrow \mathbb{N}\) is a total function.
We will say that \(f\) is a representable function (in \(N\)) if there is an
\(\mathcal{L}_{N T}\) formula \(\phi\left(x_{1}, \ldots, x_{k+1}\right)\) such
that, for all \(a_{1}, a_{2}, \ldots a_{k+1} \in \mathbb{N}\)
\begin{gather*}
    \text{If} \enspace f\left(a_{1}, \ldots, a_{k}\right) = a_{k+1}, \enspace
    \text{then} \enspace N \vdash \phi\left(\overline{a_{1}}, \ldots,
    \overline{a_{k+1}}\right) \\
    \text{If} \enspace f\left(a_{1}, \ldots, a_{k}\right) \neq a_{k+1}, \enspace
    \text{then} \enspace   N \vdash \neg
    \phi\left(\overline{a_{1}}, \ldots, \overline{a_{k+1}}\right)
\end{gather*}
\end{definition}

\input{logic/definitions/definable_set}
\input{logic/corollaries/definable_by_delta_formula_implies_representable}
