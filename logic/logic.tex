\chapter{First Order Logic}

\section{Languages}

\begin{definition}{First-Order Language}{first_order_language}
	A first order language $\mathcal{L}$ is an infinite collection of distinct symbols, no one of which is properly contained in another, sparated into the following cateogories:
	\begin{enumerate}
		\item Parentheses: (,).
		\item Connectives: $\lor, \neg$.
		\item Quantifier: $\forall$.
		\item Variables, one for each positive integer n: $v_{1}, v_{2}, \ldots, v_{n}, \ldots$ The set of variable symbols will be denoted $ \mathcal{ V }   $ .
		\item Equality symbol: $=$.
		\item Constant symbols: Some set of zero or more symbols. This set will be denoted as $ \mathcal{ C }   $ 
		\item Function symbols: For each positive integer $n$, some set of zero or more $n$-ary function symbols as $ \mathcal{ F }  $ .
		\item Relation symbols: For each positive integer $n$, some set of zero or more $n$-ary relation symbols as $ \mathcal{ R }  $ .
	\end{enumerate}
	We may denote a language $ \mathcal{ L }   $ as $ \left( \mathcal{ V } ,\mathcal{ C } , \mathcal{ F } , \mathcal{ R }      \right)  $ as these are the only parts of a the language which may vary.
\end{definition}


\input{logic/definitions/a_term's_variable_replaced_by_a_term}

\begin{itemize}
    \item  In the fourth clause of the definition above and in the first two clauses of the next definition, the parentheses are not really there. Because $u_{1}{ }_{t}^{x}$ is hard to read so the parentheses have been added in the interest of readability.
    \item If we let $t$ be $g(c)$ and we let $u$ be $f(x, y)+h(z, x, g(x))$, then $u_{t}^{x}$ is
        \[
        f(g(c), y)+h(z, g(c), g(g(c)))
        \]
\end{itemize}


\input{logic/definitions/a_term_is_substitutable_for_a_variable.tex}

To understand the motivation behind the fourth clause, consider the formula:
\[
\phi :\equiv \left( \forall x \right) \left( \forall y \right) x = y
\]

Then one might want to say that $ \left( \left( \forall y \right) x = y \right) _{ t }^{ x }   $  where $ t $ is any term

\input{logic/definitions/free_variable_in_a_formula.tex}

\subsubsection*{Examples}
\begin{itemize}
    \item Thus, if we look at the formula
    $$
    \forall v_{2} \neg\left(\forall v_{3}\right)\left(v_{1}=S\left(v_{2}\right) \vee v_{3}=v_{2}\right)
    $$
    the variable $v_{1}$ is free whereas the variables $v_{2}$ and $v_{3}$ are not free. 
    \item A slightly more complicated example is
        \[
        \left(\forall v_{1} \forall v_{2}\left(v_{1}+v_{2}=0\right)\right) \vee v_{1}=S(0)
        \]
        \begin{itemize}
            \item In this formula, $v_{1}$ is free whereas $v_{2}$ is not free. Especially when a formula is presented informally, you must be careful about the scope of the quantifiers and the placement of parentheses.
        \end{itemize}
\end{itemize}

\subsubsection*{Notes}
\begin{itemize}
    \item We will have occasion to use the informal notation $\forall x \phi(x)$. This will mean that $\phi$ is a formula and $x$ is among the free variables of $\phi$. 
    \item If we then write $\phi(t)$, where $t$ is an $\mathcal{L}$-term, that will denote the formula obtained by taking $\phi$ and replacing each occurrence of the variable $x$ with the term $t$. 
\end{itemize}

\begin{definition}{L-Structure}{l-structure}
Fix a language $\mathcal{L}$. An $\mathcal{L}$-structure $\mathfrak{A}$ is a nonempty set $A$, called the universe of $\mathfrak{A}$, together with:
\begin{enumerate}
    \item For each constant symbol $c$ of $\mathcal{L}$, an element $c^{\mathfrak{A}}$ of $A$,
    \item For each $n$-ary function symbol $f$ of $\mathcal{L}$, a function $f^{\mathfrak{A}}: A^{n} \rightarrow A$, and
    \item For each $n$-ary relation symbol $R$ of $\mathcal{L}$, an $n$-ary relation $R^{\mathfrak{A}}$ on $A$ (i.e., a subset of $A^{n}$ ).
\end{enumerate}
We may denote an $\mathcal{L}$-Structure as follows
\[
\left( A, \mathcal{ C } ^{  \mathfrak{ A }   }  \stackrel{\mathtt{D}}{=} \left\{ c ^{ \mathfrak{ A }   } : c \in \mathcal{ C }   \right\}, \mathcal{ F } ^{ \mathfrak{ A }   } = \left\{ f ^{ \mathfrak{ A }   } : f \in  \mathcal{ F }   \right\}, \mathcal{ R } ^{ \mathfrak{ A }   } = \left\{ R ^{ \mathfrak{ A }   } : R \in  \mathcal{ R }   \right\}        \right) 
\]
\end{definition}


\begin{definition}{Restricted L-Structure}{restricted_l-structure}
    Suppose $ \mathcal{ L } =  \left( \mathcal{ V } , \mathcal{ C } , \mathcal{ F } , \mathcal{ R }       \right)    $ and $ \mathcal{ L } ^{ \prime  } =  \left( \mathcal{ V } ^{ \prime  }  , \mathcal{ C } ^{ \prime  }  , \mathcal{ F } ^{  \prime  }  , \mathcal{ R } ^{ \prime  }        \right)   $  are languages and $ \mathfrak{A} ^{ \prime  } =\left(A ^{ \prime  } , \mathcal{ C ^{ \prime  }  } ^{ \mathfrak{ A } ^{\prime} }  , \mathcal{ F ^{ \prime  }  } ^{ \mathfrak{ A } ^{\prime}   },\mathcal{ R ^{ \prime  }  } ^{ \mathfrak{ A } ^{\prime}   }         \right)   $ be an $\mathcal{L} ^{  \prime  } $-Structure, then we define 
    \[
        \mathfrak{A} ^{ \prime  } \upharpoonright_{\mathcal{L}} \stackrel{\mathtt{D}}{=} \left( A ^{ \prime  } ,  \left\{ c ^{ \mathfrak{ A } ^{ \prime  }    } : c \in  \mathcal{ C }   \right\}, \left\{ f ^{ \mathfrak{ A } ^{ \prime  }    } : f \in  \mathcal{ F }   \right\}, \left\{ R ^{ \mathfrak{ A } ^{ \prime  }    } : R \in  \mathcal{ R }   \right\}     \right)  
    \]
    In other words we have just created an $\mathcal{L}$-Structure $ \mathfrak{ A } ^{ \prime  } \upharpoonright_{\mathcal{L}} $ which intereprets everything the same was as $ \mathfrak{ A } ^{ \prime  }   $ but it is now defined for $ \mathcal{ L }   $ .
\end{definition}



\section{Deductions}
\input{logic/definitions/deduction_of_a_formula}

\begin{definition}{Logical Axioms}{logical_axioms}
 Given a first order language $ \mathcal{ L }   $ the set $\Lambda$ of logical
 axioms is the collection of all $ \mathcal{ L }-$formulas that fall into one of
 the following categories:
\begin{gather*}
    x=x  ~\text{for each variable}~ x \\
    \left[ \left( x _{ 1 } = y _{ 2 }  \right) \land \left( x _{ 2 } = y _{
    2}\right) \land \ldots \land \left( x _{ n } =  y _{ n }    \right)
    \right] \rightarrow \\
    \qquad \qquad \qquad \left( f\left( x_{1} , x_{2} , \dotsc , x_{n} \right) = f\left( y_{1} ,
    y_{2} , \dotsc , y_{n} \right)  \right) \\
    \left[\left(x_{1}=y_{1}\right) \wedge\left(x_{2}=y_{2}\right) \wedge \cdots
    \left(x_{n}=y_{n}\right)\right] \rightarrow \\
    \qquad \qquad \qquad \left(R\left(x_{1}, x_{2}, \ldots, x_{n}\right) \rightarrow
    R\left(y_{1}, y_{2}, \ldots, y_{n}\right)\right)\\
    \left( \forall x \phi  \right) \rightarrow \phi _{ t }^{ x } ~\text{if $ t $
    is substitutable for $ x $ in $ \phi  $ }~  \\
    \phi _{ t }^{ x } \to \left( \exists x \phi \right) ~\text{if $ t $ is
    substitutable for $ x $ in $ \phi  $ }~  
\end{gather*}
To refer to them easily we label them by  moving down the above list E1, E2, E3,
Q1, Q2 
\end{definition}


\begin{lemma}{Universal connection to Variable Assignment Function}{forall vaf}
    \[
    \Sigma \vdash \theta \text { if and only if } \Sigma \vdash \forall x \theta
    \]
\end{lemma}

%Note this lemma might seem quite strange, but note it actually makes sense, %todo{finish why}

\begin{proof}
    \begin{itemize}
        \item $ \Rightarrow $ 
        \begin{itemize}
            \item Suppose that $ \Sigma \vdash \theta$, therefore we have a deduction $ \mathcal{D}$ of $ \theta$, then the proof
                \begin{gather*}
                    \mathcal{D}\\
                    \left[ \left( \forall y \left( y =  y \right) \right) \land  \neg \left( \forall y \left( y =  y \right) \right) \right] \rightarrow \theta \tag{taut. PC}\\
                    \left[ \left( \forall y \left( y =  y \right) \right) \land  \neg \left( \forall y \left( y =  y \right) \right) \right] \rightarrow \left(  \forall  x \right)\theta \tag{QR}\\
                    \left(  \forall x \right) \theta \tag{PC}
                \end{gather*}
        \end{itemize}
        \item $ \Leftarrow $
        \begin{itemize}
            \item Suppose that $ \Sigma \vdash \forall x \theta$, so we have a deduction of it, call it  $\mathcal{D}$, then the following deduction suffices
            \begin{gather*}
               \mathcal{D} \\
               \forall x \theta\\
               \forall x \theta \rightarrow \theta _{x}^{x}\\
                \theta _{x}^{x}
            \end{gather*}
        \end{itemize}
    \end{itemize}
\end{proof}


\section{Completeness}

\input{logic/definitions/consistent_set_of_l_formulas}

\input{logic/propositions/contradiction_has_no_model}

\begin{proposition}
{Implying a Contradiction Means no Model}{implying_a_contradiction_means_no_model}
Let \(\Sigma \) be a set of sentences, then
\[
\Sigma \models \bot
\]
if and only if \(\Sigma \) has no model
\end{proposition}
\begin{proof}
    \begin{itemize}
        \item Recall that \(\Sigma \models \bot \) means that for any
            \(\mathcal{L}\)-Structure \(\mathfrak{A} \) we have that
            \[
            \mathfrak{A} \models \Sigma \enspace \text{implies} \enspace
            \mathfrak{A} \models \bot
            \]
            but recall that \(\mathfrak{A} \models \bot \) is always false,
            therefore for the implication to hold we require that \(\mathfrak{A}
            \models \Sigma \) to be false, which means \(\Sigma \) has no model
            as \(\mathfrak{A} \) was arbitrary.
        \item Suppose that \(\Sigma \) has no model, then that means for any
        \(\mathcal{L}\)-Structure \(\mathfrak{A} \) we have that \(\mathfrak{A}
        \not \models \Sigma \), so \( \Sigma \models \bot  \) vacuously holds.
    \end{itemize}
\end{proof}


\input{logic/propositions/logical_contradiction}

\begin{lemma}{Constant Extension still Consistent}{constant_extension_still_consistent}
If $\Sigma$ is a consistent set of $\mathcal{L}$-sentences and $\mathcal{L}^{\prime}$ is an extension by constants of $\mathcal{L}$, then $\Sigma$ is consistent when viewed as a set of $\mathcal{L}^{\prime}$-sentences.
\end{lemma}
\begin{proof}
    \begin{itemize}
        \item Suppose for the sake of contradiction that $ \Sigma  $ is not consistent when viewed as a set of $\mathcal{L}^{\prime}$-sentences, so $ \Sigma \vdash \bot $, considering the set of all deductions of $ \bot  $ from $ \Sigma  $ we may find a deduction $ \mathcal{ D }   $  which uses the least number of the newly added constants by the well ordering principle, let this number be $ n \in  \mathbb{N} $ and that $ n > 0 $ or else we would have a deduction from $ \mathcal{ L }   $ of $ \bot  $ which would be a contradiction.
        \item Let $ v $ be a variable that isn't used in $ \mathcal{ D }   $, note that we can do this because there are infinitely many variables, but only finitely many of them can be used in a deduction and let $ c $ be one of the newly added constants which is used in $ \mathcal{ D }   $ and let $ \mathcal{ D } _{ v } ^{ c }    $ be created where for each line $ \phi \in  \mathcal{ D }  $ we replace it with $ \phi _{ v }^{ c } $ , and note that the last line of $ \mathcal{ D } _{ v }   $ is still $ \bot  $, at this point we don't know if $ \mathcal{ D } _{ v }   $ is a deduction, so we have to check that it is.
        \begin{itemize}
            \item Note that if it is a deduction, then we've arrived at a contradiction since we have a deduction of $ \bot  $ with a number of newly added constants which is less than $ n $ (and $ n $ is minimal).
        \end{itemize}
        \item If $ \phi  $ is an equality axiom or an element of $ \Sigma $ then $ \phi _{ v } :\equiv \phi  $ because equality axioms only contain variables and not constants, also $ \Sigma  $ is a set of $ \mathcal{ L }   $ sentences and so it can't contain any of the new constants. Therefore equality axioms and any sentence from $ \Sigma  $ is not modified by this procedure, leaving them as valid steps in the deduction.
        \item If $ \phi  $ is $ \left( \forall x \right) \theta \rightarrow \theta _{ t }^{ x }  $ then $ \phi _{ v }  $ is $ \left( \forall x \right)\theta _{ v } \rightarrow \left( \theta _{ v }  \right) _{ t _{ v }  }^{ x  }  $, to see why we use $ t _{ v }  $ as well as $ \theta _{ v }  $ try $ \theta :\equiv c =  x $ and $ t :\equiv   c + 3 $ 
    \end{itemize}
\end{proof}



Notice that if we write $ \Sigma \models \bot $ it means that for any $\mathcal{L}$-Structure $ \mathfrak{ A }   $ if $ \mathfrak{ A } \models \Sigma   $ then $ \mathfrak{ A } \models \bot  $ by the above discussion that forces $ \mathfrak{ A } \models \Sigma   $ to be false, and therefore $ \Sigma  $ has no model.


\begin{theorem}{Completeness Theorem}{completeness}
Suppose that $\Sigma$ is a set of $\mathcal{L}$-formulas, where $ \mathcal{L}$ is a countable langauge  and $\phi$ is an $\mathcal{L}$-formula. If $\Sigma \models \phi$, then $\Sigma \vdash \phi$.

\section*{Setup}

\begin{itemize}
    \item We start by assuming that $ \Sigma \models \phi$, we must show that $ \Sigma \vdash \phi$.
    \item If $ \phi$ is not a sentence then we can always prove $ \phi'$ which is the same as $ \phi$ with all of it's variables bound
    \begin{itemize}
        \item We can do that by appending $ \left( \forall  x _{f}  \right)$ where each $ x_{f}$  is a free varaible of $ \phi$ to the front of $ \phi$ 
    \end{itemize}
\item Therefore we will prove it for all sentences $ \phi$ %\todo[inline]{justify why this is equivalent}
\end{itemize}

\end{theorem}


\section{Incompleteness}

\begin{definition}
{Representable Set}{representable_set}
A set \(A \subseteq \mathbb{N}^{k}\) is said to be representable (in \(N\)) if
there is an \(\mathcal{L}_{N T}\)-formula \(\phi(\undervec{x})\) such that
\[
\begin{array}
{ll}
\forall \undervec{a} \in A & N \vdash \phi(\overline{\undervec{a}}) \\
\forall \undervec{b} \notin A & N \vdash \neg \phi(\overline{\undervec{b}})
\end{array}
\]
In this case we will say that the formula \(\phi\) represents the set \(A\).
\end{definition}

\input{logic/definitions/weakly_representable_set}
\input{logic/definitions/total_function}
\input{logic/definitions/partial_function}
\begin{definition}
{Representable Function}{representable_function}
Suppose that \(f: \mathbb{N}^{k} \rightarrow \mathbb{N}\) is a total function.
We will say that \(f\) is a representable function (in \(N\)) if there is an
\(\mathcal{L}_{N T}\) formula \(\phi\left(x_{1}, \ldots, x_{k+1}\right)\) such
that, for all \(a_{1}, a_{2}, \ldots a_{k+1} \in \mathbb{N}\)
\begin{gather*}
    \text{If} \enspace f\left(a_{1}, \ldots, a_{k}\right) = a_{k+1}, \enspace
    \text{then} \enspace N \vdash \phi\left(\overline{a_{1}}, \ldots,
    \overline{a_{k+1}}\right) \\
    \text{If} \enspace f\left(a_{1}, \ldots, a_{k}\right) \neq a_{k+1}, \enspace
    \text{then} \enspace   N \vdash \neg
    \phi\left(\overline{a_{1}}, \ldots, \overline{a_{k+1}}\right)
\end{gather*}
\end{definition}

\input{logic/definitions/definable_set}
\input{logic/corollaries/definable_by_delta_formula_implies_representable}
