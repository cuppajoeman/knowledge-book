\begin{definition}{First-Order Language}{first_order_language}
	A first order language $\mathcal{L}$ is an infinite collection of distinct symbols, no one of which is properly contained in another, sparated into the following cateogories:
	\begin{enumerate}
		\item Parentheses: (,).
		\item Connectives: $\lor, \neg$.
		\item Quantifier: $\forall$.
		\item Variables, one for each positive integer n: $v_{1}, v_{2}, \ldots, v_{n}, \ldots$ The set of variable symbols will be denoted $ \mathcal{ V }   $ .
		\item Equality symbol: $=$.
		\item Constant symbols: Some set of zero or more symbols. This set will be denoted as $ \mathcal{ C }   $ 
		\item Function symbols: For each positive integer $n$, some set of zero or more $n$-ary function symbols as $ \mathcal{ F }  $ .
		\item Relation symbols: For each positive integer $n$, some set of zero or more $n$-ary relation symbols as $ \mathcal{ R }  $ .
	\end{enumerate}
	We may denote a language $ \mathcal{ L }   $ as $ \left( \mathcal{ V } ,\mathcal{ C } , \mathcal{ F } , \mathcal{ R }      \right)  $ as these are the only parts of a the language which may vary.
\end{definition}
