\begin{proposition}
{Subset of N is Representable iff it's Characteristic
Function is}{subset_of_n_is_representable_iff_it's_characteristic_function_is}
Prove that a subset \(A\) of the natural numbers is representable if, and only
if, its characteristic function \(\chi_{A}: \mathbf{N} \rightarrow \mathbf{N}\)
is representable.
\end{proposition}
\begin{proof}
    \( \Rightarrow \) Suppose that the subset \( A \) is representable, so we
    have some formula \( \phi  \) such that 
    \begin{gather*}
        \forall a \in  A, N \vdash \phi \left( a \right) \\
        \forall a \not\in   A, N \vdash \neg \phi \left( a \right)
    \end{gather*}
    We want to prove that \( A \)'s characteristic function is representable so
    consider the following formula
    \[
        \psi :\equiv \left( \phi \left( x \right) \land \left( y = 1 \right)
        \right)   \lor \left( \neg \phi \left( x \right) \land \left( y = 0 \right)   \right) 
    \]
    \begin{enumerate}
        \item Suppose \( a \in A \), in that case \( \chi _{ A } \left( a
        \right) = 1  \) and \( N \vdash \phi \left( a \right) \) 
        \begin{itemize}
            \item In this case  \( \chi _{ A } \left( a \right) = 1 \) then we
            can see that \( \psi  \)  holds via the fact that we have a
            deduction of \( \phi \left( a \right)  \) and that we can always
            deduce \( 1 = 1 \).
            \item We know that \( \chi _{ A } \left( a \right) \neq 0 \) and so
            we check that \( N \vdash \neg \psi  \) which follows from the fact
            that we have a deduction of \( \phi \left( a \right)  \) but \( y =
            0\) thus in terms of logic we can see that \( \phi \left( a \right)
            \land 0 = 1\), \( \neg \phi \left( a \right) \land 0 = 0  \) are
            both false, well in that case \( \neg \psi  \) follows from
            propositional consequence.
        \end{itemize}
        \item Suppose \( a \not\in  A \), in that case \( \chi _{ A } \left( a
        \right) = 0  \) and \( N \vdash \neg \phi \left( a \right) \) 
        \begin{itemize}
            \item The proof is identical to the above where we simply note that
            when \( \chi _{ A } \left( a \right) = 0   \) the same logic we
            used with \( \left( \phi \left( x \right) \land \left( y = 1 \right)
            \right)  \) will work for \( \left( \neg \phi \left( x \right) \land
            \left( y = 0 \right) \right)  \) and that \( \chi _{ A } \left( a
        \right) \neq 1 \) and then as before \( \neg \psi  \) will hold
        \end{itemize}
    \end{enumerate}
    \( \Leftarrow  \) We suppose that \( \chi _{ A }  \) is representable and we
    want to show that the set \( A \) is representable, we will consider the
    formula \( \psi \left( x \right)  :\equiv \phi \left( x, 1 \right)  \) .
    Let \( a \in \mathbb{N}  \), if \( a \in A \) then we need to verify that \(
    N \vdash \psi \left( a \right) \), that's certainly true since \( N \vdash
    \phi \left( a, 1 \right)  \). Then suppose that \( a \not\in  A \),
    therefore \( \chi _{ A } \left( a  \right) = 0 \) and thus since \( \chi _{
    A} \left( a \right) \neq 1 \) we can see that \( N \vdash  \neg \phi \left(
    a, 1\right)  \) or namely \( \mathbb{N} \vdash \neg \psi \left( a \right)   \) 
\end{proof}

