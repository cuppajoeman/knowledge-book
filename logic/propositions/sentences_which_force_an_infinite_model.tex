\begin{proposition}{Sentences which force an Infinite
    Model}{sentences_which_force_an_infinite_model}
    We define: 
    \[
     \phi _{ n }  :\equiv \left( \exists x _{ 1 }   \right) \ldots \left(
     \exists x _{ n }   \right) \left[ \bigwedge _{ i, j \din \left[ n \right] } x _{ i
     } \neq x _{ j } \right]  \quad \text{and} \quad \Phi \stackrel{\mathtt{D}}{=}
     \left\{ \phi  _{ n } : n \in  \mathbb{N} \right\} 
    \]
    Then any model of \( \Phi  \) has an infinite universe.
\end{proposition}
\begin{proof}
\begin{itemize}
    \item Let \( \mathfrak{ A }   \) be a model of \( \Phi  \), let's first
    figure out what it means for \( \mathfrak{ A } \models \phi _{ k }   \) for
    some \( k \in  \mathbb{N}  \). 
    \( \mathfrak{ A } \models \phi _{ k }   \) if and only if for each there
    exists \( a_{1} , a_{2} , \dotsc , a_{n} \in  A \) such that for every \( i,
    j \din \left[ k \right] \) we have \( \mathfrak{ A } \models x _{ i } \neq x
    _{ j } \left[ s \left[ x _{ i }  \mid a _{ i }  \right] \left[ x _{ j }
   \mid a _{ j }  \right]   \right] \) (Note that in reality the modified
   assignment function would be \( \left[ s \left[ x _{ 1 } \mid a _{ 1 }
   \right] \ldots \left[ x _{ k }  \mid a _{ k }  \right]   \right]  \) but it
   only matters what it does to \( x _{ i }  \) and \( x _{ j }  \)). 
   \item Now recall that due to the modification of the assignment function \(
   \mathfrak{ A } \models x _{ i } \neq x _{ j } \left[ s \left[ x _{ i }  \mid
   a _{ i } \right] \left[ x _{ j } \mid a _{ j }  \right]  \right]   \) simply
   says that \( a _{ i }  \) is not the same element of \( A \) as \( a _{ j }
   \), and in total it claims that any two elements are different, thus they are
   all distinct, and so any model of \( \phi _{ k }  \) has at least \( k \)
   elements. Thus any model of \( \Phi  \) has more elements than any natural
   number, so that any model has a universe of infintely many elements.
\end{itemize}
\end{proof}




