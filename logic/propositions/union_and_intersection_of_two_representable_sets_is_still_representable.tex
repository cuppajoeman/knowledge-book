\begin{proposition}{Union and Intersection of two Representable Sets is still
Representable}{union_and_intersection_of_two_representable_sets_is_still_representable}
Suppose that \( A \) and \( B \) are representable sets, then \( A \cup B \) and
\( A \cap B \) are representable sets.
\end{proposition}
\begin{proof}
    Suppose that \( A \) and \( B \) are represented by \( \phi , \psi  \)
    respectively, then \( \phi \lor \psi  \) represents \( A \cup B \).
    Let \( a \in  A \cup  B \), we'll show that \( N \vdash \phi \left(
    \overline{a} \right) \lor \psi \left( \overline{a}  \right)  \) if \( \cdot
    a \in  A\) then we know that \( N \vdash \phi \left( \overline{a}  \right)
    \) in that case \( \phi \left( \overline{a}  \right) \lor \psi \left(
    \overline{a}  \right)  \) follows by PC. Note that if \( a \not\in A \) then
    it's in \( B \) and the \( \phi \left( \overline{a}  \right) \lor \psi
    \left( \overline{a}  \right)  \) follows in the exact same way, in all cases
    we've seen that \( N \vdash \phi \left( \overline{a}  \right) \lor \psi
    \left( \overline{a}  \right)   \) .
    So now we can suppose that \( a \not\in A \cup  B \) and we must show that \( N
    \vdash  \neg  \left[ \phi \left( \overline{a}  \right) \lor \psi \left(
    \overline{a}  \right) \right]   \) since \( a \not\in A \) and \( a \not\in
    B\), we know that we have both \( N \vdash \neg  \phi \left(
    \overline{a}  \right)  \land \neg \psi \left( \overline{a}  \right)  \)
    which is logically equivalent to \( \neg \left( \phi \left( \overline{a}
    \right) \lor \psi \left( \overline{a}  \right)  \right)  \) thus we have a
    deduction of it via PC.\\
    The proof of \( A \cap B \) is quite similar, so we'll move quicker, we're
    going to show that  \( \phi \land \psi  \) is the representation of \( A
    \cap B \) Suppose \( a \in  A \land B  \) then similarly (in an opposite
    way) to the second case of the above we can see that \( N \vdash \phi \left(
    \overline{a} \right) \land  \phi \left( \overline{a}  \right)  \) holds, for
    when we have \( a \not\in A \land B \) we get the same case work, so that if
    \( a \not\in A \) then \( N \vdash \neg \phi \left( \overline{a}  \right)
    \) making it possible to deduce \( \neg \left( \phi \left( \overline{a}
    \right) \land \psi \left( \overline{a}  \right)  \right)  \) by PC, the case
    for when \( a \not\in B \) is the same. \\
    For the converse, let's consider the set of all subsets of the naturals,
    namely \( \mathcal{ P }  \left( \mathbb{N}\right)  \), since this set has
    uncountably many elements and the fact that two different sets require two
    at least two different formulas, this set is not representable as if it were
    we would have uncountably many formulas which is a contradiction. Thus there
    is some subset \( X \subseteq N \) which isn't representable, and here we'll
    note that \( \mathbb{N} \setminus X \) is neither representable, for it if
    it was we could derive that \( X \) is representable by using the negation
    of the formula which represented \( \mathbb{N} \setminus X \), now \( X \)
    and \( \mathbb{N} \setminus X \) are not representable, but \( X \cup N
    \setminus X =  \mathbb{N}  \) and \( \mathbb{N}  \) is representable,
    additionally \( X \cap \mathbb{N} \setminus X = \varnothing  \) is
    representable so the converse doesn't hold.
\end{proof}
