\begin{proposition}
{Implying a Contradiction Means no Model}{implying_a_contradiction_means_no_model}
Let \(\Sigma \) be a set of sentences, then
\[
\Sigma \models \bot
\]
if and only if \(\Sigma \) has no model
\end{proposition}
\begin{proof}
    \begin{itemize}
        \item Recall that \(\Sigma \models \bot \) means that for any
            \(\mathcal{L}\)-Structure \(\mathfrak{A} \) we have that
            \[
            \mathfrak{A} \models \Sigma \enspace \text{implies} \enspace
            \mathfrak{A} \models \bot
            \]
            but recall that \(\mathfrak{A} \models \bot \) is always false,
            therefore for the implication to hold we require that \(\mathfrak{A}
            \models \Sigma \) to be false, which means \(\Sigma \) has no model
            as \(\mathfrak{A} \) was arbitrary.
        \item Suppose that \(\Sigma \) has no model, then that means for any
        \(\mathcal{L}\)-Structure \(\mathfrak{A} \) we have that \(\mathfrak{A}
        \not \models \Sigma \), so \( \Sigma \models \bot  \) vacuously holds.
    \end{itemize}
\end{proof}
