\begin{proposition}{The Set of Representable Functions is
Countable}{the_set_of_representable_functions_is_countable}
The set of representable functions from \( \mathbb{N} \to \mathbb{N}  \) is
countable
\end{proposition}
\begin{proof}
    Recall that there are countably many formulas of a language, therefore given
    any function that would be a na\"ive upperbound for the number of formulas \(
    \phi  \) which could represent it. 
    Consider two distinct functions \( f, g : \mathbb{N} ^{ k }  \to \mathbb{N}
    \), that means that there is at least one \( k \) tuple \( \undervec{a} \)
    where \( m = f\left( \undervec{a} \right) \neq g\left( \undervec{a} \right)  \),
    so that if \( \phi  \) represents \( f \) then since \( \left( \undervec{a},
    m\right)  \) is in the graph of \( f \) then \( N \vdash \phi \left(
    \undervec{a}, m \right)  \), but \( \left( \undervec{a}, m \right)  \) is
    not in the graph of \( g \) so if this \( \phi  \) also represents \( g \)
    then we would have \( N \vdash \neg \phi \left( \undervec{a}, m \right)  \)
    which would imply that \( N \) is inconsistent which is a contradiction.\\
    Therefore if we assume for the sake of contradiction that there are
    uncountably many representable functions, that would mean that there are
    uncountably many formulas of \( \mathcal{ L } _{ NT }   \) since given two
    functions they will they will each have at least one formula which is
    different representing it, this is a contradiction since there are countably
    many \( \mathcal{ L } _{ NT }   \) formulas.
\end{proof}
