\begin{proposition}
{Negation of a Sigma Formula is a Pi Formula}{negation_of_a_sigma_formula_is_a_pi_formula}
    Suppose \(\sigma \) is a \(\Sigma \)-formula, then \(\neg \sigma \) is equivalent
a \(\Pi \)-formula
\end{proposition}
\begin{proof}
We will proceed by induction on the complexity of the formula, in other words,
we will do structural induction
    \begin{itemize}
        \item Base Case
        \begin{itemize}
            \item Suppose \(\phi \) is an atomic \(\Sigma \)-formula, that means
            that it is an atomic formula in the sense of a normal formula,
            therefore it's negation is the negation of an atomic formula and is
            therefore a \(\Pi \)-formula
            \item Suppose \(\phi :\equiv \neg \alpha \) where \(\alpha \) is an
            atomic formula then by
            \hyperref[proposition:double_negation_equivalence]{the fact that
            double negation ``cancels''} we see that \(\phi \) is equivalent to
            \(\alpha \) which is an atomic formula, and therefore \(\alpha \) is
            a \(\Pi \)-formula
        \end{itemize}
        \item Induction Step
        \begin{itemize}
\item Suppose \(\alpha \) and \(\beta \) are \(\Sigma
        \)-formulas where the claim holds
            \begin{itemize}
                \item \(\phi :\equiv \alpha \lor \beta \), then \(\neg \phi \) is
                equivalent to \(\neg \alpha \land \neg \beta \) and therefore by
                our induction hypothesis we know that \(\neg \alpha \) and \(
                \neg \beta \) are equivent to some \(\Pi \)-formulas \(\pi _{
                \alpha}, \pi _{\beta} \) then \(\phi \) is equivalent to
                \(\pi _{\alpha} \land \pi _{\beta} \) which is a \(\Pi
                \)-formula by the third clause. \item \(\phi :\equiv \alpha
                \land \beta \) then \(\neg \phi \) is equivalent to \(\neg \alpha
                \lor \neg \beta \) and by the same argument above, we can see
                that this is a \(\Pi \)-formula \item \(\phi :\equiv \left(
                \forall x < t\right) \alpha \), then it's negation is \(\left(
                \exists x\right) \left(x < t \land \neg \alpha\right) \) by
                our induction hypothesis we know that \(\neg \alpha \) is
                equivalent to some \(\Pi \)-formula \(\pi _{\neg \alpha} \) so \(
                \left(\exists x\right) \left(x < t \land \neg \alpha \right) \)is
                equivalent to 
                \[
                \left(\exists x\right) \left(x < t \land \pi
                _{\neg \alpha}\right) :\equiv \left(\exists x < t\right) \pi
                _{\neg \alpha} 
                \]
                and therefore we may say that \(\phi \) is
                equivalent to a \(\Pi \)-formula.
                \item Suppose \(\phi :\equiv \left(\forall x \le t\right) \)
                then \(\neg \phi \) is equivalent to 
                \[
                    \left(\exists x\right)
                    \neg \left(\left(x < t \lor x = t\right) \rightarrow \alpha
                    \right) :\equiv \left(\exists x\right) \neg \left(\neg
                    \left(x < t \lor x = t\right) \lor \alpha\right) 
                \]
                which
                is equivalent to \(\left(\exists x\right) \left(\left(x < t
                \lor x = t\right) \land \neg \alpha\right) :\equiv \left(
                \exists x \le t\right) \neg \alpha \) and since \(\neg
                \alpha \) is equivalent to some \(\Pi \)-formula, \(\pi _{\neg
                \alpha} \) by our inductive hypothesis, we know that \(\left(
                \exists x \le t\right) \pi _{\neg \alpha} \) is a \(\Pi
                \)-formula.
                \item Suppose \(\phi :\equiv \left(\exists x < t\right)
                \alpha \) then the negation is equivalent to \(\left(\forall x
                \right) \neg \left(\neg \left(x < t\right) \lor \neg \alpha
                \right) :\equiv \left(\forall x\right) \left(x < t
                \rightarrow \neg \alpha\right) :\equiv \left(\forall x < t
                \right) \neg \alpha \), and following similar logic as above
                we can see that \(\left(\forall x < t\right) \neg \alpha \) is
                a \(\Pi \)-formula
                \item \(\phi :\equiv \left(\exists x \le t\right) \alpha \)
                the negation is equivalent to 
                \[
                \left(\forall x\right) \left(
                \neg \left(\left(x < t\right) \lor \left(x = t\right) \lor
                \neg \alpha\right)\right) 
                \]
                which is the same as
                \[
                \left(\forall x\right)
                \left(\left(x < t\right) \lor \left(x = t\right)\right)
                \rightarrow \neg \alpha :\equiv \left(\forall x \le t\right)
                \neg \alpha 
                \] as needed.
            \end{itemize}
        \end{itemize}
    \end{itemize}
\end{proof}
