\chapter*{Preface}
This book contains knowledge that that me or my peers have obtained, the purpose
is to explain things fundamentally and in full detail so that someone who has
never touched the subject may be able to understand it. It will focus on
conveying the ideas that are involved in synthesizing the new knowledge with
less of a focus on the results themselves.

It differs from a normal textbook in that it is open source and will fall under
more continuous development rather than having editions that periodically come
out. It also differs in the sense that it welcomes other users to improve the
book.

Here is a link to the book:
\url{https://raw.githubusercontent.com/cuppajoeman/knowledge-book/main/build/book.pdf}

\section*{Structure of book}
The book is partitioned into different sections based on the domain it is
involved with. There may be shared definitions and theorems throughout the
chapters, but in general it will start more elementary and get more advanced.

\section*{Knowledge}
\label{sec:knowledge}
In this book you will find many results, will will characterize them as being
one of the following
\begin{itemize}
    \item Theorems - Results that are of importantance and who's proof is not
    easily found (maybe using a novel idea)
    \item Propositions - Results of less importance who's proof could be
    constructed without a novel idea
    \item Lemmas - Results that are technical intermediate steps which has no
    standing as an independent result on first observation \footnote{But
    sometimes they escape, as their usage becomes more than just an intermediate
    step,  as Zorn's or Fatou's Lemmas did}
    \item Corollaries - Results which follow readily  from an existing result of
    greater importance
\end{itemize}

\section*{Recommendations}
\subsection*{Skill}
By now you might know that in order to actually get better at mathematics you
have to engage with it. This book may be used as a reference at times, but I
highly recommend trying to re-prove statements or coming up with your own ideas
before instantly looking at the solutions.
\subsection*{Reading}
When reading this book, being able to jump back and forth between sections of
the book will allow you to move through content much more quickly, to this end,
we recommand the use of the pdf viewer zathura, which emulates vim commands
allowing a user to completely operate on the pdf with keyboard.

\section*{About the companion website}
The website\footnote{\url{https://github.com/cuppajoeman/knowledge-book}} for
this file contains:
\begin{itemize}
  \item A link to (freely downlodable) latest version of this document.
  \item Link to download \LaTeX source for this document.
\end{itemize}

\section*{Acknowledgements}
\begin{itemize}
    \item A special word of thanks to professors who wanted to make sure I
    understood and learned as much as possible Alfonso
    Gracia-Saz\footnote{\url{https://www.math.toronto.edu/cms/alfonso-memorial/}},
    Jean-Baptiste
    Campesato\footnote{\url{https://math.univ-angers.fr/~campesato/}}, Valentine
    Chiche-Lapierre and Gal Gross\footnote{\url{https://www.galgr.com/}}
    \item Thanks to Z-Module, riv, PlanckWalk, franciman, qergle from \#math on
    \url{https://libera.chat/}.
\end{itemize}

\section*{Scaling}
As this project becomes larger eventually there will be a collision in terms of
files and how we can reference things with labels, at that point work will have
to be done to make sure that labls and file names are unique by using UUID's.


\chapter*{Contributing}
 
Contributions to the project are very welcome, let's delve into how to get
started with this.

Here is a list of things that can be worked on:

\begin{itemize}
  \item Content Based 
  \begin{itemize}
      \item Adding Definitions, Theorems, \ldots
      \item Finishing TODO's
      \item Formatting of the book
  \end{itemize}
  \item Structural Layout of Project 
  \begin{itemize}
      \item Organization
      \item Simplyfing the existing structure of the directories 
      \item Making scripts which set up new structures
  \end{itemize}
  \item External
  \begin{itemize}
      \item Adding explanatory content to help onboard new users
      \item Getting others involved
      \item Creating infrastructure to support users (Github discussions)
  \end{itemize}
\end{itemize}

\section*{Example}
I will describe how to do this in detail in the below paragraphs, here is the
TLDR of them. Let's say you have a theorem you'd like to add, here is a quick
outline of what you have to do. Fork the project,  find the structure that your
theorem belongs in, one possible structure could be first order logic, then go
ahead and copy the blank \texttt{theorem.tex} file to
\texttt{my\_new\_theorem.tex}, after that write up your theorem, include it in
that structures content file, compile the project and verify it's there, then
push to your forked repository and create a pull request against the main
repository on github.

\section*{Setup for Contributing}
In order to contribute to this project you will need to understand the basics of
git, \LaTeX and the structure of the project.

\subsection*{Git}
If you aren't familiar with what git is, I recommend you watch the videos on the
following page \url{https://git-scm.com/doc}, and remember the link to the docs
for future reference. 

Next you can make a github account on \url{github.com}, the way you will get a
copy of the project is by forking the project. To do that visit
\url{https://github.com/cuppajoeman/knowledge-book} and click fork in the top
right corner, once you've forked it we can now download it to our computer.

If you have access to a terminal then you can do the following:

\begin{term}
git clone <link gotten from your forked repo>
\end{term}

From here on out you will make changes to the project and make commits using
git. Once you're happy with the changes you've made, you can save your git
commits online by running

\begin{term}
git push
\end{term}

Otherwise you can download \url{https://desktop.github.com/}. And after you've
made edits to the book, you can push, then finally then go to your forked
version of the project and make a pull request into the main project. To do this
visit your forked version of the project on github and navigate to the pull
requests tab, then make sure the base repository is
\texttt{cuppajoeman/knowledge-book} and add relevant info explaining the pull
request. Once someone has reviewed your changes it will be merged into the main
project.

\subsection*{\LaTeX}

This document is entirely written in \LaTeX which is a language which lets us
format mathematics and make figures easily.

If this system is entirely new to you, then don't worry, you'll be able to pick
up most of what you need by looking at examples being used in this project,
otherwise take a look at
\url{http://www.docs.is.ed.ac.uk/skills/documents/3722/3722-2014.pdf}.

To be able to \LaTeX up and running, it is dependent on the operating system you
are using, but the one thing that remains invariant is that you will need a way
to compile it and a way to edit it.


\subsubsection*{Compilers}
\begin{enumerate}
    \item \url{https://miktex.org/} - windows
    \item \url{https://tug.org/mactex/} - mac
    \item \url{https://www.tug.org/texlive/} - linux 
    \begin{itemize}
        \item Note this will probably be available in your distributions package
        repositories.
    \end{itemize}
\end{enumerate}

\subsubsection*{Editors}
\begin{enumerate}
   \item \url{https://tug.org/texworks/} - minimal
   \item \url{https://tug.org/texworks/},
   \url{https://www.xm1math.net/texmaker/} - more fully featured
    \item vim, emacs, other terminal based editors
    \begin{itemize}
        \item Allows you to be the most efficient
    \end{itemize}
\end{enumerate}

Once you've obtained an editor, if it has some type of built-in compiler then
you will want to change the output directory to be the build directory. This
makes sure that when you compile the document that the root directory of the
project isn't filled with irrelevant files. For each editor it will be
different, but the general setup will be to get to some sort of configuration
panel (or file) and to change the editors compile command to this:
\texttt{pdflatex --output-directory=build}. Some editors might have a built in
mode for this, for example with tex maker one can tick the ``use build directory
for output files", for users who use vim, there is a built in option for vimtex
like so:

\begin{lstlisting}
let g:vimtex_compiler_latexmk = {
            \ 'build_dir' : 'build',
            \}
\end{lstlisting}

 
\section*{Communication}

Feel free to ask any question on the discord server:
\url{https://discord.gg/ReceZrGuN6}, it will also be a place for general
discussion. Another option we can look at is github discussions.

\section*{Content Based}

To understand how to add content to the project, the best step is to understand
fundamentally what a structure is.

A \textbf{structure} is a directory which has the following :
\begin{itemize}
    \item content.tex
    \begin{itemize}
        \item This file includes all the content in the other directories of
        this directory
    \end{itemize}
    \item definitions
    \item theorems
    \item lemmas
    \item lemmas
    \item corollaries
    \item sub\_structures
    \begin{itemize}
        \item This directory contains other \textbf{structures}
    \end{itemize}
\end{itemize}

Note that the other directories; definitions, theorems, lemmas, lemmas,
corollaries are non-recursive directories that contain content (Let's call them
content directories from now on) \hyperref[sec:knowledge]{see here} for exactly
what these directories contain. 

If you want to add a new structure, the best thing to do is to verify with other
members of the project if it warrants it's own structure, otherwise it can be
added as a substructure of an existing one.

Supposing that you are on a unix based operating system, then here is how one
could create a new structure:

\subsection*{Manually}

\begin{term}
cp -r structure new_structure
cd new_structure
mv content.tex new_structure.tex
nvim new_structure.tex
\end{term}

Otherwise if you're adding a new theorem, it could be:

\begin{term}
cd existing_structure/theorems
cp theorem.tex my_new_theorem.tex
nvim my_new_theorem.tex
...
git add -A && git commit -m "add my new theorem" && git push
\end{term}

\subsection*{Using the Script}
Alternatively for a more streamlined experience we can use some of the scripts
we have for creating new content. (Note that right now this script is only for
creating new content files, but eventually will create new structures as well)

To get started make sure you have \texttt{fzf} installed and the python package
\texttt{pyperclip}.

First to learn about the script run:

\begin{term}
python scripts/main.py -h 
\end{term}

And once you have read this we may create a new theorem like

\begin{term}
python scripts/main.py t "My new Theorem" -c 
\end{term}

Which asks you for which structure the theorem should belong to and creates the
file for us, the -c option copies $\texttt{\\input\{\textit{path to new
theorem}/my\_new\_theorem\}}$ to the clipboard so that it may be included in the
main content file associated with the structure.

\subsection*{With gui}
If you're on windows or another operating system without access to terminal, you
can always take the directory \texttt{structure} and copy it to a new name. To
make new content file, go to the revelent content directory and copy the blank
bootsrap file, that is if you wanted to make a new definition you can go to the
structures definitions folder and then copy \texttt{definition.tex} to
\texttt{my\_new\_definition.tex} and start working on it.


