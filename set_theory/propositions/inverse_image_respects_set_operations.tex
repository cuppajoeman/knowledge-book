\begin{proposition}{Inverse Image Respects set
Operations}{inverse_image_respects_set_operations}
Let \( f : A \to B \) and \( S _{ \alpha  }, \alpha \in \mathcal{ J }   \) be
subsets of \( B \), then for \( \texttt{X} \in  \left\{ \bigcup , \bigcap  \right\}  \) 
\[
 \underset{ \alpha \in  \mathcal{ J }   }{ \texttt{ X }  }  f ^{-1} \left( S _{
 \alpha  }  \right) = f ^{-1} \left( \underset{ \alpha \in \mathcal{ J }   }{
 \texttt{ X }  }  S _{ \alpha  }  \right) 
\]
It also respects set difference so that for any \( X, Y \subseteq B \) we have
\[
f ^{-1} \left( X \setminus Y \right) =  f ^{-1} \left( X \right) \setminus f
^{-1} \left( Y \right) 
\]
\end{proposition}
\begin{proof}
    Consider the following string:\\
    \( S\left( \texttt{ X } , \texttt{ Y }  \right) = \) ``
    \begin{gather*}
        x \in \underset{ \alpha \in \mathcal{ J }   }{ \texttt{ X }  }  f ^{-1} \left( S _{ \alpha  }
        \right) \\
        \Updownarrow \\
        \underset{ \alpha \in \mathcal{ J }   }{ \texttt{ Y }  } x \in  f ^{-1} \left( S _{ \alpha
        }  \right)  \\
        \Updownarrow \\
        \underset{ \alpha \in \mathcal{ J }   }{ \texttt{ Y }  } f  \left( x \right) \in  S _{
        \alpha } \\
        \Updownarrow \\
        f  \left( x \right) \in  \underset{ \alpha \in \mathcal{ J }   }{
        \texttt{ X }  }  \\
        \Updownarrow \\
        x \in  f ^{-1} \left( \underset{\alpha \in \mathcal{ J }}{\texttt{ X } } \right) 
    \end{gather*}
    ''\\
    Notice that  \( S\left( \bigcup , \bigvee  \right)  \) is a proof of \( \bigcup
    _{ \alpha  \in \mathcal{ J }   } f ^{-1} \left( S _{ \alpha  }  \right) = f
    ^{-1} \left( \bigcup _{ \alpha  \in  \mathcal{ J }   } S _{ \alpha  }
    \right) \) and \( S\left( \bigcap , \bigwedge  \right)  \) is a proof of \( \bigcap
    _{ \alpha  \in \mathcal{ J }   } f ^{-1} \left( S _{ \alpha  }  \right) = f
    ^{-1} \left( \bigcap _{ \alpha  \in  \mathcal{ J }   } S _{ \alpha  }
    \right) \)\\
    As for set difference consider \( k \in f ^{-1} \left( X \setminus Y \right)
    \) that's true if and only if \( f\left( k \right) \in X \setminus Y\) if
    and only if  \( f\left( k \right) \in  X \) and \( f\left( k \right) \not\in
    Y \) equivalently that is \( k \in f ^{-1} \left( X \right)  \) and \( k \in
    f ^{-1} \left( Y \right)  \) so then \( k \in  f ^{-1} \left( X \right)
    \setminus f ^{-1} \left( Y \right)  \) so that \( k \in  f ^{-1} \left( X
    \setminus Y \right) \) if and only if  \(  k \in f ^{-1} \left( X \right)
    \setminus f ^{-1} \left( Y \right)  \) so \( f ^{-1} \left( X \setminus Y
    \right) = f ^{-1} \left( X \right)  \setminus f ^{-1} \left( Y \right)  \),
    as needed.
\end{proof}
