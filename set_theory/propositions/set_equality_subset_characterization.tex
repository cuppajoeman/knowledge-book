\begin{proposition}{Set Equality Subset Characterization}{set_equality_subset_characterization}
    Two sets \( A, B \) are equal if and only if \( A \subseteq B \land B
    \subseteq A \)
\end{proposition}
\begin{proof}
    Suppose \( A \subseteq B \land  B \subseteq A \), then for the sake of
    contradiction assume without loss of generality that there is an element \(
    x \in  A\) but that \( x \not\in B \), we assumed \( A \subseteq B \) so
    since \( x \in  A \) it implies that \( x \in  B \) which is a
    contradiction.\\
    Now suppose that \( A = B \) but that it's false that \( A \subseteq B \land
    B \subseteq A\) without loss of generality we assume that \( A \subseteq B
    \) is false, which by definition means that there is some \( a \in  A \)
    such that \( a \not\in B \), but then \( A \neq B \) which is a
    contradiction.
\end{proof}
