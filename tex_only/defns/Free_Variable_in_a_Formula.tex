\begin{defn*}{Free Variable in a Formula}
    Suppose that $v$ is a variable and $\phi$ is a formula. We will say that $v$ is free in $\phi$ if
    \begin{enumerate}
        \item $\phi$ is atomic and $v$ occurs in (is a symbol in) $\phi$, or                               
        \item $\phi: \equiv(\neg \alpha)$ and $v$ is free in $\alpha$, or
        \item $\phi: \equiv(\alpha \vee \beta)$ and $v$ is free in at least one of $\alpha$ or $\beta$, or
        \item $\phi: \equiv(\forall u)(\alpha)$ and $v$ is not $u$ and $v$ is free in $\alpha$.
    \end{enumerate}
    \subsubsection*{Examples}
    \begin{itemize}
        \item Thus, if we look at the formula
        $$
        \forall v_{2} \neg\left(\forall v_{3}\right)\left(v_{1}=S\left(v_{2}\right) \vee v_{3}=v_{2}\right)
        $$
        the variable $v_{1}$ is free whereas the variables $v_{2}$ and $v_{3}$ are not free. 
        \item A slightly more complicated example is
            \[
            \left(\forall v_{1} \forall v_{2}\left(v_{1}+v_{2}=0\right)\right) \vee v_{1}=S(0)
            \]
            \begin{itemize}
                \item In this formula, $v_{1}$ is free whereas $v_{2}$ is not free. Especially when a formula is presented informally, you must be careful about the scope of the quantifiers and the placement of parentheses.
            \end{itemize}
    \end{itemize}
    \subsubsection*{Notes}
    \begin{itemize}
        \item We will have occasion to use the informal notation $\forall x \phi(x)$. This will mean that $\phi$ is a formula and $x$ is among the free variables of $\phi$. 
        \item If we then write $\phi(t)$, where $t$ is an $\mathcal{L}$-term, that will denote the formula obtained by taking $\phi$ and replacing each occurrence of the variable $x$ with the term $t$. 
    \end{itemize}

\end{defn*}
