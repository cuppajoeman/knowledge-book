\begin{proposition}
{Topology Generated by a Basis (Union) is a Topology}
    \[
    \mathcal{T}_{\mathcal{B}} \text{ is a topology}
    \]
    \begin{pf}
        \begin{itemize}
            \item \(\varnothing \in \mathcal{T}_{\mathcal{B}} \) is true because \(\varnothing \subseteq \mathcal{B} \) and \(\bigcup \varnothing = \varnothing \). It's also clear that \(X \in \mathcal{T}_{B} \) since from the definition of basis we know that \(X = \bigcup \mathcal{B} \)
            \item We'll show that \(\mathcal{T} _{\mathcal{B}} \) is closed under arbitrary unions
            \begin{itemize}
                \item Let \(V_{\alpha}, \alpha \in I \) be elements of \(\mathcal{T} _{\mathcal{B}} \) for some indexing set \(I \), one must show that
                    \[
                    \bigcup _{\alpha \in I} V_{\alpha} \in \mathcal{T} _{\mathcal{B}}
                    \]
                \item By the definition of \(\mathcal{T} _{\mathcal{B}} \) for each \(\alpha \) we have \(\mathcal{C}_ {\alpha} \) such that \(V_{\alpha} = \bigcup \mathcal{C} _{\alpha} \), therefore
                    \[
                    \bigcup _{\alpha \in I} V_{\alpha} = \bigcup _{\alpha \in I} \left(\bigcup \mathcal{C} _{\alpha}\right) = \bigcup _{\alpha \in I} \left(\bigcup _{X \in \mathcal{C} _{\alpha}} X\right) = \bigcup _{X \in \left(\bigcup _{\alpha \in I} \mathcal{C} _{\alpha}\right)} X = \bigcup \left(\bigcup _{\alpha \in I} \mathcal{C} _{\alpha}\right)
                    \]
            \end{itemize}
            \item $ \mathcal{ T } _{ \mathcal{ B }  }  $ is closed under finite intersections, we'll show it's closed under pairwise intersections and the general result may be proven inductively.
            \begin{itemize}
                \item Suppose $ U = \bigcup \mathcal{ A }  $ and $ V =  \bigcup \mathcal{ C }  $ are elements of $ \mathcal{ T } _{ \mathcal{ B }  }  $ we must show that $ U \cap V \in  \mathcal{ T } _{ \mathcal{ B }  }  $ 
                    \[
                    U \cap  V =  \left( \bigcup \mathcal{ A }  \right) \cap  \left( \bigcup \mathcal{ C }  \right) =  \bigcup \left\{ A \cap C: \text{ where } A \in  \mathcal{ A } , C \in  \mathcal{ C }    \right\} 
                    \]
                    where the last equality comes from the fact that if a point is in the intersection of $ \bigcup \mathcal{ A }  $ and $ \bigcup \mathcal{ C }  $ then it must be in the intersection of some $ A \in  \mathcal{ A }  $ and some $ C \in  \mathcal{ C }  $
                \item To show this is an element of $ \mathcal{ T } _{ \mathcal{ B }  }  $ all we have to do is to show that the set $\left\{ A \cap C: \text{ where } A \in  \mathcal{ A } , C \in  \mathcal{ C }    \right\} \subseteq \mathcal{ T } _{\mathcal{ B }  }   $ and then use the fact that $ \mathcal{ T } _{ \mathcal{ B }  }  $ is closed under arbitrary unions.
                \begin{itemize}
                    \item Let $ J \cap  K $ be from the set under consideration and now our goal is to show that $ J \cap  K  \in  \mathcal{ T } _{  \mathcal{ B }  } $, that is we have some $ \mathcal{ R }  $ such that $ J \cap K =  \bigcup \mathcal{ R }  $ 
                    \item Recall that $ J \in  \mathcal{ A } \subseteq \mathcal{ B } \subseteq P\left( X \right)  $ therefore $ J \subseteq X $ and by the same logic $ K \subseteq X $ therefore $ J \cap  K \subseteq X $.
                    \item Thus for every $ x \in  J \cap  K $ we have $ B_{ x }  $ such that $ x \in  B_{ x } \subseteq J \cap K $ from the definition of basis for a set, so we have
                        \[
                        J \cap K \subseteq \left[ \bigcup _{ x \in  J \cap K } B_{ x }  \right] \subseteq J \cap K
                        \]
                        Since each $ B_{ x } \subseteq J \cap K $ 
                    \item In other words
                        \[
                        J \cap K = \left[ \bigcup _{ x \in  J \cap K } B_{ x }   \right] = \bigcup \left\{ B_{ x } : x \in  J \cap K \right\} 
                        \]
                    \item And finally noting that $ \left\{ B_{ x } : x \in J \cap K \right\} \subseteq \mathcal{ B }  $ 
                \end{itemize}
            \end{itemize}
        \end{itemize}
    \end{pf}
\end{proposition}
